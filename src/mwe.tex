\documentclass{book}

\usepackage{ifthen}

\usepackage{tikz}
\usetikzlibrary{shapes.geometric}
\usetikzlibrary{math}
\usepackage{listofitems}

%-----------------------------------------------------------
% Drawing Stones

% From [this answer by @DavidCarlisle](https://tex.stackexchange.com/a/708876/64441).
\newcommand\notwhite{black}
\newcommand\notblack{white}

% From [this answer by @DavidCarlisle](https://tex.stackexchange.com/a/708893/64441).
\ExplSyntaxOn
  \cs_generate_variant:Nn \int_from_alph:n {e}

  \NewExpandableDocumentCommand{\stringToCoordX}{ m }{
    \int_from_alph:e { \use_i:nn #1 }
  }
  \NewExpandableDocumentCommand{\stringToCoordY}{ m }{
    \boardSize + 1 - ~\int_from_alph:e { \use_ii:nn #1 }
  }
\ExplSyntaxOff

\newcommand{\setCoords}[1]{
  \pgfmathsetmacro{\x}{\stringToCoordX{#1} - 1}
  \pgfmathsetmacro{\y}{\stringToCoordY{#1} - 1}
}

\newcounter{moveCounter}
\setcounter{moveCounter}{1}

\newcommand{\stepMoveCounter}{
  \stepcounter{moveCounter}
}

%-----------------------------------------------------------
% Drawing Stones and Moves

\newcommand{\drawStoneFromSgfCoords}[2]{%
  \setCoords{#2}

  \draw[draw = \UseName{not#1}, fill = #1, line width = 0.1mm]
    (\x * \step, \y * \step)
    circle [radius = 0.2575cm];
}

\newcommand{\drawMoveFromSgfCoords}[2]{
  \expandafter\xdef\csname thecolorat#2\endcsname{#1}% SAVE COLOR AT #2
  \drawStoneFromSgfCoords{#1}{#2}
  \textLabel{#1}{#2}{\themoveCounter}

  \stepMoveCounter
}

%-----------------------------------------------------------
% Labels

\newcommand\StoneColor[1]{% COMPLEMENTARY COLOR OF REFERENCED STONE
  \ifcsname thecolorat#1\endcsname
    \csname thecolorat#1\endcsname
  \else
    white% COMPLEMENT OF DEFAULT LABEL COLOR WHEN NOT ATOP EXISTING STONE
  \fi
}

\newcommand{\textLabel}[3]{%
  \setCoords{#2}%

  \draw (\x * \step, \y * \step) 
  node[
    color     = -\StoneColor{#2},
    inner sep = 1pt
  ] {#3};
}

\newcommand{\triangleLabel}[2]{
  \setCoords{#2}

  \draw (\x * \step, \y * \step) 
    node[
      isosceles triangle,
      draw                          = #1,
      color                         = -\StoneColor{#2},
      fill                          = \StoneColor{#2},
      minimum height                = \step * 10,
      minimum width                 = \step * 10,
      line width                    = 0.5mm,
      rotate                        = 90,
      isosceles triangle apex angle = 60,
      inner sep                     = 0pt,
    ] {};
}

%-----------------------------------------------------------
% SGF Parser

% From [this answer by @StevenB.Segletes](https://tex.stackexchange.com/a/709014/64441).
\long\def\Firstof#1#2\endFirstof{#1}

\newcommand\thecolorofB{black}
\newcommand\thecolorofAB{black}
\newcommand\thecolorofW{white}
\newcommand\thecolorofAW{white}
\newcommand\thecolorofMA{white}
\newcommand\thecolorofCR{white}
\newcommand\thecolorofTR{white}
\newcommand\thecolorofSQ{white}
\newcommand\thecolorofLB{white}

\long\def\Keytypeof#1{\csname thekeytypeof#1\endcsname}

\newcommand\thekeytypeofB{M}  % black move
\newcommand\thekeytypeofAB{A} % added (edited) black stone
\newcommand\thekeytypeofW{M}  % white move
\newcommand\thekeytypeofAW{A} % added (edited) white stone
\newcommand\thekeytypeofMA{K} % cross (mark) label
\newcommand\thekeytypeofCR{C} % circle label
\newcommand\thekeytypeofTR{T} % triangle label
\newcommand\thekeytypeofSQ{S} % square label
\newcommand\thekeytypeofLB{L} % text label

\newcommand{\parseSgf}[1]{%
  \setsepchar{(||)/;/]/[/:}%
  \readlist*\Z{#1}%

  \foreachitem \Branch \in \Z[]{%
  \foreachitem \Group \in \Z[\Branchcnt]{%
    \foreachitem \Key \in \Z[\Branchcnt, \Groupcnt]{%
     \if\relax\Key\relax% IF BLANK KEY & VALUE, SKIP
     \else
      \itemtomacro\Z[\Branchcnt, \Groupcnt, \Keycnt, 1]\KeyName
      \if\relax\KeyName\relax% IF VALUE, BUT NO KEYNAME, USE RECENT KEYNAME
        \let\KeyName\MostRecentKeyname
      \else
        \xdef\MostRecentKeyname{\KeyName}%
      \fi
      \itemtomacro\Z[\Branchcnt, \Groupcnt, \Keycnt, 2, 1]\KeyValue

      \edef\tmp{{\csname thecolorof\KeyName\endcsname}{\KeyValue}}%

      \if\Keytypeof\KeyName M
        \expandafter\drawMoveFromSgfCoords\tmp
      \fi
      \if\Keytypeof\KeyName A
        \expandafter\drawStoneFromSgfCoords\tmp
      \fi
      \if\Keytypeof\KeyName K
        \expandafter\crossLabel\tmp
      \fi
      \if\Keytypeof\KeyName C
        \expandafter\circleLabel\tmp
      \fi
      \if\Keytypeof\KeyName T
        \expandafter\triangleLabel\tmp
      \fi
      \if\Keytypeof\KeyName S
        \expandafter\squareLabel\tmp
      \fi
      \if\Keytypeof\KeyName L
        \expandafter\textLabel\tmp{\Z[\Branchcnt, \Groupcnt, \Keycnt, 2, 2]} 
      \fi
     \fi
    }
  }%
  }%
}

%-----------------------------------------------------------
% Setup

\pgfmathsetmacro{\boardDimension}{10}
\pgfmathsetmacro{\boardSize}{19}
\pgfmathsetmacro{\step}{\boardDimension / (\boardSize - 1)}

%-----------------------------------------------------------

\usepackage{catchfile}

% From [this question/answer](https://tex.stackexchange.com/a/726141/64441)
\newcommand\parseSgfFile[1]{%
  \CatchFileDef{\mysgf}{#1}{}%
  \parseSgf{\mysgf}% 
}

%-----------------------------------------------------------
% Grid

\newcommand{\calculateStep}{
  \pgfmathsetmacro{\step}{\boardDimension / (\boardSize - 1)} % chktex 1
}

% From [this answer by @UlrichDiez](https://tex.stackexchange.com/a/709341/64441).
\pgfkeys{%
  %---------------------------------------------------------
  /phili/goGrid/.cd, 
    %-------------------------------------------------------
    % Dimensions
    board dimension/.store in    = \boardDimension,
    board dimension              = 10cm,
    board size/.store in         = \boardSize,
    board size                   = 19,
    %-------------------------------------------------------
    % Outline
    outline line width/.store in = \boardOutlineWidth,
    outline line width           = 0.7mm,
  %---------------------------------------------------------
  /phili/goban/.search also={/phili/goGrid},
  /phili/goban/.cd,  
    %-------------------------------------------------------
    % Scale
    scale/.store in              = \scale,
    scale                        = 1,
    %-------------------------------------------------------
    % Partial Boards
    horizontal clip start/.store in = \horClipStart,
    horizontal clip start           = -1,
    horizontal clip end/.store in   = \horClipEnd, % This is more like the width?
    horizontal clip end             = -1,
    vertical clip start/.store in   = \verClipStart,
    vertical clip start             = -1,
    vertical clip end/.store in     = \verClipEnd, % This is more like the height?
    vertical clip end               = -1,
  %---------------------------------------------------------
}

% Parameters
%
% - `board dimension` (in cm)
% - `board size` (square)
% 
% Example: A 19x19 board with size 10cm x 10cm:
%
% ```tex
% \goGrid[board dimension    = 10,
%         board size         = 9,
%         scale              = 1,
%         outline line width = 0.5mm]
% ```
\newcommand{\goGrid}[1][]{
  \pgfkeys{/phili/goGrid/.cd, #1}

  \calculateStep

  \draw[step=\step] (0, 0) grid
    (\boardDimension, \boardDimension);
  
  \boardOutline{\boardDimension}

  \drawHoshis
}

% Reference: [Drawing a Non-Jagged Grid Outline](https://tex.stackexchange.com/q/709298/64441)
%
% Parameters
%
% 1: dimension (in cm)
% 
% Example: `\boardOutline{10}`
\newcommand{\boardOutline}[1]{
  \draw[step       = #1,
        line width = \boardOutlineWidth,
        line cap   = rect] 
    (0, 0) grid (#1, #1);
}

% Example: A 19x19 board with size 10cm x 10cm: `\drawHoshis`
\newcommand{\drawHoshis}{
  \tikzmath{
    \hoshiRadius = \step * 0.125;
    %
    \centerHoshi = ceil(\boardSize / 2);
    %
    int \hoshiDistance;
    if \boardSize<12 then {
      \hoshiDistance = 3;
    } else {
      \hoshiDistance = 4;
    };
    %
    \hoshiComplement = \boardSize - \hoshiDistance + 1;
  }

  \drawCenterHoshi
  \drawCornerHoshis
  \ifnum\boardSize>6\relax
    \drawCornerHoshis
  \fi
  \ifthenelse{\isodd{\boardSize}}{
    \ifnum\boardSize>13\relax
      \drawSideHoshis
    \fi
  }{}
}

\newcommand{\drawCenterHoshi}{
  \pgfmathsetmacro{\centerHoshiCoord}{(\centerHoshi - 1) * \step}

  \filldraw (\centerHoshiCoord, \centerHoshiCoord)
    circle [radius=\hoshiRadius];
}

\newcommand{\drawCornerHoshis}{
  \def\cornerHoshisArray{%
    {\hoshiDistance, \hoshiDistance},%
    {\hoshiComplement, \hoshiDistance},%
    {\hoshiDistance, \hoshiComplement },%
    {\hoshiComplement, \hoshiComplement}%
  }

  \loopOverHoshis{\cornerHoshisArray}
}

\newcommand{\loopOverHoshis}[1]{
  \foreach \sloc in #1 {
    \pgfmathsetmacro{\hoshiCoordX}{\step * ({\sloc}[0] - 1)}
    \pgfmathsetmacro{\hoshiCoordY}{\step * ({\sloc}[1] - 1)}

    \filldraw (\hoshiCoordX, \hoshiCoordY)
      circle [radius=\hoshiRadius];
  }
}

\newcommand{\drawSideHoshis}{
  \def\sideHoshisArray{%
    {\hoshiDistance, \centerHoshi},%
    {\centerHoshi, \hoshiComplement},%
    {\centerHoshi, \hoshiDistance},%
    {\hoshiComplement, \centerHoshi}%
  }    

  \loopOverHoshis{\sideHoshisArray}
}

%-----------------------------------------------------------
% Goban Env

\newcommand\partialBoardClipping{
  \ifthenelse{
    \equal{\horClipStart}{-1} \AND
    \equal{\horClipEnd}{-1} \AND
    \equal{\verClipStart}{-1} \AND
    \equal{\verClipEnd}{-1}
  }{}{
    \tikzmath{
      \horStart = (-1.5 + \horClipStart) * \step;
      \horEnd   = (\horClipEnd) * \step;
      \verStart = (-1.5 + \verClipStart) * \step;
      \verEnd   = (\verClipEnd) * \step;
    }

    \clip (\horStart, \verStart) rectangle 
          (\horEnd, \verEnd);
  }
}

\newenvironment{goban}[1][]{
  \pgfkeys{/phili/goban/.cd, #1}

  \begin{tikzpicture}[scale = \scale,
                      transform shape]
    \calculateStep

    \partialBoardClipping

    \goGrid
}{
    \setcounter{moveCounter}{1}
  \end{tikzpicture}
}

%-----------------------------------------------------------
% Problem Diagram

\newcounter{problemCounter}
\setcounter{problemCounter}{0}

\newcommand\problemNumber[1]{
  \pgfmathsetmacro{\baselinePad}{
    \ifnum\theproblemCounter>9 
      2.25pt 
    \else
      1.75pt
    \fi
  }

  \pgfmathsetmacro{\innerSep}{
    \ifnum\theproblemCounter>9 
      2.5pt 
    \else
      4pt
    \fi
  }

  \begin{tikzpicture}[baseline=\baselinePad]
    \fontfamily{lmr}

    \node[draw,
          shape        = circle,
          inner sep    = \innerSep,
          line width   = 0.025cm,
          minimum size = 0.45cm] 
      (char)
      {\textmd{#1}};
  \end{tikzpicture}
}

\newcommand\problemDiagram[3]{
  \stepcounter{problemCounter}

  \begin{figure}[t]
    \centering

    \vspace{-0.5cm}

    \hbox{
      \begin{goban}[board dimension       = 14,
                    board size            = 19,
                    scale                 = 1,
                    outline line width    = 0.55mm,
                    horizontal clip start = 9,
                    horizontal clip end   = 19,
                    vertical clip start   = 9,
                    vertical clip end     = 19]
        \parseSgfFile{sgf/#1/#2.sgf}
      \end{goban}
    }

    \vspace{0.675cm}
    
    \begin{minipage}[t]{0.2\linewidth}
      \hbox{
        \hspace{0.15cm}
        \problemNumber{\theproblemCounter}
      }
    \end{minipage}\hfill
    \begin{minipage}[t]{0.7\linewidth}
      #3
    \end{minipage}\hspace{0.7cm}
  \end{figure}
}

%-----------------------------------------------------------
% Answer Diagrams

\newcounter{answerCounter}
\setcounter{answerCounter}{0}

\newcommand\answerDiagram[3]{
  \stepcounter{answerCounter}
  
  \begin{figure}
    \vspace{0.5cm}

    \begin{minipage}[c]{0.5\linewidth}
      \begin{goban}[board dimension       = 8.3,
                    board size            = 19,
                    scale                 = 1,
                    outline line width    = 0.425mm,
                    horizontal clip start = 11,
                    horizontal clip end   = 19,
                    vertical clip start   = 11,
                    vertical clip end     = 19]
        \parseSgfFile{sgf/#1/#2.solution.\theanswerCounter.sgf}
      \end{goban}
    \end{minipage}\hfill
    \begin{minipage}[c]{0.49\linewidth}
      #3
    \end{minipage} 
  \end{figure}
}

\newcommand\answerDiagrams[4]{
  \setcounter{answerCounter}{1}
  
  % \answerDiagram{#1}{#2}{#3}
  \answerDiagram{#1}{#2}{#4}
}

%-----------------------------------------------------------
% Diagrams

\usepackage[labelformat = simple]{subfig}
\renewcommand{\thesubfigure}{\relax} 

\newcommand\problemAnswerDiagram[5]{
  \problemDiagram
    {#1}
    {#2}
    {#3}
  \answerDiagrams
    {#1}
    {#2}
    {#4}
    {#5}
}

%-----------------------------------------------------------
% Utils

\newcommand\emptypage{
  \newpage 
  \ 
  \newpage
}

% An unnumbered page.
% From [this answer](https://tex.stackexchange.com/a/331068/64441).
\newcommand\blankpage{
  \pagenumbering{gobble}
  \clearpage
  \begingroup
    \null
    \thispagestyle{empty}%
    \addtocounter{page}{-1}%
    \hypersetup{pageanchor=false}%
    \clearpage
  \endgroup
}

\newcommand\shref[2]{%
  \normalsize\href{#1}{\path{#2}}\normalsize%
}

%-----------------------------------------------------------

\begin{document}
  % \begin{tikzpicture}
  %   \draw[step=\step] (0, 0) grid (10, 10);
  %   \parseSgfFile{sgf/captura/captura.9.solution.1.sgf}
  % \end{tikzpicture}

  % \begin{tikzpicture}
  %   \draw[step=\step] (0, 0) grid (10, 10);
  %   \parseSgfFile{sgf/captura/captura.7.solution.2.sgf}
  % \end{tikzpicture}

  % \begin{goban}[board dimension       = 14,
  %               board size            = 19,
  %               scale                 = 1,
  %               outline line width    = 0.55mm,
  %               horizontal clip start = 9,
  %               horizontal clip end   = 19,
  %               vertical clip start   = 9,
  %               vertical clip end     = 19]
  %   \parseSgfFile{sgf/captura/captura.7.solution.2.sgf}
  % \end{goban}

  % \problemDiagram{captura}
  %                {captura.7}
  %                {Here}

  % \answerDiagrams{captura}
  %                {captura.7}
  %                {Here}
  %                {Here}

  % \problemAnswerDiagram{captura}
  %                       {captura.7}
  %                       {Here}
  %                       {Here}
  %                       {Here}

  % \newcommand\coverRuler{
  \rule{\textwidth}{1.6pt}\vspace*{-\baselineskip}\vspace*{2pt}
  \rule{\textwidth}{0.4pt}
}

\begin{titlepage}
  \centering
  
  \scshape % Small Caps

  \vspace*{-1.5cm}
  
  %---------------------------------------------------------
  % Título
  
  \coverRuler
  
  \vspace{0.7\baselineskip}
  
  \huge{Tsumegos para Iniciantes}\\
  \vspace*{10pt}
  \large{Volume 1 | 30-20 kyu}\\
  \vspace*{10pt}
  \Large{Liberdades e Vida ou Morte}

  \vspace{0.275\baselineskip}
  
  \coverRuler

  %---------------------------------------------------------

  \vspace*{-0.0675cm}

  %---------------------------------------------------------
  % Imagem
  
  \begin{figure}[h]
    \centering
    \hbox{
      \hspace{0.6cm}
      \begin{goban}[board dimension       = 6,
                    board size            = 9,
                    scale                 = 1,
                    outline line width    = 0.45mm,
                    horizontal clip start = 1,
                    horizontal clip end   = 9,
                    vertical clip start   = 1,
                    vertical clip end     = 9]
        \parseSgfFile{sgf/cover_9x9.sgf}
      \end{goban}
    }
  \end{figure}

  %---------------------------------------------------------
  
  \vfill

  %---------------------------------------------------------
  % Autor

  \Large{Philippe Fanaro}

  %---------------------------------------------------------
\end{titlepage}
  % % \blankpage

  % \frontmatter

  % \tableofcontents

  % \chapter{Prefacio}
  % % \newcommand\shref[2]{
  \small{\href{#1}{\path{#2}}}
}

\chapter{Prefácio}
  
Há 3 pilares para se aprimorar no Go, qual seja o seu nível, desde iniciante até campeão mundial: jogar, revisar e praticar \emph{tsumegos}. Para quem não os conhece, tsumegos são problemas localizados no jogo de Go.

Evidentemente, jogando e revisando é possível praticar tsumegos, mas o volume por unidade de tempo será bem mais baixo, já que, em partidas, há muito mais com o que se preocupar. Tsumegos são o equivalente de musculação para atletas de esportes físicos: uma maneira eficiente de se treinar em algo específico relacionado ao esporte desejado.

O propósito principal deste livro e da coleção \emph{Tsumegos para Iniciantes} não é, no entanto, ser uma coletânea exaustiva de tsumegos para iniciantes. O intuito aqui é servir de referência de técnicas e padrões essenciais, além de ter um volume mínimo para que se possa ser considerado como uma sessão de treinamento. 

Caso o leitor queira um maior volume e diversidade de tsumegos, hoje em dia, não há recurso melhor do que o \shref{https://101weiqi.com}{101weiqi.com}. Outro recurso importante é o servidor OGS (\shref{https://online-go.com}{online-go.com}), que também possui muitos tsumegos, além de uma excelente página de referências: \shref{https://online-go.com/docs/other-go-resources}{online-go.com/docs/other-go-resources}.

\bigskip
\bigskip

Para quem não me conhece, sou Philippe Fanaro, e jogo Go desde o final de 2012. Ao redor de quando escrevo este livro, oscilo entre 2 e 3 dan no servidor KGS, ou algo entre 5 e 6 dan no servidor Fox.

Venho compartilhando conteúdo de Go online especialmente pelo meu canal de YouTube em português: \shref{https://youtube.com/@Fanaro}{youtube.com/@Fanaro}. Mas, também, pelo meu blog: \shref{https://fanaro.io}{fanaro.io}.

Outro conteúdo útil a mencionar é a tradução do livro \emph{Como Jogar Go --- Uma Introdução Concisa}, de Richard Bozulich e James Davies, que eu fiz em código aberto alguns anos atrás, ela está disponível aqui, cujo PDF é gratuito: \shref{https://github.com/FanaroEngineering/traducao\_como\_jogar\_go}{github.com/FanaroEngineering/traducao_como_jogar_go}.

\bigskip
\bigskip

Philippe Fanaro

12 de Setembro de 2024

\pagebreak

% TODO: add another empty page
% TODO: center vertically

\begin{center}
  \Large \emph{Sempre Preto a jogar.}
\end{center}

  % \mainmatter
  % \part{Liberdades}
    % \chapter{Captura}
    %   Here
    %   Blah

      \answerDiagrams{captura}
                     {captura.7}
                     {Here}
                     {Here}

      \answerDiagrams{captura}
                     {captura.8}
                     {Here}
                     {Here}
                     
      \answerDiagrams{captura}
                     {captura.7}
                     {Here}
                     {Here}
      
      % \problemAnswerDiagram
      %   {captura}
      %   {captura.8}
      %   {Há uma pedra branca solitária no topo. O que Preto pode fazer com ela?}
      %   {\emph{Correto.} Preto pode capturá-la e, assim, resolver o problem do corte que existia em A.}
      %   {\emph{Incorreto.} Preto poderia ter capturado a pedra, o que resolveria o corte de uma maneira mais eficiente, sem jogar em seu próprio território. E esta defesa dá chances para que Branco conecte todas as suas pedras mais tarde.}

      % \problemAnswerDiagram
      %   {captura}
      %   {captura.9}
      %   {Capturas no canto são geralmente de extremo valor pois podem garantir não somente território, mas, também, o espaço vital de grupos, como veremos mais tarde nos exercícios de vida ou morte deste livro.}
      %   {\emph{Correto.} A captura da pedra do canto limpa o canto para Preto, e ainda deixa a pedra A com somente 1 liberdade.}
      %   {\emph{Incorreto.} Localmente pelo menos, não jogar aqui como Preto ajuda Branco a se estabilizar localmente, enquanto contra-ataca severamente.}

      % \problemAnswerDiagram
      %   {captura}
      %   {captura.7}
      %   {O grupo branco mais ao topo possui 2 pedras. Isso muda algo em relação à possibilidade de captura?}
      %   {\emph{Correto.} Apesar de o grupo branco ter mais pedras, ele ainda possui somente 1 liberdade. Esta forma é conhecida como ``casco de tartaruga'', que é o número mínimo de pedras necessário para se capturar 2 pedras. Também era possível capturar com A. (Note que ambas as pedras brancas serão retiradas do tabuleiro.)}
      %   {\emph{Variação.} Ao fugir, Branco não somente resgata suas pedras como expõe múltiplas fraquezas no exterior preto.}
\end{document}