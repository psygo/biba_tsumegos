\documentclass{book}

\renewcommand{\footnotesize}{\normalsize}
\renewcommand{\normalsize}{\fontsize{12pt}{13pt}}

\usepackage[T1]{fontenc}
\usepackage{lmodern}
\usepackage{fontspec}
\setmainfont{Adobe Garamond Pro}

\usepackage[portuguese]{babel}
\usepackage{microtype}

\begin{document}
  Ao longo do livro, os comentários das respostas focam em criar uma ponte entre... a visão de alguém que está começando e aquela de um jogador mais avançado. Mas, mesmo com esses comentários, progredir linearmente neste livro pode ser bem desafiador. Se for este o caso, então melhor ainda, pois é sinal de que há muito o que aprender. É, porém, compreensível a estranheza se a expectativa fosse a de um material de treinamento. Foi um dos focos mas da criação desta coleção, mas, como há muitas técnicas e uma quantidade limitada de papel, o volume de exercícios por tipo de sequência não é tão alto. Mas persevere! Como treinamento, sugiro fortemente refazer o livro múltiplas vezes, provavelmente espaçadas por uma ou mais semanas. Repetição será essencial para tornar as técnicas intuitivas: errar na primeira vez será fácil, errar depois da décima garanto que será muito mais difícil!
\end{document}