\documentclass[12pt]{book}

% ~A6
\usepackage[%
  paperheight   = 148mm, 
  paperwidth    = 105mm,
  bindingoffset = 0mm,
  left          = 12.5mm,
  right         = 12.5mm,
  top           = 15.2mm,
  bottom        = 13.2mm,
  footskip      = 6.35mm,
]{geometry}

\usepackage{config}

\begin{document}
  \begin{titlepage}
    \setlength\parindent{0ex}

    \vspace*{-1.5cm}

    \begin{center}
      \LARGE{\scshape Técnicas de Go}
      
      \vspace*{0.3cm}
      
      \large{\scshape Liberdades e Vida ou Morte}
    \end{center}
    
    \vspace*{0.325cm}
    
    Começando desde basicamente as regras, com 200 problemas, estabelecemos padrões e técnicas de referência essenciais no jogo de Go, em uma progressão estruturada e comentada acessível a iniciantes, mas que vale a pena até mesmo para jogadores mais avançados.

    \vspace*{0.1cm}

    \noindent\rule[0.1pt]{\linewidth}{0.25pt}

    \vspace*{0.325cm}

    Philippe Fanaro é um jogador amador \emph{dan} brasileiro, formado em engenharia elétrica pela Escola Politécnica da USP, que procura trazer mais acesso, em língua portuguesa, ao conteúdo de Go disponível em inglês e línguas asiáticas.

    \vspace*{0.0375cm}
    
    \begin{figure}[ht]
      \centering
      \hbox{
        \hspace{0.95cm}
        \begin{goban}[board dimension       = 8.75,
                      board size            = 19,
                      scale                 = 1,
                      outline line width    = 0.45mm,
                      horizontal clip start = 9,
                      horizontal clip end   = 19,
                      vertical clip start   = 14,
                      vertical clip end     = 19]
          \parseSgfFile{sgf/liberdades/escadas/escadas.3.sgf}
        \end{goban}
      }
    \end{figure}
  \end{titlepage}
\end{document}