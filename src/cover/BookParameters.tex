% ----------------------------------------------------------------------------

\usepackage{tikz}
\usepackage{pgf}
\usepackage{xifthen}

\usepackage[T1]{fontenc}
\usepackage[portuguese]{babel}
\usepackage{microtype}

\usepackage{lmodern}

% Don't forget to change to LuaLaTeX on `makecover.sh`
\usepackage{fontspec}
% From [this comment](https://www.reddit.com/r/identifythisfont/comments/1fiahvg/comment/lngfilr/?utm_source=share&utm_medium=web3x&utm_name=web3xcss&utm_term=1&utm_content=share_button)
\setmainfont{Adobe Garamond Pro}

\renewcommand{\footnotesize}{\normalsize}
\renewcommand{\normalsize}{\fontsize{12pt}{13pt}}


\usepackage[colorlinks  = true,
            allcolors   = black,
            pdfsubject  = { Go; Baduk; Weiqi },
            pdftitle    = { Técnicas de Go --- Volume I --- Captura e Vida ou Morte },
            pdfauthor   = { Philippe Fanaro },
            pdfkeywords = { Go; Baduk; Weiqi; Introdução },
            pdfproducer = { LaTeX },
            pdfcreator  = { pdflatex; bibtex },
            bookmarksnumbered]{hyperref}
\usepackage{xurl}

\usepackage{../../utils}

% At 215 pages, the spine is wide enough to hold a publishers logo
% and a nice sized title. At 315 pages the spine can hold a subtitle.
% The logo is not available below 215 pages, and below 150 pages the
% title size is very reduced. A blank spine is used when the book has
% fewer than 123 pages.
%
% Note that some printers require specific page count multiples:
% -- Createspace: page count MUST be divisible by 2.
% -- Ingram: page count MUST be divisible by 2.
% -- Blurb : page count MUST be divisible by 6.
% Add blank pages as needed in final PDF generations!
\pgfmathsetmacro\TotalPageCount{498} % Must be manually entered
\pgfmathsetmacro\PaperHeightPt{148mm}
\pgfmathsetmacro\PaperWidthPt{105mm}

%\def\ShowCoverStatistics{} % Enable to show some numbers on back cover.

\def\TheMainTitle{Técnicas de Go}
\def\TheSubTitle{Captura e Vida ou Morte}
\def\TheSubTitleA{Volume I}
\def\TheSubTitleB{Captura e Vida ou Morte}
\def\TheBookSeries{Técnicas de Go}

\def\TheAuthor{Philippe Fanaro}
\def\TheAuthorA{Fanaro}
\def\TheAuthorB{}
\def\TheAuthorC{}
\def\TheAuthorLNF{Bozulich, Richard}

\def\TheCopyrightYear{2024}
\def\TheEdition{Primeira Edição}

\def\ThePublisherName{Philippe Fanaro} % Just the short form name
\def\ThePublisherLineA{Philippe Fanaro} % Just one part of the name
\def\ThePublisherLineB{Philippe Fanaro} % Just one part of the name
\def\ThePublisher{\ThePublisherName} % Add Ltd, Inc, LLC etc here
\def\ThePublisherAddrA{São Paulo}
\def\ThePublisherAddrB{São Paulo}
\def\ThePublisherCity{São Paulo}
\def\ThePublisherState{São Paulo}

\def\ThePrinter{Imagem Digital}

\def\TheSubjectArea{Go, Baduk, Weiqi}

\def\TheKeywords{Go, Baduk, Weiqi}

% ----------------------------------------------------------------------------

% NOTE Each version of a book usually requires its own ISBN. You can
% obtain blocks of 10 ISBNs for $250 from https://www.myidentifiers.com/
% You must know you Imprint Name (publisher). 
% Dashes MUST be placed in the correct postions in the ISBNs.

% ISBN generated by Amazon KDP.

\def\PrintISBN{} % chktex 8
\def\PrintISBNShort{} % chktex 8
%\def\HardcoverISBN{978-4-4444444-4-6}
%\def\HardcoverISBNShort{4-4444444-4-9}
\def\EbookISBN{} % chktex 8
\def\EbookISBNShort{} % chktex 8

% ----------------------------------------------------------------------------

% The ISBN barcode is created with the GhostScript barcode.ps generator:
% see Terry Burton's PostScript % based barcode generator, at: 
%       https://github.com/bwipp/postscriptbarcode.
% Use 99.99 if your price exceeds $100 (good luck selling that puppy).
% Comment out these entries if you don't want a price in the barcode.

\def\PrintPrice{7.99} % Price in dollars and cents for barcode.
%\def\HardcoverPrice{19.95} % Price in dollars and cents for barcode.

% ----------------------------------------------------------------------------

% NOTE that P-CIP (Publisher's Catalog in Publication) data block must be
% aquired from http://www.quality-books.com/pcip.htm. This is placed on the
% copyright page, and is required by all libraries and many booksellers.
% The P-CIP costs $100 and cannot be obtained until much of a first draft
% is available as an ASCII text file. 
%
% Only US Publishers (not independent or self), can obtain an LC-CIP
% (Library of Congress Catalog in Publication) data block directly from 
% the Library of Congress.
\def\TheCIPType{}
%\def\TheCIPType{Library of Congress Cataloging-in-Publication Data}

% Library of Congress Catalog Subject Headings
\def\TheCIPSubjectHeadings{}

% Dewey Decimal System classification number
\def\TheDDSN{}

% LCCN: Library of Congress classification number (not a Control Number or PCN)
\def\TheLCCN{}

% ----------------------------------------------------------------------------

% The PCN (also called an LCCN) are acquired from the Library of Congress.
% This is done by contacting: http://www.loc.gov/publish/pcn/newaccount.html
% after you have purchased an ISBN (13 digit) from Bowker, preferably for both
% your print and eBook versions.

% PCN: Library of Congress’ Preassigned Control Number (PCN)
% Also called : Library of Congress Control Number
% Also called : Library of Congress Card Number
%
% This takes about a day to get assigned and is free.
% \def\TheLCPCN{2017xxxxxx}

% ----------------------------------------------------------------------------

% \def\TheCopyrightKeywords{1. History; 2. Nonsense; 3. Made up stuff}

% \def\ShortDescription{}

\def\BackDescription{
  \large
  
  Começando desde basicamente as regras, com 200 problemas, estabelecemos padrões e técnicas de referência essenciais no jogo de Go, em uma progressão estruturada e comentada acessível a iniciantes, mas que vale a pena até mesmo para jogadores mais avançados.
}

\def\AuthorBio{
  \large

  Philippe Fanaro é um jogador amador \emph{dan} brasileiro, formado em engenharia elétrica pela Escola Politécnica da USP, que procura trazer mais acesso, em língua portuguesa, ao conteúdo de Go disponível em inglês e línguas asiáticas.
}

% ----------------------------------------------------------------------------
% Computed parameters for cover and jacket design 

% CreateSpace and Ingram, use different paper stock so the spine width must
% be adjusted to refect that. 
% For example Createspace, White paper: multiply page count by 0.002252

\pgfmathsetmacro\SinglePageThicknessPt{0.002252in} % Createspace, White B&W Paper
%\pgfmathsetmacro\SinglePageThicknessPt{0.002143in} % Blurb, Standard and Econony B&W Paper
%\pgfmathsetmacro\SinglePageThicknessPt{0.002602in} % Blurb, Standard and Economy Color Paper
%\pgfmathsetmacro\SinglePageThicknessPt{0.002110in} % Ingram, Standard White B&W Paper (50lb)
%\pgfmathsetmacro\SinglePageThicknessPt{0.002110in} % Ingram, Standard Color Paper (50lb)
%\pgfmathsetmacro\SinglePageThicknessPt{0.002720in} % Ingram, Premium Color Paper (70lb)

\pgfmathsetmacro\HorBleedPt{0.125in}
\pgfmathsetmacro\VerBleedPt{0.125in}
\pgfmathsetmacro\FoldVariancePt{0.0625in}

% Every book will vary slightly when bound. Allow for 0.0625" variance on either
% side of the fold lines for your cover. For example, if your spine width is 1",
% your text should be no wider than 0.875". Because of this variance, avoid hard
% edges or lines that end on the fold line.

% Cover Width calculation at 6" x 9" cover with 60 B&W pages on white paper:
%    0.125" + 6" + (60 * 0.002252)" + 6" + .125" = 12.385"
% Cover Height calculation: 6" x 9": 0.125" + 9" + .125" = 9.25"

\pgfmathsetmacro\SpineWidthPt{\SinglePageThicknessPt*\TotalPageCount}
\pgfmathsetmacro\CoverWidthPt{\PaperWidthPt+\HorBleedPt}
\pgfmathsetmacro\JacketWidthPt{\CoverWidthPt+\SpineWidthPt+\CoverWidthPt}
\pgfmathsetmacro\CoverHeightPt{\VerBleedPt+\PaperHeightPt+\VerBleedPt}

\pgfmathsetmacro\PtsPerInch{72.27}% Slightly more than 72 - odd.

\pgfmathsetmacro\SpineWidth{\SpineWidthPt    / \PtsPerInch} % chktex 1
\pgfmathsetmacro\PaperWidth{\PaperWidthPt    / \PtsPerInch} % chktex 1
\pgfmathsetmacro\CoverWidth{\CoverWidthPt    / \PtsPerInch} % chktex 1
\pgfmathsetmacro\JacketWidth{\JacketWidthPt  / \PtsPerInch} % chktex 1

\pgfmathsetmacro\PaperHeight{\PaperHeightPt  / \PtsPerInch} % chktex 1
\pgfmathsetmacro\CoverHeight{\CoverHeightPt  / \PtsPerInch} % chktex 1

\pgfmathsetmacro\HorBleed{\HorBleedPt        / \PtsPerInch} % chktex 1
\pgfmathsetmacro\VerBleed{\VerBleedPt        / \PtsPerInch} % chktex 1
\pgfmathsetmacro\FoldVariance{\FoldVariancePt/ \PtsPerInch} % chktex 1
 
% ----------------------------------------------------------------------------

\DeclareOldFontCommand{\bf}{\normalfont\bfseries}{\mathbf}
\renewcommand{\bf}[1]{\textbf{#1}}% Legacy \bf support.

\newcommand{\SetBool}[2]{%
  \ifthenelse{\equal{#2}{true}\OR\equal{#2}{on}\OR\equal{#2}{yes}}
  {\setboolean{#1}{true}}{\setboolean{#1}{false}}}

\newcommand{\Boolean}[2]{\newboolean{#1}
  \ifthenelse{\isempty{#2}}{}{\SetBool{#1}{#2}}}
  
% End of Book Parameters.
% ----------------------------------------------------------------------------
