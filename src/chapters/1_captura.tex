\chapter{Captura}

% 1
\problemDiagrams
  {É possível capturar a pedra branca?}
\answerDiagrams
  {\emph{Correto.} Sim, é possível capturar a pedra branca diretamente, já que ela só possui 1 liberdade. (Note que a pedra branca será retirada do tabuleiro.)}
  {\emph{Variação.} Mesmo se Branco tentar resgatar ou contra-capturar a pedra preta, sua pedra do canto ainda possui somente 1 liberdade e pode ser capturada.}

% 2
\problemDiagrams
  {E se a pedra branca estiver no lado ao invés do canto, ainda é possível capturá-la?}
\answerDiagrams
  {\emph{Correto.} Branco não deveria jogar 2, pois não há para onde fugir. Suas pedras terão sempre 1 liberdade, mas vão colidir com o canto esquerdo no final.}
  {\emph{Variação.} Preto pode sempre fazer \emph{tenuki}, isto é, ignorar movimentos locais e jogar em outro lugar, mas aí é Branco quem poderá capturar a pedra preta.}

% 3
\problemDiagrams
  {É possível capturar a pedra branca?}
\answerDiagrams
  {\emph{Correto.} A captura da pedra branca configura uma forma chamada \emph{ponnuki}, que é o nú-mero mínimo de pedras para se capturar uma pedra adversária, quando ela não está nem no canto e nem na borda. (Note que a pedra branca será retirada do tabuleiro.)}
  {\emph{Incorreto.} Ao escapar com esta pedra, Branco vai de 1 liberdade para 3, o que é bastante eficiente, isto é, 2 liberdades por movimento; em lutas, liberdades são um bem crucial. Desta maneira, Branco também expõe os cortes A e B no exterior preto.}
