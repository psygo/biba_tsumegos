\chapter{Captura}

\problemDiagrams
\answerDiagrams%
  {A captura da pedra branca configura uma forma chamada \emph{ponnuki}, que é o número mínimo de pedras para se capturar uma pedra adversária, quando ela não está nem no canto e nem na borda. (Note que a pedra branca será retirada do tabuleiro.)}%
  {Ao escapar com esta pedra, Branco vai de 1 liberdade para 3, o que é bastante eficiente, isto é, 2 liberdades por movimento. Em lutas, liberdades são um bem crucial. Branco também expõe cortes no exterior preto.}%


\problemDiagrams
\answerDiagrams%
  {A captura da pedra branca configura uma forma chamada \emph{ponnuki}, que é o número mínimo de pedras para se capturar uma pedra adversária, quando ela não está nem no canto e nem na borda. (Note que a pedra branca será retirada do tabuleiro.)}%
  {Ao escapar com esta pedra, Branco vai de 1 liberdade para 3, o que é bastante eficiente, isto é, 2 liberdades por movimento. Em lutas, liberdades são um bem crucial. Branco também expõe cortes no exterior preto.}%