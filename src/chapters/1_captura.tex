\chapter{Captura}

\clearpage

% 1
\problemDiagrams
  {captura}
  {1}
  {É possível capturar a pedra branca?}
% \answerDiagrams
%   {\emph{Correto.} Sim, é possível capturar a pedra branca diretamente, já que ela só possui 1 liberdade. (Note que a pedra branca será retirada do tabuleiro.)}
%   {\emph{Variação.} Mesmo se Branco tentar resgatar ou contra-capturar a pedra preta, sua pedra do canto ainda possui somente 1 liberdade e pode ser capturada.}

% % 2
% \problemDiagrams
%   {E se a pedra branca estiver no lado ao invés do canto, ainda é possível capturá-la?}
% \answerDiagrams
%   {\emph{Correto.} Branco não deveria jogar 2, pois não há para onde fugir. Suas pedras terão sempre 1 liberdade, mas vão colidir com o canto esquerdo no final.}
%   {\emph{Variação.} Preto pode sempre fazer \emph{tenuki}, isto é, ignorar movimentos locais e jogar em outro lugar, mas aí é Branco quem poderá capturar a pedra preta.}

% % 3
% \problemDiagrams
%   {Agora a pedra branca está no centro. Ela ainda pode ser capturada?}
% \answerDiagrams
%   {\emph{Correto.} A captura da pedra branca configura uma forma chamada \emph{ponnuki}, que é o nú-mero mínimo de pedras para se capturar uma pedra adversária, quando ela não está nem no canto e nem na borda. (Note que a pedra branca será retirada do tabuleiro.)}
%   {\emph{Variação.} Ao escapar com esta pedra, Branco vai de 1 liberdade para 3, o que é bastante eficiente, isto é, 2 liberdades por movimento; em lutas, liberdades são um bem crucial. Desta maneira, Branco também expõe os cortes A e B no exterior preto.}

% % 4
% \problemDiagrams
%   {Preto pode conectar suas pedras?}
% \answerDiagrams
%   {\emph{Correto.} Conectar as pedras zeraria as liberdades de todas as pedras do grupo, o que as automaticamente capturaria, ou seja, seria suicídio, uma jogada inválida.}
%   {\emph{Variação.} Branco tem sempre a opção de capturar ambas as pedras.}

% % 5
% \problemDiagrams
%   {Preto pode capturar as pedras brancas?}
% \answerDiagrams
%   {\emph{Correto.} Note que não é suicídio jogar em 1, pois a regra da captura possui precedência. Isto é, primeiro aplicamos a regra da captura, se possível, e, só depois, examinamos se é suicídio. E segue que, se algo for capturado, haverá mais de uma liberdade.}
%   {\emph{Variação.} Mais tarde, se Branco conseguir pedras no exterior, quem pode ser capturado é o Preto! Antes de 1, Preto poderia finalmente capturar o canto, e, dessa maneira, evitar de ser capturado.}

% % 6
% \problemDiagrams
%   {É possível capturar algo branco? Ou é suicídio?}
% \answerDiagrams
%   {\emph{Correto.} Sim, é possível capturar duas pedras brancas. A pedra preta que captura estará imediatamente em \emph{atari}, isto é, ela possuirá somente 1 liberdade, então Branco terá a opção de uma contra- ou recaptura.}
%   {\emph{Variação.} Caso Preto opte por não capturar, Branco pode salvar as pedras e deixar Preto com fraquezas no exterior. Isso não quer dizer que não capturar é um erro, pois pode haver outros movimentos mais importantes no tabuleiro.}

% % 7
% \problemDiagrams
%   {O grupo branco possui 2 pedras. Isso muda algo em relação à possibilidade de captura?}
% \answerDiagrams
%   {\emph{Correto.} Apesar de o grupo branco ter mais pedras, ele ainda possui somente 1 liberdade. Esta forma é conhecida como ``casco de tartaruga", que é o número mínimo de pedras necessário para se capturar 2 pedras. (Note que ambas as pedras brancas serão retiradas do tabuleiro.)}
%   {\emph{Variação.} Ao fugir, Branco não somente resgata suas pedras como expõe múltiplas fraquezas no exterior Preto.}
