\chapter{Formas Mortas}

\emptypage

\problemAnswerDiagram
  {formas_mortas}
  {formas_mortas.1}
  {Como matar o grupo branco?}
  {\emph{Correto.} Ao jogar no eixo de simetria, que é geralmente um bom movimento em problemas simétricos, Preto impossibilita que Branco dissocie seu espaço de olho, além de garantir que o espaço de olho seja reduzido a uma das formas mortas.}
  {\emph{Incorreto.} Se Preto não jogar nada, Branco pode viver.}

\problemAnswerDiagram
  {formas_mortas}
  {formas_mortas.2}
  {Será que conseguimos reduzir este problema a algo mais simples?}
  {\emph{Correto.} Ao jogar 1, Preto impossibilita separar o espaço de olho em 2. (Se Branco jogasse em 1, estaria vivo.)}
  {\emph{Continuação.} Preto reduzirá o espaço de olho para uma forma morta de 3 pedras. Em japonês, o termo para forma morta é \emph{nakade}.}

\problemAnswerDiagram
  {formas_mortas}
  {formas_mortas.11}
  {E agora? Preto está cercado. Resta somente uma corrida de liberdades para capturar o canto. Será que ele vence?}
  {\emph{Correto.} Preto ganhará a corrida por 1 liberdade.}
  {\emph{Continuação.} Branco pode capturar as pedras no canto, mas não conseguirá aumentar suas liberdades.}

\problemAnswerDiagram
  {formas_mortas}
  {formas_mortas.5}
  {O ponto-chave do seu oponente é frequentemente o seu ponto-chave, mas para matar.}
  {\emph{Correto.} No máximo, Bran-co conseguirá um olho grande de 2 ou 3 pedras.}
  {\emph{Incorreto.} Agora, há 2 olhos.}

\problemAnswerDiagram
  {formas_mortas}
  {formas_mortas.12}
  {Preto consegue capturar o canto antes de ser capturado? Será que ele consegue até fazer tenuki? Se sim, quantos?}
  {\emph{Correto.} Preto consegue até jogar em outro lugar 1 vez! Branco só consegue 3 liberdades no canto, devido a uma recaptura de forma morta. Esta é uma das magias das formas mortas. É uma leitura avançada, mas é essencial ver como ela acontece mesmo sem conseguir lê-la.}
  {\emph{Continuação.} A corrida termina em sucesso para Preto.}

\problemAnswerDiagram
  {formas_mortas}
  {formas_mortas.6}
  {Talvez não pareça, mas, uma vez que formas mortas se tornarem mais intuitivas, este problema parecerá idêntico a vários dos anteriores.}
  {\emph{Correto.} Novamente, uma forma morta de 3 pedras.}
  {\emph{Incorreto.} Se Branco puder jogar, ele conseguirá dividir seu espaço interno.}

\problemAnswerDiagram
  {formas_mortas}
  {formas_mortas.3}
  {Você consegue acertar o ponto vital do grupo branco?}
  {\emph{Correto.} Preto 1 é o ponto-chave para separar o espaço vital em 2.}
  {\emph{Incorreto.} Branco ainda consegue dividir seu espaço de olho.}

\problemAnswerDiagram
  {formas_mortas}
  {formas_mortas.8}
  {Preto possui movimentos forçados contra Branco.}
  {\emph{Correto.} O atari ajuda a criar uma forma morta.}
  {\emph{Variação.} Simplesmente subir também funciona, pois Branco não consegue fazer o auto-atari de A.}

\problemAnswerDiagram
  {formas_mortas}
  {formas_mortas.4}
  {Branco possui um problema de falta de liberdades.}
  {\emph{Correto.} Branco não consegue fazer atari em 3 pois seria auto-atari.}
  {\emph{Incorreto.} Branco ainda consegue dividir seu espaço de olho.}

\problemAnswerDiagram
  {formas_mortas}
  {formas_mortas.7}
  {Agora, Branco possui um espaço vital de 4 ``quadrado''. É possível viver com este espaço?}
  {\emph{Correto.} Não, não é possível viver com o espaço de 4 quadrado, Branco já está morto.}
  {\emph{Variação.} Preto impossibilita a criação de 2 olhos.}

\problemAnswerDiagram
  {formas_mortas}
  {formas_mortas.13}
  {Preto pode fazer tenuki? Quantos?}
  {\emph{Correto.} Não, não é possível fazer tenuki. Tanto o grupo preto quanto o canto branco possuem 3 liberdades, quem jogar primeiro, ganha. Preto 3 em A também funcionaria, assim como Preto 1 em 3, que é simétrico.}
  {\emph{Incorreto.} Se Branco conseguir jogar primeiro, Preto será capturado..}

\problemAnswerDiagram
  {formas_mortas}
  {formas_mortas.9}
  {Branco possui uma outra variação de espaço vital de 4 intersecções agora.}
  {\emph{Correto.} Ao acertar no centro deste espaço de 4 intersecções, Preto mata o grupo inteiro.}
  {\emph{Variação.} Branco poderia viver, se fosse seu turno.}

\problemAnswerDiagram
  {formas_mortas}
  {formas_mortas.10}
  {Agora, chegamos à praticamente a última forma morta que aparece comumente em partidas e problemas de vida ou morte.}
  {\emph{Correto.} Esta forma morta é chamada de ``bulky five'' em inglês, ou ``cinco pesado'', ou ``cinco corpulento'', em português. Seu ponto fraco é Preto 1.}
  {\emph{Incorreto.} Branco faz um olho a partir de 2, e Preto não consegue prosseguir.}

\problemAnswerDiagram
  {formas_mortas}
  {formas_mortas.14}
  {Esta corrida de liberdades não é nada fácil. Ter maestria sobre essas corridas de formas mortas é algo que levará um tempo e repetição. Preto consegue fazer tenuki? Talvez Preto já até esteja morto?}
  {\emph{Correto.} Corridas de liberdades com formas mortas são notoriamente recursivas. Dentro de uma forma morta de 5, temos que lidar com uma de 4, e depois uma de 3.}
  {\emph{Continuação.} Daqui em diante, a corrida se reduz a uma que já examinamos. A forma morta de 5 toma, então 8 movimentos para ser capturada (sem contar com o primeiro movimento jogado em 1).}

\problemAnswerDiagram
  {formas_mortas}
  {formas_mortas.15}
  {Uma outra variação de forma morta de 5 intersecções. Como matar?}
  {\emph{Correto.} Outro caso de o ponto de simetria, ou do meio, sendo o ponto-chave.}
  {\emph{Variação.} O ponto-chave para matar é o também o de viver.}

\problemAnswerDiagram
  {formas_mortas}
  {formas_mortas.16}
  {E se Branco tiver uma intersecção a mais?}
  {\emph{Correto.} A maioria dos jogadores ainda consideraria Preto 1 como ``jogar no meio ou no centro da forma''.}
  {\emph{Variação.} Branco pode utilizar o mesmo raciocínio para viver.}