\chapter{Formas Mortas}

\emptypage

\problemAnswerDiagram
  {formas_mortas}
  {formas_mortas.1}
  {Como matar o grupo branco?}
  {\emph{Correto.} Ao jogar no eixo de simetria, que é geralmente um bom movimento em problemas simétricos, Preto impossibilita que Branco dissocie seu espaço de olho, além de garantir que o espaço de olho seja reduzido a uma das formas mortas.}
  {\emph{Incorreto.} Se Preto não jogar nada, Branco pode viver.}

\problemAnswerDiagram
  {formas_mortas}
  {formas_mortas.2}
  {Será que conseguimos reduzir este problema a algo mais simples?}
  {\emph{Correto.} Ao jogar 1, Preto impossibilita separar o espaço de olho em 2. (Se Branco jogasse em 1, estaria vivo.)}
  {\emph{Continuação.} Preto reduzirá o espaço de olho para uma forma morta de 3 pedras.}

\problemAnswerDiagram
  {formas_mortas}
  {formas_mortas.5}
  {O ponto-chave do seu oponente é frequentemente o seu ponto-chave, mas para matar.}
  {\emph{Correto.} No máximo, Bran-co conseguirá um olho grande de 2 ou 3 pedras.}
  {\emph{Incorreto.} Agora, há 2 olhos.}

\problemAnswerDiagram
  {formas_mortas}
  {formas_mortas.3}
  {Você consegue acertar o ponto vital do grupo branco?}
  {\emph{Correto.} Preto 1 é o ponto-chave para separar o espaço vital em 2.}
  {\emph{Incorreto.} Branco ainda consegue dividir seu espaço de olho.}

\problemAnswerDiagram
  {formas_mortas}
  {formas_mortas.4}
  {Branco possui um problema de falta de liberdades.}
  {\emph{Correto.} Branco não consegue fazer atari em 3 pois seria auto-atari.}
  {\emph{Incorreto.} Branco ainda consegue dividir seu espaço de olho.}