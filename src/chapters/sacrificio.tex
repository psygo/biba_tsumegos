\chapter{Sacrifício}

\emptypage

\problemAnswerDiagram
  {sacrificio}
  {sacrificio.1}
  {Já entramos em técnicas de captura bastante avançadas. E esta técnica especificamente é devastadora localmente caso consigamos aplicá-la.}
  {\emph{Correto.} Com um sacrifício, forçamos Preto a se colocar em atari. Esta técnica em que sacrificamos uma pedra para deixar Preto imediatamente em atari é chamada de ``snapback'' em inglês, o que pode ser traduzido talvez como ``ricochete'' em português.}
  {\emph{Continuação.} Em seguida, capturamos diretamente. Tipicamente, esta técnica é tão potente que Branco deveria ter jogado em A ao invés de B, caso B tenha sido a última jogada branca.}

\problemAnswerDiagram
  {sacrificio}
  {sacrificio.2}
  {Branco parece ter se esquecido de algo.}
  {\emph{Correto.} Preto pode capturar duas pedras brancas através de um sacrifício inicial. Note também que a captura termina com a pedra A possibilitando uma outra futura captura em B, portanto, se Branco quiser proteger o canto, terá que conectar em B ainda.}
  {\emph{Variação.} Branco pode proteger com 1 ou A, mas A geralmente leva a mais forma de olho localmente, o que pode ser importante no futuro.}

\problemAnswerDiagram
  {sacrificio}
  {sacrificio.3}
  {Um problema não muito simples, mas, com boa técnica, conseguiremos simplificá-lo drasticamente.}
  {\emph{Correto.} Mais uma vez, em problemas simétricos, o eixo de simetria, o meio, é sempre um excelente indício. Neste caso, ao jogar no meio conseguimos aplicar um sacrifício por ambos os lados.}
  {\emph{Incorreto.} Preto não consegue conectar em A pois seria suicídio, portanto, os dois olhos já estão configurados.}

\problemAnswerDiagram
  {sacrificio}
  {sacrificio.4}
  {Uma situação um pouco mais delicada. Será que a mesma técnica funciona?}
  {\emph{Correto.} Aqui, é melhor capturar diretamente, pois Preto possui problemas de liberdades no exterior com a forma marcada.}
  {\emph{Subótimo.} Caso Preto foque em otimizar o fim de jogo, Branco consegue movimentos forçados no centro, o que será muito provavelmente muito mais valioso do que o lucro preto no topo.}

\problemAnswerDiagram
  {sacrificio}
  {sacrificio.5}
  {Se você conseguir resolver este problema em uma partida real, com certeza, os espectadores ficarão impressionados.}
  {\emph{Correto.} Branco não consegue resgatar suas pedras se Preto cortar em 1.}
  {\emph{Incorreto.} Branco toma para si o ponto-chave para a captura.}

\problemAnswerDiagram
  {sacrificio}
  {sacrificio.6}
  {Branco achou que já tinha bastante forma de olho, o suficiente para pelo menos 2 olhos.}
  {\emph{Correto.} Ao capturar com 1, o grupo à direita não possui espaço o suficiente para 2 olhos, então o grupo inteiro morre.}
  {\emph{Variação.} Se Branco tivesse respondido com uma conexão, o resultado seria um ko ainda, Branco não vive incondicionalmente.}