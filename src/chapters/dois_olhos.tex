\chapter{Dois Olhos}

\emptypage

\problemAnswerDiagram
  {dois_olhos}
  {dois_olhos.1}
  {Este exercício é uma das essências do Go. É possível permanecer incondicionalmente no tabuleiro?}
  {\emph{Correto.} Sim, com 1, estabelecemos ``dois olhos'', o que impossibilita a captura do grupo. Branco precisaria jogar tanto em A quanto em B para capturar, mas ele só possui um movimento por turno, portanto, cada um deles será considerado suicídio!}
  {\emph{Incorreto.} Preto talvez pense que seja possível capturar a pedra 1, mas ele morrerá primeiro. Note que, o melhor movimento do seu adversário (jogar 1 para viver como Preto) é também seu melhor movimento.}

\problemAnswerDiagram
  {dois_olhos}
  {dois_olhos.2}
  {Talvez o grupo estar na lateral, e não no canto, mude o seu status.}
  {\emph{Correto.} Não, é exatamente a mesma forma do problema anterior.}
  {\emph{Incorreto.} E o mesmo movimento para matar.}

\problemAnswerDiagram
  {dois_olhos}
  {dois_olhos.3}
  {Cercar 3 intersecções é insuficiente para viver, a não ser que seja seu turno. E se estivermos cercando de 4 a 5?}
  {\emph{Correto.} Cercar 4 é suficiente, pois é impossível que Branco faça com que todo o espaço vire algo indivisível. Se Branco jogar em 3, Preto responderá em 2.}
  {\emph{Incorreto.} Branco em 2 é uma tática ligeiramente mais avançada --- e que veremos em breve --- para criar um olho falso. Preto acaba efetivamente cercando somente 3 intersecções.}

\problemAnswerDiagram
  {dois_olhos}
  {dois_olhos.6}
  {Preto precisa de mais um movimento no canto para garantir sua vida?}
  {\emph{Correto.} Sim, é preciso reforçar, pois é preciso criar dois espaços separados.}
  {\emph{Incorreto.} Branco pode impossibilitar a separação do espaço interno, efetivamente criando um olho grande, o que é insuficiente para Preto viver.}

\problemAnswerDiagram
  {dois_olhos}
  {dois_olhos.5}
  {Esta é uma outra forma que aparece com frequência em partidas. Como completá-la para que viva?}
  {\emph{Correto.} Preto configurará um olho no extremo canto, e outro na diagonal.}
  {\emph{Incorreto.} Ao descer, Branco pega o bom movimento preto para si, e faz com que o espaço interno não possa ser dissociado.}