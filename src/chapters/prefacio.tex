\chapter{Prefácio}

\small
  
Há 3 pilares para se aprimorar no Go, qual seja o seu nível, desde iniciante até campeão mundial: jogar, revisar e praticar \emph{tsumegos}. Para quem não os conhece, tsumegos são problemas localizados no jogo de Go.

Evidentemente, jogando e revisando é possível praticar tsumegos, mas o volume por unidade de tempo será bem mais baixo, já que, em partidas, há muito mais com o que se preocupar. Tsumegos são o equivalente de musculação para atletas de esportes físicos: uma maneira eficiente de se treinar em algo específico relacionado ao esporte desejado.

O propósito principal deste livro e da coleção \emph{Tsumegos para Iniciantes} não é, no entanto, ser uma coletânea exaustiva de tsumegos para iniciantes. O intuito aqui é servir de referência de técnicas e padrões essenciais, além de ter um volume mínimo para que se possa ser considerado como uma sessão de treinamento. Inclusive, como treinamento, sugiro fortemente refazer o livro múltiplas vezes, provavelmente espaçado por uma ou mais semanas. Repetição será essencial para tornar as técnicas intuitivas.

Caso o leitor queira um maior volume e diversidade de tsumegos, hoje em dia, não há recurso melhor do que o \shref{https://101weiqi.com}{101weiqi.com}, com mais de 160,000 problemas aprovados por moderadores. Outro recurso importante é o servidor OGS (\shref{https://online-go.com}{online-go.com}), que também possui muitos tsumegos, além de uma excelente página de referências: \shref{https://online-go.com/docs/other-go-resources}{online-go.com/docs/other-go-resources}.

\bigskip

Para quem não me conhece, sou Philippe Fanaro, e jogo Go desde o final de 2012. Ao redor de quando escrevo este livro, oscilo entre 2 e 3 dan no servidor KGS, ou algo entre 5 e 6 dan no servidor Fox.

Venho compartilhando conteúdo de Go online especialmente pelo meu canal de YouTube em português: \shref{https://youtube.com/@Fanaro}{youtube.com/@Fanaro}. Mas, também, pelo meu blog: \shref{https://fanaro.io}{fanaro.io}.

Outro conteúdo útil a mencionar é a tradução do livro \emph{Como Jogar Go --- Uma Introdução Concisa}, de Richard Bozulich e James Davies, que eu fiz em código aberto alguns anos atrás. Ela está disponível aqui, e seu PDF é gratuito: \shref{https://github.com/FanaroEngineering/traducao\_como\_jogar\_go}{github.com/FanaroEngineering/traducao_como_jogar_go}.

Por fim, especialmente para quem mora em São Paulo-SP, há a Brasil Nihon Kiin, a sede da associação japonesa de Go (e shogui) no Brasil. Eles também possuem presença online.

\bigskip
\bigskip

Philippe Fanaro

12 de Setembro de 2024

%-----------------------------------------------------------
% Preto a Jogar

% \pagebreak
% \pagebreak
% \newpage\null\thispagestyle{empty}\newpage
\newpage 

\ % The empty page

\newpage

\topskip0pt
\vspace*{\fill}

\begin{center}
  \Large \emph{Sempre Preto a jogar.}
\end{center}

\topskip0pt
\vspace*{\fill}

%-----------------------------------------------------------