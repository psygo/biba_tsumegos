\chapter{Prefácio}

Há 3 pilares para se aprimorar no Go, qual seja o seu nível, desde iniciante até campeão mundial: jogar, revisar e praticar \emph{tsumegos}. Para quem não os conhece, tsumegos são problemas localizados, no jogo de Go.

Evidentemente, jogando e revisando também é possível praticar tsumegos, mas o volume por unidade de tempo será bem mais baixo, já que, em partidas, há muito mais com o que se preocupar. Tsumegos são o equivalente de musculação para atletas de esportes físicos: uma maneira eficiente de se treinar algo específico relacionado ao esporte desejado.

O propósito principal deste livro, e da coleção \emph{Tsumegos para Iniciantes}, é servir de referência de técnicas e padrões essenciais para aqueles que acabaram de aprender as regras ou já jogaram suas primeiras dezenas, ou mesmo centenas, de partidas. E criar uma ponte entre a visão de alguém que está começando e a de um jogador mais avançado, além de ter um volume mínimo de exercícios para que se possa ser considerado como uma sessão de treinamento. Inclusive, como treinamento, sugiro fortemente refazer o livro múltiplas vezes, provavelmente espaçadas por uma ou mais semanas. Repetição será essencial para tornar as técnicas intuitivas: errar na primeira vez será fácil, errar depois da décima garanto que será muito mais difícil!

Caso o leitor queira um maior volume e diversidade de tsumegos, hoje em dia, não há recurso melhor do que o \shref{https://101weiqi.com}{101weiqi.com}, com mais de 160,000 problemas aprovados por moderadores. A interface desse site está em chinês, mas ela pode ser traduzida para o inglê em grande parte pela extensão de navegador \shref{https://chromewebstore.google.com/detail/101weiqilocalizer/emhhlhigmokehndjjmgnailciakdmoba}{101weiqiLocalizer} (o Google Tradutor também ajuda bastante).

E outro recurso importante é o servidor OGS (\shref{https://online-go.com}{online-go.com}), que também possui muitos tsumegos, além de uma excelente página de referências.

\bigskip
\bigskip

Para aqueles que não me conhecem, jogo Go desde o final de 2012. Ao redor de quando escrevo este livro, oscilo entre 2 e 3 dan --- \emph{dan} é o equivalente de mestre ou faixa-preta no Go, na verdade, o termo faixa-preta foi inspirado pelo Go! --- no servidor KGS, ou algo ao redor de 5 dan no servidor Fox.

Há já pelo menos 5 anos, compartilho conteúdo de Go online especialmente pelo meu canal de YouTube em português: \shref{https://youtube.com/@Fanaro}{youtube.com/@Fanaro}. Mas, também, pelo meu blog: \shref{https://fanaro.io}{fanaro.io}; e meu novo canal em inglês: \shref{https://www.youtube.com/@gowithfanaro}{youtube.com/@gowithfanaro}.

Outro conteúdo útil a mencionar é a tradução do livro \emph{Como Jogar Go --- Uma Introdução Concisa}, de Richard Bozulich e James Davies, que eu fiz em código aberto alguns anos atrás. Ela está disponível aqui, e seu PDF é gratuito: \shref{https://github.com/FanaroEngineering/traducao\_como\_jogar\_go}{github.com/FanaroEngineering/traducao_como_jogar_go}.

Como recurso extra, para aqueles que moram em ou perto de São Paulo-SP, existe a Brasil Nihon Kiin, que é a sede da associação japonesa de Go (e shogui) no Brasil. Além de possuírem muitos membros que são jogadores avançados, eles também oferecem aulas, grupos de estudos, torneios, livros, e fazem uma interface com a cultura japonesa no Brasil --- eles também possuem presença online.

Por fim, gostaria de agradecer aos meus amigos e revisores que ajudaram a melhorar este livro com sugestões de problems, diagramas e correções de português. São eles: Samuel Karasin, Felipe Herman van Riemsdijk e Renan Cruz.

\bigskip
\smallskip
\smallskip
\smallskip

\hspace*{\fill} Philippe Fanaro \hspace{0.115cm}

\hspace*{\fill} Outubro de 2024 \hspace{0.05cm}