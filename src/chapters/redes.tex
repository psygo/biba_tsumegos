\chapter{Redes}

\emptypage

\problemAnswerDiagram
  {redes}
  {redes.1}
  {É possível capturar a pedra branca incondicionalmente com somente um movimento?}
  {\emph{Correto.} Redes já são um tópico bastante complexo para quem está começando. Com o movimento 1, Preto aprisiona a pedra branca, não há escapatória, e nem como ganhar mais liberdades.}
  {\emph{Incorreto.} Capturar assim pode ser uma opção em alguns casos específicos, mas, em geral, não é ideal pois é uma captura condicional por escada. Jogar em A também configuraria uma escada, mas em outra direção.}


\problemAnswerDiagram
  {redes}
  {redes.2}
  {Capturar pedras de corte é frequentemente algo de extrema importância no Go.}
  {\emph{Correto.} Novamente, Preto não consegue aumentar suas liberdades ou utilizar fraquezas no exterior preto para poder escapar.}
  {\emph{Incorreto.} Seja esta escada ou a de A, ambas são capturas condicionais, o que é quase sempre subótimo.}

\problemAnswerDiagram
  {redes}
  {redes.3}
  {A parede branca à esquerda parece dificultar as coisas para o Preto. Ou não?}
  {\emph{Correto.} Não muda em nada em relação à rede do problema 1.}
  {\emph{Incorreto.} Não há escada!}

\problemAnswerDiagram
  {redes}
  {redes.4}
  {O grupo preto mais abaixo está um pouco pressionado pelo grupo branco do canto, isso muda a rede a ser aplicada?}
  {\emph{Correto.} O grupo preto possui 3 liberdades, o que é o suficiente para se garantir uma rede padrão.}
  {\emph{Incorreto.} No geral, 1 é incorreto, pois acaba em uma escada que vai para o lado. Mas escadas indo para o lado podem ser ocasionalmente melhores do que uma rede.}

\problemAnswerDiagram
  {redes}
  {redes.5}
  {Esta já é uma rede bem mais avançada.}
  {\emph{Correto.} O importante é que Branco não vai conseguir mais de 2 liberdades com a pedra de corte, o que é menos do que as pedras pretas que a cercam.}
  {\emph{Incorreto.} Às vezes, 1 pode sera a rede ótima, mas, neste caso, Branco está muito forte no exterior.}