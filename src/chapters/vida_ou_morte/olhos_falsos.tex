\chapter{Olhos Falsos}

\emptypage

\problemAnswerDiagram
  {vida_ou_morte/olhos_falsos}
  {olhos_falsos.4}
  {Preto protegeu parte dos olhos do canto. É o suficiente?}
  {\emph{Correto.} Preto precisa de mais uma proteção para garantir seus dois olhos.}
  {\emph{Incorreto.} Branco pode falsificar o segundo olho com um sacrifício.}

\problemAnswerDiagram
  {vida_ou_morte/olhos_falsos}
  {olhos_falsos.1}
  {Tão perto de viver quanto de morrer.}
  {\emph{Correto.} Ao conectar, Preto possui espaço interno o suficiente para criar dois olhos. Se Branco jogar em A, Preto joga em B, e vice-versa.}
  {\emph{Incorreto.} A intersecção de A vira olho falso.}

\problemAnswerDiagram
  {vida_ou_morte/olhos_falsos}
  {olhos_falsos.2}
  {Um problema bastante traiçoeiro.}
  {\emph{Correto.} Preto protege seus dois olhos, em A e B.}
  {\emph{Incorreto.} Ao jogar 2, Branco falsifica o olho de A diretamente, e Preto terá que conectar em B mais tarde, ou seja, também é olho falso!}