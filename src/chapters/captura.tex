\chapter{Captura}

\emptypage

\problemAnswerDiagram
  {captura}
  {captura.1}
  {É possível capturar a pedra branca? (Em livros de tsumegos, costuma-se padronizar sempre Preto a jogar.)}
  {\emph{Correto.} Sim, é possível capturar a pedra branca diretamente, já que ela só possui 1 liberdade. (Note que a pedra branca será retirada do tabuleiro.)}
  {\emph{Variação.} Mesmo se Branco tentar resgatar ou contra-capturar a pedra preta, sua pedra do canto ainda possui somente 1 liberdade e pode ser capturada.}
  
\problemAnswerDiagram
  {captura}
  {captura.2}
  {E se a pedra branca estiver no lado ao invés do canto, ainda é possível capturá-la?}
  {\emph{Correto.} Branco não deveria jogar 2, pois não há para onde fugir. Suas pedras terão sempre 1 liberdade, e vão colidir com o canto esquerdo no final.}
  {\emph{Variação.} Preto pode sempre fazer \emph{tenuki}, isto é, ignorar movimentos locais e jogar em outro lugar, mas aí é Branco quem poderá capturar a pedra preta.}

\problemAnswerDiagram
  {captura}
  {captura.3}
  {Agora a pedra branca está no centro. Ela ainda pode ser capturada?}
  {\emph{Correto.} A captura da pedra branca configura uma forma chamada \emph{ponnuki}, que é o número mínimo de pedras para se capturar uma pedra adversária, quando ela não está nem no canto e nem na borda. (Note que a pedra branca será retirada do tabuleiro.)}
  {\emph{Variação.} Ao escapar com esta pedra, Branco vai de 1 liberdade para 3, o que é bastante eficiente, isto é, 2 liberdades por movimento; em lutas, liberdades são talvez o bem mais crucial. Desta maneira, Branco também expõe os cortes A e B no exterior preto.}

\problemAnswerDiagram
  {captura}
  {captura.8}
  {Há uma pedra branca solitária no topo. O que Preto pode fazer com ela?}
  {\emph{Correto.} Preto pode capturá-la e, assim, resolver o problem do corte que existia em A.}
  {\emph{Incorreto.} Preto poderia ter capturado a pedra, o que resolveria o corte de uma maneira mais eficiente, sem jogar em seu próprio território. E esta defesa dá chances para que Branco conecte todas as suas pedras mais tarde.}

\problemAnswerDiagram
  {captura}
  {captura.9}
  {Capturas no canto são geralmente de extremo valor pois podem garantir não somente território, mas, também, o espaço vital de grupos, como veremos mais tarde nos exercícios de vida ou morte deste livro.}
  {\emph{Correto.} A captura da pedra do canto limpa o canto para Preto, e ainda deixa a pedra A com somente 1 liberdade.}
  {\emph{Incorreto.} Localmente pelo menos, não jogar aqui como Preto ajuda Branco a se estabilizar localmente, enquanto contra-ataca severamente.}

\problemAnswerDiagram
  {captura}
  {captura.7}
  {O grupo branco mais ao topo possui 2 pedras. Isso muda algo em relação à possibilidade de captura?}
  {\emph{Correto.} Apesar de o grupo branco ter mais pedras, ele ainda possui somente 1 liberdade. Esta forma é conhecida como ``casco de tartaruga'', que é o número mínimo de pedras necessário para se capturar 2 pedras. Também era possível capturar com A. (Note que ambas as pedras brancas serão retiradas do tabuleiro.)}
  {\emph{Variação.} Ao fugir, Branco não somente resgata suas pedras como expõe múltiplas fraquezas no exterior preto.}

\problemAnswerDiagram
  {captura}
  {captura.10}
  {Ambos os lados estão, localmente, em uma situação crítica.}
  {\emph{Correto.} Preto captura dois pontos, praticamente garante o canto, e a pedra marcada é indiretamente engolida.}
  {\emph{Variação.} Branco reverte para uma captura para si, e o grupo preto agora está flutuando e instável.}

\problemAnswerDiagram
  {captura}
  {captura.23}
  {Há mais de 1 grupo preto prestes a ser capturado.}
  {\emph{Correto.} Preto captura as principais pedras de corte e resolve todos os seus problemas.}
  {\emph{Incorreto.} Resgatar pedras diretamente é muitas vezes o melhor momento. Mas por que resgatar se podemos corrigir capturando diretamente?}

\problemAnswerDiagram
  {captura}
  {captura.24}
  {Há mais de 1 grupo preto prestes a ser capturado.}
  {\emph{Correto.} Preto captura as principais pedras de corte e resolve todos os seus problemas.}
  {\emph{Incorreto.} Resgatar pedras diretamente é muitas vezes o melhor momento. Mas por que resgatar se podemos corrigir capturando diretamente?}

\chapter{Duplo-Atari}

\emptypage

\problemAnswerDiagram
  {captura}
  {captura.25}
  {É difícil imaginar uma situação no começo do jogo em que Preto não responderia neste local.}
  {\emph{Correto.} Além de Preto limpar os problemas de corte de sua forma, a pedra branca marcada é completamente inutilizada.}
  {\emph{Incorreto.} Ao conectar suas pedras, Branco expõe múltiplos cortes e disconecta a pedra A.}

\chapter{Suicídio}

\emptypage

\problemAnswerDiagram
  {captura}
  {captura.4}
  {Preto pode conectar suas pedras?}
  {\emph{Correto.} Conectar as pedras zeraria as liberdades de todas as pedras do grupo, o que as automaticamente capturaria, ou seja, seria suicídio, uma jogada inválida.}
  {\emph{Variação.} Branco tem sempre a opção de capturar ambas as pedras.}

\problemAnswerDiagram
  {captura}
  {captura.5}
  {Preto pode capturar as pedras brancas?}
  {\emph{Correto.} Note que não é suicídio jogar em 1, pois a regra da captura possui precedência. Isto é, primeiro aplicamos a regra da captura, se possível, e, só depois, examinamos se é suicídio. E segue que, se algo for capturado, haverá mais de uma liberdade.}
  {\emph{Variação.} Mais tarde, se Branco conseguir pedras no exterior, quem pode ser capturado é o Preto! Antes de 1, Preto poderia finalmente capturar o canto, e, dessa maneira, evitar de ser capturado.}

\chapter{Recaptura}

\emptypage

\problemAnswerDiagram
  {captura}
  {captura.6}
  {É possível capturar algo branco? Ou é suicídio?}
  {\emph{Correto.} Sim, é possível capturar duas pedras brancas. Em seguida, a pedra preta que efetua a captura estará imediatamente em \emph{atari}, isto é, ela possuirá somente 1 liberdade, então Branco terá a opção de uma contra- ou recaptura.}
  {\emph{Variação.} Caso Preto opte por não capturar, Branco pode salvar as pedras e deixar Preto com fraquezas no exterior. Isso não quer dizer que não capturar é um erro, pois pode haver outros movimentos mais importantes no tabuleiro.}

\chapter{Captura na Segunda Linha}

\emptypage

\problemAnswerDiagram
  {captura}
  {captura.11}
  {Localmente, uma situação assim deveria ser o equivalente a alarmes soando em uma base militar.}
  {\emph{Correto.} Preto estabiliza seu grupo e fragiliza completamente as pedras brancas.}
  {\emph{Incorreto.} Branco reverte a situação, e é agora Preto quem está desmoronando.}

\problemAnswerDiagram
  {captura}
  {captura.13}
  {Um problema similar ao anterior. Você jogaria no mesmo lugar?}
  {\emph{Correto.} Desta vez, capturamos por fora, pois, caso contrário...}
  {\emph{Incorreto.} Ao fazer atari por fora, as pedras pretas ficam em atari.}

\problemAnswerDiagram
  {captura}
  {captura.14}
  {Esta é a situação que gera, com frequência, os cenários dos 2 problemas anteriores. Este é um padrão bastante comum no Go.}
  {\emph{Correto.} Preto captura a pedra da segunda linha pois ela não tem como estender suas liberdades. Este padrão é um dos principais motivos por invasões na segunda linha raramente funcionarem.}
  {\emph{Variação.} Branco não tem como salvar suas pedras.}

\chapter{Pedras de Corte}

\emptypage

\problemAnswerDiagram
  {captura}
  {captura.12}
  {Salvar as pedras pretas não somente é uma quantia considerável de pontos, mas uma maneira de contra-atacar.}
  {\emph{Correto.} Com 2 e 4, Branco consegue espremer Preto --- em inglês, esta técnica é conhecida como ``squeeze'' ---, mas isso ainda não corrige os cortes de A a E.}
  {\emph{Incorreto.} Preto 2 parece ser uma rede --- veremos esta técnica um pouco mais à frente ---, e uma forma mais bonita e eficiente, mas as pedras pretas não possuem liberdades suficientes para capturar em rede.}

\problemAnswerDiagram
  {captura}
  {captura.15}
  {Preto precisa proteger dois lados ao mesmo tempo. É mesmo possível?}
  {\emph{Correto.} As pedras pretas A possuem liberdades o suficiente.}
  {\emph{Incorreto.} Branco não só garante o canto como o mata o grupo inteiro.}

\problemAnswerDiagram
  {captura}
  {captura.22}
  {Uma situação bastante confusa, com múltiplos grupos cortados.}
  {\emph{Correto.} Com esta captura, Preto gera liberdades para o grupo A, que estava em estado crítico, e também basicamente captura as pedras marcadas.}
  {\emph{Incorreto.} Primeiramente, jogar em 1 é auto-atari --- quando o próprio jogador se põe em atari, ``self-atari'' em inglês --- nas pedras A. Mas Branco pode ir além e capturar as pedras do topo.}

\problemAnswerDiagram
  {captura}
  {captura.19}
  {Preto está quase conseguindo conectar seus grupos.}
  {\emph{Correto.} Branco não consegue fugir.}
  {\emph{Incorreto.} A intenção foi boa, mas a execução falhou.}

\problemAnswerDiagram
  {captura}
  {captura.20}
  {Preto pode se fortalecer um pouco mais enquanto ataca Branco.}
  {\emph{Correto.} Preto retira o olho branco, garante mais pontos, e ainda protege um de seus cortes.}
  {\emph{Incorreto.} Mais tarde, ou mesmo em breve, Branco pode garantir um olho e expor os cortes A e B ao mesmo tempo.}

\problemAnswerDiagram
  {captura}
  {captura.21}
  {É imprescindível contar as liberdades dos grupos que participam de lutas.}
  {\emph{Correto.} Preto consegue capturar as pedras de corte.}
  {\emph{Incorreto.} Em geral, quando uma sequência não funciona, o melhor é simplesmente não jogá-la. Se jogamos 1 e percebemos que não funciona, é melhor não jogar 3 ou 5. No mínimo, poderíamos utilizar 1, 3 e 5 como ameaças de ko no futuro.}

\chapter{Conectar e Morrer}

\emptypage

\problemAnswerDiagram
  {captura}
  {captura.17}
  {Branco parece seguro, mas há um problema gigante em sua forma.}
  {\emph{Correto.} Se Branco conectar em 2, é capturado. Esta técnica é conhecida como ``connect and die'', ou ``conectar e morrer'', apesar de que muitos diriam que é mais um problema de falta de liberdades.}
  {\emph{Incorreto.} Em algum momento, Branco pode jogar em 1 para viver.}

\problemAnswerDiagram
  {captura}
  {captura.18}
  {Preto precisa de um milagre, e rápido.}
  {\emph{Correto.} Branco não consegue conectar em 3, pois a pedra preta em A garante um auto-atari. Este é um exemplo muito mais emblemático de ``conectar e morrer''.}
  {\emph{Incorreto.} Capturar em 2 corrige os problemas para Branco.}

\problemAnswerDiagram
  {captura}
  {captura.16}
  {Mais uma situação extremamente suspeita em termos de liberdades.}
  {\emph{Correto.} Branco não consegue conectar em 3 diretamente pois seria auto-atari!}
  {\emph{Incorreto.} Caso Preto não perceba o problema, Branco toma o bom movimento preto para si. Branco A também funcionaria, mas essencialmente tem o mesmo efeito no canto nesta situação, além de retirar uma liberdade das pedras pretas no exterior.}