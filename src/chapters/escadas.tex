\chapter{Escadas}

\emptypage

\problemAnswerDiagram
  {escadas}
  {escadas.1}
  {Será que há mais de uma maneira de se capturar a pedra branca?}
  {\emph{Correto.} Este padrão é conhecido como \emph{shicho} em japonês, ou ``escada'' em português. Ao chegar na borda do tabuleiro, Branco não conseguirá mais estender suas liberdades.}
  {\emph{Incorreto.} Se a configuração do problema estivesse deslocada uma linha para a direita, jogar 1 seria uma opção, mas, nesta situação, jogar 1 é um desastre.}

\problemAnswerDiagram
  {escadas}
  {escadas.2}
  {Agora, Branco possui uma pedra no caminho da escada. Isso muda alguma coisa?}
  {\emph{Correto.} A pedra branca ajudará a estender as liberdades das pedras sob a escada, impossibilitando sua conclusão. Caso a escada não funcione, jogar o padrão é um dano indireto pois Preto fica com múltiplos ataris duplos no exterior.}
  {\emph{Variação.} O melhor que Preto pode fazer em uma situação destas, localmente, é um compromisso ou negociação.}


\problemAnswerDiagram
  {escadas}
  {escadas.3}
  {Em relação ao problema anterior, a pedra branca agora está deslocada para o topo. A escada funciona desta vez?}
  {\emph{Correto.} Ainda não funciona. A pedra branca A colocará a pedra B em atari em algum momento.}
  {\emph{Variação.} Novamente, o melhor que Preto pode fazer em uma situação destas, localmente, é um compromisso ou negociação. Não há tanto o que reclamar, pois a pedra branca marcada foi bastante danificada. Branco pode decidir jogar de outra maneira, caso a pedra branca marcada seja importante.}