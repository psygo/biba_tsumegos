\documentclass[12pt]{book}

% A4
\usepackage[
  paperheight   = 297mm, 
  paperwidth    = 210mm,
  bindingoffset = 10mm,
  left          = 7.5mm,
  right         = 12.5mm,
  top           = 20mm,
  bottom        = 15mm,
  footskip      = 5mm,
]{geometry}

\usepackage{config}

\begin{document}
  \haengmaTitlePage{3}
  
  \tableofcontents
  
  \chapter{How to Respond (응수하는 방법)}
  
  \emptypage

  \haengmaProblem{haengma_3/1/1_2.sgf}
  \haengmaProblem{haengma_3/1/3_4.sgf}
  \haengmaProblem{haengma_3/1/5_6.sgf}
  \haengmaProblem{haengma_3/1/7_8.sgf}
  \haengmaProblem{haengma_3/1/9_10.sgf}
  \haengmaProblem{haengma_3/1/11_12.sgf}
  \haengmaProblem{haengma_3/1/13_14.sgf}
  \haengmaProblem{haengma_3/1/15_16.sgf}
  \haengmaProblem{haengma_3/1/17_18.sgf}
  \haengmaProblem{haengma_3/1/19_20.sgf}
  \haengmaProblem{haengma_3/1/21_22.sgf}
  \haengmaProblem{haengma_3/1/23_24.sgf}
  \haengmaProblem{haengma_3/1/25_26.sgf}
  \haengmaProblem{haengma_3/1/27_28.sgf}
  \haengmaProblem{haengma_3/1/29_30.sgf}
  \haengmaProblem{haengma_3/1/31_32.sgf}
  \haengmaProblem{haengma_3/1/33_34.sgf}
  \haengmaProblem{haengma_3/1/35_36.sgf}

  \chapter{How to Connect Your Stones (돌의 연결과 이음)}
  
  \emptypage
  
  \setcounter{problemCounter}{0}

  \haengmaProblem{haengma_3/2/1_2.sgf}

  \chapter{How to Catch a Stone (돌을 잡는 방법)}
  
  \emptypage
  
  \setcounter{problemCounter}{0}

  \haengmaProblem{haengma_3/3/1_2.sgf}
  \haengmaProblem{haengma_3/3/27_28.sgf}
  \haengmaProblem{haengma_3/3/29_30.sgf}
\end{document}
