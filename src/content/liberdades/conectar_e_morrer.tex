\chapter{Conectar e Morrer}

\emptypage

\problemAnswerDiagram
  {liberdades/conectar_e_morrer}
  {conectar_e_morrer.1}
  {Branco parece seguro, mas há um problema gigante em sua forma.}
  {\emph{Correto.} Se Branco conectar em 2, é capturado. Esta técnica é conhecida como ``connect and die'', ou ``conectar e morrer'', apesar de que muitos diriam que é mais um problema de falta de liberdades.}
  {\emph{Incorreto.} Em algum momento, Branco pode jogar em 1 para viver.}

\problemAnswerDiagram
  {liberdades/conectar_e_morrer}
  {conectar_e_morrer.2}
  {Preto precisa de um milagre, e rápido.}
  {\emph{Correto.} Branco não consegue conectar em 3, pois a pedra preta em A garante um auto-atari. Este é um exemplo muito mais emblemático de ``conectar e morrer''.}
  {\emph{Incorreto.} Capturar em 2 corrige os problemas para Branco.}

\problemAnswerDiagram
  {liberdades/conectar_e_morrer}
  {conectar_e_morrer.3}
  {Mais uma situação extremamente suspeita em termos de liberdades.}
  {\emph{Correto.} Branco não consegue conectar em 3 diretamente pois seria auto-atari!}
  {\emph{Incorreto.} Caso Preto não perceba o problema, Branco toma o bom movimento preto para si. Branco A também funcionaria, mas essencialmente tem o mesmo efeito no canto nesta situação, além de retirar uma liberdade das pedras pretas no exterior.}

\clearedpage
\clearedpage