\chapter{Conectar e Morrer}

\emptypage

\problemAnswerDiagram[%
  sgf folder      = liberdades/conectar_e_morrer,
  sgf filename    = conectar_e_morrer.2,
  problem text    = {Preto precisa de um milagre, e rápido.},
  answer text one = {\emph{Correto.} Branco não consegue conectar em 2, pois a pedra preta em 1 garante um auto-atari. Este é um exemplo emblemático de ``conectar e morrer''.},
  answer text two = {\emph{Variação.} Conectar em 1 corrige os problemas para Branco. Tanto A quanto B também corrigiriam o problema.},
]

\problemAnswerDiagram[%
  sgf folder               = liberdades/conectar_e_morrer,
  sgf filename             = conectar_e_morrer.3,
  problem text             = {Uma situação extremamente suspeita em termos de liberdades.},
  answer text one          = {\emph{Correto.} Branco não consegue conectar em 3 diretamente pois seria auto-atari!},
  answer text two          = {\emph{Incorreto.} Caso Preto não perceba o problema, Branco captura em 2 para se prevenir, o que mata o grupo preto do canto também.},
  answer diagram clip vert = 10,
]

\problemAnswerDiagram[%
  sgf folder      = liberdades/conectar_e_morrer,
  sgf filename    = conectar_e_morrer.4,
  problem text    = {Será que Preto ainda consegue viver no canto? Ou a solução está em outro lugar?},
  answer text one = {\emph{Correto.} Depois de Preto 1, se Branco conectar em A, Preto captura tudo jogando em 2. E não há como fazer dois olhos no canto.},
  answer text two = {\emph{Incorreto.} Talvez Preto 1 pareça bom, mas Branco 2 limpa todos os problemas para Branco.},
]

\problemAnswerDiagram[%
  sgf folder      = liberdades/conectar_e_morrer,
  sgf filename    = conectar_e_morrer.1,
  problem text    = {Branco parece forte e seguro, mas há um problema gigante em sua forma.},
  answer text one = {\emph{Correto.} Se Branco conectar em 3, terá tudo capturado com A. Esta técnica é conhecida como ``connect and die'' em inglês, apesar de que muitos diriam que é somente mais um exemplo de escassez ou falta de liberdades. Como benefício extra, a pedra B termina em atari também.},
  answer text two = {\emph{Incorreto.} Preto nem sequer conseguirá viver no canto desta maneira.},
]

\problemAnswerDiagram[%
  sgf folder      = liberdades/conectar_e_morrer,
  sgf filename    = conectar_e_morrer.5,
  problem text    = {Lembre que o canto pode gerar mais liberdades do que o esperado, se não tomarmos cuidado.},
  answer text one = {\emph{Correto.} Branco não consegue se conectar com a pedra A pois permaneceria sob atari de qualquer maneira.},
  answer text two = {\emph{Incorreto.} Jogar em 1 parece também reduzir as liberdades brancas, entretanto, Preto não conseguirá se aproximar diretamente.},
]

\clearedpage
\clearedpage