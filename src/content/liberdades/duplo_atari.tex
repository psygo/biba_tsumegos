\chapter{Duplo-Atari}

\emptypage

\problemAnswerDiagram[%
  sgf folder      = liberdades/duplo_atari,
  sgf filename    = duplo_atari.1,
  problem text    = {É difícil imaginar uma situação no começo do jogo em que Preto não responderia neste local.},
  answer text one = {\emph{Correto.} Além de Preto limpar os problemas de corte de sua forma, a pedra branca marcada é completamente inutilizada.},
  answer text two = {\emph{Variação.} Ao conectar suas pedras, Branco expõe múltiplos cortes e desconecta a pedra A.},
]

\problemAnswerDiagram[%
  sgf folder      = liberdades/duplo_atari,
  sgf filename    = duplo_atari.3,
  problem text    = {Nem sempre será fácil perceber este tipo de problema em partidas reais.},
  answer text one = {\emph{Correto.} Branco tem seus grupos totalmente atravessados. Note que, se Branco conseguir proteger em 1, Preto pode sofrer um ataque no topo, pois seu grupo ainda não está completamente vivo.},
  answer text two = {\emph{Variação.} O canto é completamente destruído. Ambas as variações são devastadoras contra Branco.},
]

\problemAnswerDiagram[%
  sgf folder      = liberdades/duplo_atari,
  sgf filename    = duplo_atari.2,
  problem text    = {Branco foi para a festa enquanto sua casa pega fogo.},
  answer text one = {\emph{Correto.} Branco na verdade possui múltiplas maneiras de minimizar o dano, mas isso não muda o fato de que é um desastre localmente. Se Branco achar que o topo é mais importante, pode continuar com algo como A depois de Preto 3.},
  answer text two = {\emph{Variação.} Uma outra opção para ambos, dentre outras. Branco, em seguida, tentará salvar o grupo A.},
]

\problemAnswerDiagram[%
  sgf folder      = liberdades/duplo_atari,
  sgf filename    = duplo_atari.5,
  problem text    = {Após Preto 1, Branco reage com 2, o que parece natural. Mas há algo de errado aqui.},
  answer text one = {\emph{Correto.} Branco sofre um duplo-atari, e a pedra marcada acaba sendo perdida.},
  answer text two = {\emph{Incorreto.} Ingenuamente bloquear o canto garante a chance que Branco queria para corrigir seus problemas.},
]

\problemAnswerDiagram[%
  sgf folder      = liberdades/duplo_atari,
  sgf filename    = duplo_atari.4,
  problem text    = {Às vezes, nossa intuição pode ser uma armadilha.},
  answer text one = {\emph{Correto.} Branco não pode conectar em 3 pois seria auto-atari. E o importante neste caso é capturar as 2 pedras brancas à esquerda, o que danifica e separa a pedra marcada. E o canto nem sequer está claramente vivo ainda.},
  answer text two = {\emph{Incorreto.} Ao jogar o duplo-atari, Preto oferece mais opções a Branco, que protegerá as pedras do topo, de maior importância.},
]

\clearedpage
\clearedpage