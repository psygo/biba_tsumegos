\chapter{Duplo-Atari}

\emptypage

\problemAnswerDiagram
  {liberdades/duplo_atari}
  {duplo_atari.1}
  {É difícil imaginar uma situação no começo do jogo em que Preto não responderia neste local.}
  {\emph{Correto.} Além de Preto limpar os problemas de corte de sua forma, a pedra branca marcada é completamente inutilizada.}
  {\emph{Incorreto.} Ao conectar suas pedras, Branco expõe múltiplos cortes e disconecta a pedra A.}

\problemAnswerDiagram
  {liberdades/duplo_atari}
  {duplo_atari.3}
  {Nem sempre será fácil de perceber este tipo de problema em partidas reais.}
  {\emph{Correto.} Branco tem seus grupos totalmente atravessados. Note que, se Branco conseguir proteger em 1, Preto pode sofrer um ataque no topo, pois seu grupo ainda não está completamente vivo.}
  {\emph{Variação.} O canto é completamente destruído.}

\problemAnswerDiagram
  {liberdades/duplo_atari}
  {duplo_atari.2}
  {Branco foi para a festa enquanto sua casa pega fogo.}
  {\emph{Correto.} Branco na verdade possui múltiplas maneiras de minimizar o dano, mas isso não muda o fato de que é um desastre localmente. Se Branco achar que o topo é o grande, pode continuar com A depois de Preto 3.}
  {\emph{Variação.} Branco pode optar por tentar salvar o canto e depois fugir com as pedras A.}

\clearedpage
\clearedpage