\chapter{Duplo-Atari}

\emptypage

\problemAnswerDiagram[%
  sgf folder      = liberdades/duplo_atari,
  sgf filename    = duplo_atari.1,
  problem text    = {É difícil imaginar uma situação no começo do jogo em que Preto não responderia neste local.},
  answer text one = {\emph{Correto.} Além de Preto limpar os problemas de corte de sua forma, a pedra branca marcada é completamente inutilizada.},
  answer text two = {\emph{Incorreto.} Ao conectar suas pedras, Branco expõe múltiplos cortes e disconecta a pedra A.},
]

\problemAnswerDiagram[%
  sgf folder      = liberdades/duplo_atari,
  sgf filename    = duplo_atari.3,
  problem text    = {Nem sempre será fácil perceber este tipo de problema em partidas reais.},
  answer text one = {\emph{Correto.} Branco tem seus grupos totalmente atravessados. Note que, se Branco conseguir proteger em 1, Preto pode sofrer um ataque no topo, pois seu grupo ainda não está completamente vivo.},
  answer text two = {\emph{Variação.} O canto é completamente destruído. Ambas as variações são devastadoras contra Branco.},
]

\problemAnswerDiagram[%
  sgf folder      = liberdades/duplo_atari,
  sgf filename    = duplo_atari.2,
  problem text    = {Branco foi para a festa enquanto sua casa pega fogo.},
  answer text one = {\emph{Correto.} Branco na verdade possui múltiplas maneiras de minimizar o dano, mas isso não muda o fato de que é um desastre localmente. Se Branco achar que o topo é o grande, pode continuar com A depois de Preto 3.},
  answer text two = {\emph{Variação.} Branco pode optar por tentar salvar o canto e depois fugir com as pedras A.},
]

\clearedpage
\clearedpage