\chapter{Espremer}

\emptypage

\problemAnswerDiagram[%
  sgf folder      = liberdades/espremer,
  sgf filename    = espremer.1,
  problem text    = {Espremer costuma englobar diversas técnicas de captura em uma só sequência. Estritamente, utilizamos o termo mesmo se não capturarmos nada, é mais o tipo de sequência para danificar severamente a forma adversária.},
  answer text one = {\emph{Correto.} Com um sacrifício, podemos configurar uma escada. É realmente algo avançado e difícil de visualizar pela primeira vez.},
  answer text two = {\emph{Incorreto.} Jogar em 1 pode ajudar às vezes, mas aqui não é o suficiente.},
]

\problemAnswerDiagram[%
  sgf folder      = liberdades/espremer,
  sgf filename    = espremer.2,
  problem text    = {Quantas técnicas são necessárias aqui?},
  answer text one = {\emph{Correto.} Talvez 2 técnicas, atari e escada.},
  answer text two = {\emph{Incorreto.} Preto talvez pense que possa armar um ko, mas Branco pode capturar uma vez, mas isso é tudo muito desnecessário e falho.},
]

\problemAnswerDiagram[%
  sgf folder      = liberdades/espremer,
  sgf filename    = espremer.3,
  problem text    = {Um problema que eu já vi amadores quase profissionais errarem, mas que é simples se alguém apontar que há algo de interessante a se fazer.},
  answer text one = {\emph{Correto.} Preto parece que irá partir para um ko, mas a sequência se converte em um ``squeeze''. Se Branco perceber o problema após a captura em 2, é melhor simplesmente parar de jogar na região.},
  answer text two = {\emph{Incorreto.} Capturar limpa a região mas perde uma oportunidade de conseguir algo mais.},
]

\problemAnswerDiagram[%
  sgf folder      = liberdades/espremer,
  sgf filename    = espremer.4,
  problem text    = {Tente aplicar 2 das técnicas que já vimos.},
  answer text one = {\emph{Correto.} Com um sacrifício, podemos então concluir com um ``conectar e morrer''.},
  answer text two = {\emph{Incorreto.} Talvez Preto consiga viver com um ko, mas, na sequência correta, Preto não somente vive como se conecta com o grupo no exterior.},
]

\problemAnswerDiagram[%
  sgf folder      = liberdades/espremer,
  sgf filename    = espremer.5,
  problem text    = {Preto acaba de encontrar um tesuji. Você consegue finalizar a sequência?},
  answer text one = {\emph{Correto.} Preto espreme as pedras brancas de importância, e termina a sequência com um misto de escada e conectar e morrer. Preto 3 em 5 também funciona.},
  answer text two = {\emph{Incorreto.} Uma captura aqui parece gerar uma resposta, mas Branco muda o rumo para o topo e escapa com as pedras de corte.},
]

\problemAnswerDiagram[%
  sgf folder      = liberdades/espremer,
  sgf filename    = espremer.6,
  problem text    = {Uma sequência muito avançada que surge após um joseki que caiu um pouco em desuso após o surgimento da inteligência artificial (IA). No entanto, o tesuji continua muito útil, e análogo ao problema anterior.},
  answer text one = {\emph{Correto.} Preto tem liberdades suficientes no canto para espremer as pedras de corte brancas.},
  answer text two = {\emph{Incorreto.} Parece gerar a mesma sequência que a resposta, mas falha por 1 liberdade.},
]

\clearedpage
\clearedpage