\chapter{Espremer}

\emptypage

\problemAnswerDiagram[%
  sgf folder      = liberdades/espremer,
  sgf filename    = espremer.1,
  problem text    = {``Espremer'' costuma englobar diversas técnicas de captura em uma só sequência.},
  answer text one = {\emph{Correto.} Com o sacrifício de 1 em 6, podemos configurar uma escada. É realmente algo avançado e difícil de visualizar pela primeira vez.},
  answer text two = {\emph{Incorreto.} Jogar em 1 pode ajudar às vezes, mas aqui não é o suficiente.},
]

\problemAnswerDiagram[%
  sgf folder      = liberdades/espremer,
  sgf filename    = espremer.2,
  problem text    = {Quantas técnicas são necessárias aqui?},
  answer text one = {\emph{Correto.} Talvez 2 técnicas: atari e escada.},
  answer text two = {\emph{Incorreto.} Preto talvez pense que possa armar um ko, mas Branco corrige o problema exatamente onde Preto deveria ter jogado.},
]

\problemAnswerDiagram[%
  sgf folder      = liberdades/espremer,
  sgf filename    = espremer.3,
  problem text    = {Um problema que eu já vi amadores quase profissionais errarem, mas que é simples se alguém apontar que há algo de interessante a se fazer.},
  answer text one = {\emph{Correto.} Preto parece que irá partir para um ko, mas a sequência se converte em um ``squeeze'', o termo em inglês para ``espremer''. Se Branco perceber o problema após a captura em 2, é melhor simplesmente parar de jogar na região.},
  answer text two = {\emph{Incorreto.} Capturar limpa a região mas perde uma oportunidade de conseguir algo mais.},
]

\problemAnswerDiagram[%
  sgf folder      = liberdades/espremer,
  sgf filename    = espremer.4,
  problem text    = {Tente aplicar 2 das técnicas que já vimos.},
  answer text one = {\emph{Correto.} Com um sacrifício, podemos então concluir com um ``conectar e morrer''.},
  answer text two = {\emph{Incorreto.} Talvez Preto consiga viver com um ko, mas, na sequência correta, Preto não somente vive como se conecta com o grupo no exterior.},
]

\problemAnswerDiagram[%
  sgf folder      = liberdades/espremer,
  sgf filename    = espremer.5,
  problem text    = {Preto acaba de encontrar um tesuji. Você consegue finalizar a sequência?},
  answer text one = {\emph{Correto.} Preto espreme as pedras brancas de importância, e termina a sequência com um misto de escada e conectar e morrer. Preto 3 em 5 também funciona.},
  answer text two = {\emph{Incorreto.} Uma captura aqui parece gerar uma resposta, mas Branco muda o rumo para o topo e escapa com as pedras de corte.},
]

\problemAnswerDiagram[%
  sgf folder      = liberdades/espremer,
  sgf filename    = espremer.6,
  problem text    = {Uma sequência muito avançada que surge às vezes após um joseki que caiu em desuso com o surgimento da inteligência artificial (IA).},
  answer text one = {\emph{Correto.} Preto tem liberdades suficientes no canto para espremer as pedras de corte brancas. Jogar 3 em 5 também funcionaria.},
  answer text two = {\emph{Incorreto.} Parece gerar a mesma sequência que a resposta, mas falha por 1 liberdade.},
]

\problemAnswerDiagram[%
  sgf folder      = liberdades/espremer,
  sgf filename    = espremer.7,
  problem text    = {Esta posição muitas vezes surge quando temos a forma de ``pulo do elefante'' --- este termo vem do xadrez chinês, \emph{xiangqi}, em que a peça elefante pula 1 espaço na diagonal ---, de A para B.},
  answer text one = {\emph{Correto.} Branco possui um problema de escassez de liberdades. Preto sacrifica 1 pedra para configurar uma sequência de conexão e morte.},
  answer text two = {\emph{Incorreto.} Jogando fora de ordem, Branco pode fugir com as pedras de corte.},
]

\problemAnswerDiagram[%
  sgf folder               = liberdades/espremer,
  sgf filename             = espremer.8,
  problem text             = {A chave para a resposta é colocar máxima pressão no adversário.},
  answer text one          = {\emph{Correto.} Esta técnica é chamada de ``escada frouxa'', ou ``loose ladder'' em inglês. É uma escada que não funciona, mas Preto tem liberades internas suficientes para ganhar a corrida de liberdades depois de ter espremido as pedras brancas.},
  answer text two          = {\emph{Incorreto.} Uma rede parece funcionar, mas Preto não reduziu as liberdades brancas como deveria.},
  answer diagram clip vert = 10,
]

\problemAnswerDiagram[%
  sgf folder      = liberdades/espremer,
  sgf filename    = espremer.9,
  problem text    = {Branco acaba de cortar em 1 com o intuito de capturar as pedras marcadas. Existe uma maneira de resgatá-las?},
  answer text one = {\emph{Correto.} Espremendo com uma escada frouxa, conseguimos não somente resgatar nossas pedras mas capturar as pedras de corte.},
  answer text two = {\emph{Incorreto.} As pedras perdidas não representam somente pontos, mas também espaço de olho para o grupo preto.},
]

\problemAnswerDiagram[%
  sgf folder               = liberdades/espremer,
  sgf filename             = espremer.10,
  problem text             = {Preto está jogando em altíssimo nível. Você consegue seguir com a ideia? Utilizamos ``espremer'' mesmo não capturando nada, é muito mais uma maneira de danificar a forma adversária.},
  answer text one          = {\emph{Correto.} Não há como capturar as pedras brancas diretamente. No entanto, Branco provavelmente preferiria ter sido capturado ao invés de ter que carregar o peso dessas pedras que cercam território nenhum.},
  answer text two          = {\emph{Incorreto.} Desta maneira, no mínimo, Branco conseguirá capturar as pedras marcadas. E ainda há a questão de se Preto possui a escada de 9 ou não.},
  answer diagram clip vert = 9,
]

\problemAnswerDiagram[%
  sgf folder      = liberdades/espremer,
  sgf filename    = espremer.11,
  problem text    = {Preto gostaria de escapar para o centro, mas há ainda o corte em A com que se preocupar.},
  answer text one = {\emph{Correto.} Preto espreme Branco antes de finalmente consertar o corte.},
  answer text two = {\emph{Incorreto.} Simplesmente corrigir o corte nem sequer garante vida, pois há ainda o ricochete em A.},
]

\problemAnswerDiagram[%
  sgf folder               = liberdades/espremer,
  sgf filename             = espremer.12,
  problem text             = {Como diria Honinbo Shusai --- o último dos Honinbos, que dominaram o Go japonês entre os séculos \rom{16} e \rom{20}, e revolucionaram especialmente a abertura ---, Go é um jogo de cortes.},
  answer text one          = {\emph{Correto.} Cortando, conseguimos criar problemas de liberdades nos grupos. Inseong Hwang 8d, um dos amadores mais fortes no ocidente, enfatiza que é como quebrar um bloco sólido em múltiplos detritos.},
  answer text two          = {\emph{Incorreto.} Ao conectar em 2, Branco mantém seus grupos unidos, o que anula os problemas de liberdades necessários para que Preto solucione seus problemas.},
  answer diagram clip vert = 10,
]

\problemAnswerDiagram[%
  sgf folder      = liberdades/espremer,
  sgf filename    = espremer.13,
  problem text    = {É inegável a qualidade de um jogador que vê uma sequência destas com antecedência.},
  answer text one = {\emph{Correto.} Com 1 em 6, Preto utiliza de uma escada frouxa e uma escassez de liberdades para capturar as pedras brancas.},
  answer text two = {\emph{Incorreto.} Isto parece funcionar, mas Preto possui fraquezas remanescentes no exterior. Se Branco não possuir a escada de 16, pode optar por simplesmente capturar no topo jogando 14 em Preto 15.},
]

\problemAnswerDiagram[%
  sgf folder               = liberdades/espremer,
  sgf filename             = espremer.14,
  problem text             = {A forma do canto tem sido cada vez mais comum na era pós-IA, mas é necessária boa técnica para justificá-la.},
  answer text one          = {\emph{Correto.} Preto 1 garante problemas de liberdades para Branco, não há muito para onde fugir, se Branco 2 em 5, Preto 2, e Branco colidirá com a parede preta à esquerda. Esta é uma das versatilidades da forma marcada, que os franceses nomearam de ``escorpião''.},
  answer text two          = {\emph{Incorreto.} Começar por Preto 1 aqui pode parecer resultar na mesma sequência, mas Branco poderá escapar para o centro. Se Preto 3 em 4, Branco 4 em 3, e Preto perderá a corrida de liberdades com Preto 5 em A e Branco 6 em B.},
  answer diagram clip vert = 10,
]

\problemAnswerDiagram[%
  sgf folder               = liberdades/espremer,
  sgf filename             = espremer.15,
  problem text             = {Este exercício é a essência de como destruir a forma do oponente.},
  answer text one          = {\emph{Correto.} Preto 1 é contra-intuitivo para quem nunca viu algo assim, pois é impossível salvar as pedras pretas originais. Mas isso pouco importa, pois a questão é a forma final branca, que é reduzida a ``bolinhos chineses''; ou ``cachos de uvas'', como diriam os coreanos.},
  answer text two          = {\emph{Incorreto.} Preto executa uma ideia parecida com a resposta, mas com menos força e danificando menos a forma branca, que agora possui, no mínimo, um o-lho.},
  answer diagram clip vert = 10,
]