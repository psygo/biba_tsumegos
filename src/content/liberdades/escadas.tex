\chapter{Escadas}

\emptypage

\problemAnswerDiagram
  {liberdades/escadas}
  {escadas.1}
  {Será que há mais de uma maneira de se capturar a pedra branca?}
  {\emph{Correto.} Este padrão é conhecido como \emph{shicho} em japonês, ou ``escada'' em português. Ao chegar na borda do tabuleiro, Branco não conseguirá mais estender suas liberdades.}
  {\emph{Incorreto.} Se a configuração do problema estivesse deslocada uma linha para a direita, jogar 1 seria uma opção, mas, nesta situação, jogar 1 é um desastre.}

\problemAnswerDiagram
  {liberdades/escadas}
  {escadas.2}
  {Agora, Branco possui uma pedra no caminho da escada. Isso muda alguma coisa?}
  {\emph{Correto.} A pedra branca ajudará a estender as liberdades das pedras sob a escada, impossibilitando sua conclusão. Caso a escada não funcione, jogar o padrão é um dano indireto pois Preto fica com múltiplos ataris duplos no exterior.}
  {\emph{Variação.} O melhor que Preto pode fazer em uma situação destas, localmente, é um compromisso ou negociação.}

\problemAnswerDiagram
  {liberdades/escadas}
  {escadas.3}
  {Em relação ao problema anterior, a pedra branca agora está deslocada para o topo. A escada funciona desta vez?}
  {\emph{Correto.} Ainda não funciona. A pedra branca A colocará a pedra B em atari em algum momento.}
  {\emph{Variação.} Novamente, o melhor que Preto pode fazer em uma situação destas, localmente, é um compromisso ou negociação. Não há tanto o que reclamar, pois a pedra branca marcada foi bastante danificada. Branco pode decidir jogar de outra maneira, caso a pedra branca marcada seja importante.}

\problemAnswerDiagram
  {liberdades/escadas}
  {escadas.6}
  {E agora? Em algum momento, houve a troca de 1 por 2. A escada funciona ou não?}
  {\emph{Correto.} Em geral, a pedra Branca que quer quebrar a escada precisa ter 4 liberdades. Caso contrário, ela vai herdar a falta de liberdades do grupo sob a escada. Em problemas de escada, pressupõe-se que não há outras pedras no resto do tabuleiro.}
  {\emph{Incorreto.} Forçar a escada para a outra direção é muito raramente uma boa ideia pois as pedras à direita ficarão ou muito fragilizadas ou diretamente sob atari.}

\problemAnswerDiagram
  {liberdades/escadas}
  {escadas.7}
  {Este é um \emph{joseki} --- isto é, uma sequência ótima para os dois lados. Branco tem basicamente 2 opções a seguir.}
  {\emph{Correto.} A regra geral é capturar a pedra de corte, pois gera excelente forma e força. Preto faz o contra-atari do outro lado e captura o canto. Se Branco tiver a escada externa, pode optar por conectar em 3 e entrar em uma luta complexa.}
  {\emph{Variação.} Cortar deste lado é outra opção, caso Preto esteja mais interessado no exterior, e tenha a escada externa. Estes duplos cortes a partir da forma marcada configuram um padrão bastante recorrente em partidas reais, desde amadores até profissionais.}

\problemAnswerDiagram
  {liberdades/escadas}
  {escadas.4}
  {Este problema é muito avançado. Mas é uma bela referência de como escadas aparecem comumente em josekis e lutas no meio de jogo.}
  {\emph{Correto.} Ao jogar em 1 e estender em 3, Branco cria  \emph{miai} --- isto é, duas opções equivalentes --- de capturar em escada com A, e capturar as pedras de corte no topo com B.}
  {\emph{Incorreto.} A pedra branca A oferece o suporte necessário para que B cancele a escada no exterior!}

\problemAnswerDiagram
  {liberdades/escadas}
  {escadas.5}
  {Mais uma escada bastante avançada. Garanto que há jogadores de nível dan que erram este problema.}
  {\emph{Correto.} Os ataris que Preto possui no exterior ajudam Preto a armar a escada. É quase sempre melhor omitir movimentos desnecessários no Go, por isso, é melhor começar por 3 e somente jogar em 5 se Branco realmente sair com 4.}
  {\emph{Incorreto.} Começar pelo atari de 1 parece gerar a mesma sequência, porém de maneira mais simples. Mas Branco terá o tempo de capturar as pedras pretas que estão em estado crítico!}