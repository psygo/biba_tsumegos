\chapter{Escadas}

\emptypage

\problemAnswerDiagram
  {liberdades/escadas}
  {escadas.1}
  {Será que há mais de uma maneira de se capturar a pedra branca?}
  {\emph{Correto.} Este padrão é conhecido como \emph{shicho} em japonês, ou ``escada'' em português. Ao chegar na borda do tabuleiro, Branco não conseguirá mais estender suas liberdades. Tanto A quanto B funcionarão para completar a captura quando chegamos na lateral.}
  {\emph{Incorreto.} Se a configuração do problema estivesse deslocada uma linha para a direita, jogar 1 seria uma opção, mas, nesta situação, jogar 1 é um desastre.}

\problemAnswerDiagram
  {liberdades/escadas}
  {escadas.2}
  {Agora, Branco possui uma pedra no caminho da escada. Isso muda alguma coisa?}
  {\emph{Correto.} O melhor que Preto pode fazer em uma situação destas, localmente, é um compromisso ou negociação.}
  {\emph{Incorreto.} A pedra branca ajudará a estender as liberdades das pedras sob a escada, impossibilitando sua conclusão. Caso a escada não funcione, jogar o padrão é um dano indireto pois Preto fica com múltiplos ataris duplos no exterior. Pedras adversárias que impossibilitam a escada são chamadas de ``quebra-escadas''.}

\problemAnswerDiagram
  {liberdades/escadas}
  {escadas.3}
  {Em relação ao problema anterior, a pedra branca agora está deslocada para o topo. A escada funciona desta vez?}
  {\emph{Correto.} Novamente, o melhor que Preto pode fazer é um compromisso. Não há tanto o que reclamar, pois a pedra branca marcada foi bastante danificada. Branco pode decidir jogar de outra maneira, caso a pedra branca marcada seja importante.}
  {\emph{Incorreto.} A escada ainda não funciona. A pedra branca A colocará a pedra B em atari em algum momento.}

\problemAnswerDiagram
  {liberdades/escadas}
  {escadas.6}
  {E agora? Em algum momento, houve a troca de 1 por 2. A escada funciona ou não?}
  {\emph{Correto.} Em geral, a pedra Branca que quer quebrar a escada precisa ter 4 liberdades. Afinal, ela vai herdar a falta de liberdades do grupo sob a escada. Em problemas de escada, pressupõe-se que não há outras pedras no resto do tabuleiro.}
  {\emph{Incorreto.} Forçar a escada para a outra direção é muito raramente uma boa ideia pois as pedras à direita ficarão ou muito fragilizadas ou diretamente sob atari.}

\problemAnswerDiagram
  {liberdades/escadas}
  {escadas.9}
  {A troca do problema anterior foi ligeiramente modificada. É escada ou não?}
  {\emph{Correto.} A escada não funciona desta vez. O melhor que Preto pode fazer é um misto de compromisso com dano maior contra o quebra-escada branco no exterior.}
  {\emph{Incorreto.} Aqui está a escada que, efetivamente, não existe.}

\problemAnswerDiagram
  {liberdades/escadas}
  {escadas.8}
  {Parece haver mais de uma resposta.}
  {\emph{Correto.} A presença da pedra A completa a escada. Há alguns que chamam este tipo de forma de mini-escada.}
  {\emph{Variação.} Branco poderia até capturar diretamente as pedras de corte com A ou B, mas 2 cria mais problemas para Preto. Além disso, se Branco A ou B, Preto pode jogar 2 e C para se conectar por fora.}

\problemAnswerDiagram
  {liberdades/escadas}
  {escadas.7}
  {Este é um \emph{joseki} --- isto é, uma sequência ótima para os dois lados. Preto tem basicamente 2 opções a seguir.}
  {\emph{Correto.} A regra geral é capturar a pedra de corte, pois gera excelente forma e força. Preto faz o contra-atari do outro lado e captura o canto.}
  {\emph{Variação.} Cortar deste lado é outra opção, caso Preto esteja mais interessado no exterior, e tenha a escada externa. Estes duplos cortes a partir da forma marcada configuram um padrão bastante recorrente em partidas reais, desde amadores até profissionais. Se Preto não tiver a escada, Branco pode imediatamente sair em A.}

\problemAnswerDiagram
  {liberdades/escadas}
  {escadas.4}
  {Este problema é muito avançado. Mas é uma bela referência de como escadas aparecem comumente em josekis e lutas no meio de jogo.}
  {\emph{Correto.} Ao jogar em 1 e estender em 3, Branco cria  \emph{miai} --- isto é, duas opções equivalentes --- de capturar em escada com A, e capturar as pedras de corte no topo com B.}
  {\emph{Incorreto.} A pedra branca A oferece o suporte necessário para que Branco 6 cancele a escada no exterior! Em geral, no Go, a ordem dos movimentos é extremamente importante. Fazer a sequência supostamente correta fora de ordem é quase sempre errado!}

\problemAnswerDiagram
  {liberdades/escadas}
  {escadas.5}
  {Mais uma escada bastante avançada. Garanto que há jogadores de nível dan que erram este problema.}
  {\emph{Correto.} Os ataris que Preto possui no exterior ajudam-no a armar a escada. É quase sempre melhor omitir movimentos desnecessários no Go, por isso, é melhor começar por 3 e somente jogar em 5 se Branco realmente sair com 4 (o que seria um erro).}
  {\emph{Incorreto.} Começar pelo atari de 1 parece gerar a mesma sequência, porém de maneira mais simples. Mas Branco terá o tempo de capturar as pedras pretas que estão em estado crítico!}