\chapter{Escadas}

\emptypage

\problemAnswerDiagram[%
  sgf folder      = liberdades/escadas,
  sgf filename    = escadas.1,
  problem text    = {Será que há mais de uma maneira de se capturar a pedra branca? (Em problemas de escada, pressupõe-se que não há outras pedras no resto do tabuleiro.)},
  answer text one = {\emph{Correto.} Este padrão é conhecido como \emph{\gls{shicho}} em japonês, ou ``escada'' em português. Ao chegar na borda do tabuleiro, Branco não conseguirá mais estender suas liberdades. Tanto A quanto B funcionarão para completar a captura quando chegamos à segunda linha.},
  answer text two = {\emph{Incorreto.} Se a configuração do problema estivesse deslocada uma linha para a direita, jogar 1 seria uma opção, mas, nesta situação, jogar 1 é um desastre.},
]

\problemAnswerDiagram[%
  sgf folder      = liberdades/escadas,
  sgf filename    = escadas.2,
  problem text    = {Agora, Branco possui uma pedra no caminho da escada. Isso muda alguma coisa?},
  answer text one = {\emph{Correto.} O melhor que Preto pode fazer em uma situação destas, localmente, é um compromisso ou negociação. Pedras adversárias que impossibilitam a escada são chamadas de ``quebra-escadas''. Para Branco, teria sido melhor ser mais ambicioso, atacando em larga escala com A, ao invés de 1.},
  answer text two = {\emph{Incorreto.} A pedra branca A ajudará a estender as liberdades daquelas em escada, impossibilitando sua conclusão. Caso a escada não funcione, jogar o padrão é um dano indireto pois Preto fica com múltiplos ataris duplos no exterior.},
]

\problemAnswerDiagram[%
  sgf folder      = liberdades/escadas,
  sgf filename    = escadas.3,
  problem text    = {Em relação ao problema anterior, a pedra branca agora está deslocada para o topo. A escada funciona desta vez?},
  answer text one = {\emph{Correto.} Novamente, o melhor que Preto pode fazer é um compromisso. Não há tanto o que reclamar, pois a pedra branca marcada foi bastante danificada, atravessada em relação a 2, um tipo de dano que examinaremos mais à frente. Branco pode decidir jogar de outra maneira, caso a pedra branca marcada seja importante.},
  answer text two = {\emph{Variação.} Para evitar o dano à sua pedra exterior, Branco deveria jogar em 2 a-qui na verdade. Sequências de compromisso dependem muito do tabuleiro global, mas uma possibilidade para Preto é esta.},
]

\problemAnswerDiagram[%
  sgf folder      = liberdades/escadas,
  sgf filename    = escadas.6,
  problem text    = {E agora? Em algum momento, houve a troca de 1 por 2. A escada funciona ou não?},
  answer text one = {\emph{Correto.} Em geral, a pedra Branca que quer quebrar a escada precisa ter 4 liberdades. Afinal, ela vai herdar a falta de liberdades do grupo sob a escada.},
  answer text two = {\emph{Incorreto.} Forçar a escada para a outra direção é muito raramente uma boa ideia pois as pedras à direita ficarão ou muito fragilizadas ou diretamente sob atari.},
]

\problemAnswerDiagram[%
  sgf folder      = liberdades/escadas,
  sgf filename    = escadas.9,
  problem text    = {A troca do problema anterior foi ligeiramente modificada. É escada ou não?},
  answer text one = {\emph{Correto.} A escada não funciona desta vez. Como diria Toshiro Kageyama 7p, não há atalhos, é preciso ler a escada! Se Branco insistir em capturar o topo --- teria sido melhor capturar com 4 diretamente ---, podemos danificar o exterior com uma forma leve e flexível.},
  answer text two = {\emph{Variação.} Globalmente, se a partida ainda estiver no início, seria até mesmo melhor somente danificar a pedra do centro e forçar uma resposta branca com 1 ou A. Visto que o topo já foi desvalorizado, podemos jogar em outro lugar.},
]

\problemAnswerDiagram[%
  sgf folder      = liberdades/escadas,
  sgf filename    = escadas.8,
  problem text    = {Parece haver mais de uma resposta.},
  answer text one = {\emph{Correto.} A presença da pedra A completa a escada. Há alguns que chamam este tipo de forma de mini-escada.},
  answer text two = {\emph{Incorreto.} Branco poderia até capturar diretamente as pedras de corte com A ou B, mas 2 cria mais problemas para Preto. Além disso, se Branco A ou B, Preto pode jogar 2 e C para se conectar por fora.},
]

\problemAnswerDiagram[%
  sgf folder      = liberdades/escadas,
  sgf filename    = escadas.7,
  problem text    = {Este é um \emph{\gls{joseki}} --- uma sequência ótima para os dois lados --- um dos poucos que sobreviveu bem a transição para a era pós-IA. Preto possui basicamente 2 opções a seguir.},
  answer text one = {\emph{Correto.} A regra geral é capturar a pedra de corte, pois gera excelente forma e força. Preto faz o contra-atari do outro lado e captura o canto.},
  answer text two = {\emph{Variação.} Cortar deste lado é outra opção, caso Preto se interesse mais pelo exterior, e tenha a escada externa. Estes duplos cortes a partir da forma marcada configuram um padrão bastante recorrente em partidas reais, desde amadores até profissionais. Se Preto não tiver a escada, Branco pode imediatamente sair em A.},
]

\problemAnswerDiagram[%
  sgf folder               = liberdades/escadas,
  sgf filename             = escadas.4,
  problem text             = {Este problema é muito avançado. Mas é uma bela referência de como escadas aparecem comumente em josekis e lutas no meio de jogo.},
  answer text one          = {\emph{Correto.} Ao jogar em 1 e estender em 3, Preto cria  \emph{\gls{miai}} --- isto é, duas opções equivalentes --- de capturar em escada com A, ou capturar as pedras de corte à direita com B.},
  answer text two          = {\emph{Incorreto.} A pedra branca A oferece o suporte necessário para que Branco 6 cancele a escada no exterior! Em geral, no Go, a ordem dos movimentos é extremamente importante. Fazer a sequência supostamente correta fora de ordem é quase sempre errado!},
  answer diagram clip vert = 10,
]

\problemAnswerDiagram[%
  sgf folder      = liberdades/escadas,
  sgf filename    = escadas.5,
  problem text    = {Mais uma escada bastante avançada. Garanto que há jogadores de nível dan que erram este problema.},
  answer text one = {\emph{Correto.} Os ataris que Preto possui no exterior ajudam-no a armar a escada. É quase sempre melhor omitir movimentos desnecessários no Go, por isso, é melhor jogar em 5 se Branco realmente sair com 4 (o que seria um erro).},
  answer text two = {\emph{Incorreto.} Começar pelo a-tari de 1 parece gerar a mesma sequência, porém de maneira mais simples. Mas Branco terá o tempo de capturar as pedras pretas mais críticas!},
]

\problemAnswerDiagram[%
  sgf folder      = liberdades/escadas,
  sgf filename    = escadas.10,
  problem text    = {Preto acaba de encontrar uma jogada magnífica. Como continuar?},
  answer text one = {\emph{Correto.} Esta é uma escada em duas etapas, com uma mudança de direção, tudo graças aos benefícios da lateral do tabuleiro.},
  answer text two = {\emph{Continuação.} Preto muda a direção da escada para o centro.},
]