\chapter{Pedras de Corte}

\emptypage

\problemAnswerDiagram
  {liberdades/pedras_de_corte}
  {pedras_de_corte.1}
  {Salvar as pedras pretas não somente é uma quantia considerável de pontos, mas uma maneira de contra-atacar.}
  {\emph{Correto.} Com 2 e 4, Branco consegue espremer Preto --- em inglês, esta técnica é conhecida como ``squeeze'' ---, mas isso ainda não corrige os cortes de A a E.}
  {\emph{Incorreto.} Preto 2 parece ser uma rede --- veremos esta técnica um pouco mais à frente ---, e uma forma mais bonita e eficiente, mas as pedras pretas não possuem liberdades suficientes para capturar em rede.}

\problemAnswerDiagram
  {liberdades/pedras_de_corte}
  {pedras_de_corte.2}
  {Preto precisa proteger dois lados ao mesmo tempo. É mesmo possível?}
  {\emph{Correto.} As pedras pretas A possuem liberdades o suficiente.}
  {\emph{Incorreto.} Branco não só garante o canto como o mata o grupo inteiro.}

\problemAnswerDiagram
  {liberdades/pedras_de_corte}
  {pedras_de_corte.3}
  {Uma situação bastante confusa, com múltiplos grupos cortados.}
  {\emph{Correto.} Com esta captura, Preto gera liberdades para o grupo A, que estava em estado crítico, e também basicamente captura as pedras marcadas.}
  {\emph{Incorreto.} Primeiramente, jogar em 1 é auto-atari --- quando o próprio jogador se põe em atari, ``self-atari'' em inglês --- nas pedras A. Mas Branco pode ir além e capturar as pedras do topo.}

\problemAnswerDiagram
  {liberdades/pedras_de_corte}
  {pedras_de_corte.4}
  {Preto está quase conseguindo conectar seus grupos.}
  {\emph{Correto.} Branco não consegue fugir.}
  {\emph{Incorreto.} A intenção foi boa, mas a execução falhou.}

\problemAnswerDiagram
  {liberdades/pedras_de_corte}
  {pedras_de_corte.5}
  {Preto pode se fortalecer um pouco mais enquanto ataca Branco.}
  {\emph{Correto.} Preto retira o olho branco, garante mais pontos, e ainda protege um de seus cortes.}
  {\emph{Incorreto.} Mais tarde, ou mesmo em breve, Branco pode garantir um olho e expor os cortes A e B ao mesmo tempo.}

\problemAnswerDiagram
  {liberdades/pedras_de_corte}
  {pedras_de_corte.6}
  {É imprescindível contar as liberdades dos grupos que participam de lutas.}
  {\emph{Correto.} Preto consegue capturar as pedras de corte.}
  {\emph{Incorreto.} Em geral, quando uma sequência não funciona, o melhor é simplesmente não jogá-la. Se jogamos 1 e percebemos que não funciona, é melhor não jogar 3 ou 5. No mínimo, poderíamos utilizar 1, 3 e 5 como ameaças de ko no futuro.}

\clearedpage
\clearedpage