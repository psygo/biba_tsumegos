\chapter{Pedras de Corte}

\emptypage

\problemAnswerDiagram[%
  sgf folder      = liberdades/pedras_de_corte,
  sgf filename    = pedras_de_corte.1,
  problem text    = {Salvar as pedras pretas não somente é uma quantia considerável de pontos, mas uma maneira de contra-atacar.},
  answer text one = {\emph{Correto.} Com 2 e 4, Branco consegue espremer Preto --- em inglês, esta técnica é conhecida como ``squeeze'' ---, mas isso ainda não corrige os cortes de A a E.},
  answer text two = {\emph{Incorreto.} Preto 2 parece ser uma rede --- veremos esta técnica um pouco mais à frente ---, e uma forma mais bonita e eficiente, mas as pedras pretas não possuem liberdades suficientes para capturar em rede.},
]

\problemAnswerDiagram[%
  sgf folder      = liberdades/pedras_de_corte,
  sgf filename    = pedras_de_corte.2,
  problem text    = {Preto precisa proteger dois lados ao mesmo tempo. É mesmo possível?},
  answer text one = {\emph{Correto.} As pedras pretas A possuem liberdades o suficiente.},
  answer text two = {\emph{Incorreto.} Branco não só garante o canto como o mata o grupo inteiro.},
]

\problemAnswerDiagram[%
  sgf folder      = liberdades/pedras_de_corte,
  sgf filename    = pedras_de_corte.3,
  problem text    = {Uma situação bastante confusa, com múltiplos grupos cortados.},
  answer text one = {\emph{Correto.} Com esta captura, Preto gera liberdades para o grupo A, que estava em estado crítico, e também basicamente captura as pedras marcadas.},
  answer text two = {\emph{Incorreto.} Primeiramente, jogar em 1 é auto-atari --- quando o próprio jogador se põe em atari, ``self-atari'' em inglês --- nas pedras A. Mas Branco pode ir além e capturar as pedras do topo.},
]

% \problemAnswerDiagram[%
%   sgf folder      = liberdades/pedras_de_corte,
%   sgf filename    = pedras_de_corte.4,
%   problem text    = {Preto está quase conseguindo conectar seus grupos.},
%   answer text one = {\emph{Correto.} Branco não consegue fugir.},
%   answer text two = {\emph{Incorreto.} A intenção foi boa, mas a execução falhou.},
% ]

\problemAnswerDiagram[%
  sgf folder      = liberdades/pedras_de_corte,
  sgf filename    = pedras_de_corte.5,
  problem text    = {Preto pode se fortalecer um pouco mais enquanto ataca Branco.},
  answer text one = {\emph{Correto.} Preto retira o olho branco, garante mais pontos, e ainda protege um de seus cortes.},
  answer text two = {\emph{Incorreto.} Mais tarde, ou mesmo em breve, Branco pode garantir um olho e expor os cortes A e B ao mesmo tempo.},
]

\problemAnswerDiagram[%
  sgf folder      = liberdades/pedras_de_corte,
  sgf filename    = pedras_de_corte.6,
  problem text    = {É imprescindível contar as liberdades dos grupos que participam de lutas.},
  answer text one = {\emph{Correto.} Preto consegue capturar as pedras de corte.},
  answer text two = {\emph{Incorreto.} Em geral, quando uma sequência não funciona, o melhor é simplesmente não jogá-la. Se jogamos 1 e percebemos que não funciona, é melhor não jogar 3 ou 5. No mínimo, poderíamos utilizar 1, 3 e 5 como ameaças de ko no futuro.},
]

\problemAnswerDiagram[%
  sgf folder      = liberdades/pedras_de_corte,
  sgf filename    = pedras_de_corte.7,
  problem text    = {Este problema é a essência das lutas no Go.},
  answer text one = {\emph{Correto.} Esta resposta é tão óbvia quanto útil. E, com alguma experiência, veremos que ela aparece indiretamente com razoável frequência em lutas.},
  answer text two = {\emph{Incorreto.} O ponto-chave é o mesmo para Branco. Não somente os grupos brancos se conectam, mas os grupos pretos se disconectam, e perdem 1 liberdade cada um.},
]

\problemAnswerDiagram[%
  sgf folder      = liberdades/pedras_de_corte,
  sgf filename    = pedras_de_corte.9,
  problem text    = {Esta é uma variação um pouco rara de um dos josekis mais complexos do Go atual, em que Branco corta em 13, ao invés de continuar empurrando em A. Como proceder?},
  answer text one = {\emph{Correto.} Preto deveria espremer as pedras de corte brancas, minimizando o canto e dificultando a forma branca. Não se preocupe tanto com os movimentos após Preto 1, é uma sequência muito avançada e com outras possibilidades.},
  answer text two = {\emph{Incorreto.} Salvar as pedras coloca mais pressão em algumas das pedras brancas, mas ter sua forma atravessada é um dano grande demais.},
]

\problemAnswerDiagram[%
  sgf folder      = liberdades/pedras_de_corte,
  sgf filename    = pedras_de_corte.8,
  problem text    = {Uma variação do problema de atravessar o adversário.},
  answer text one = {\emph{Correto.} Se Preto tivesse uma pedra em A ao invés de B, jogaria em B de qualquer maneira, ou seja, forma final seria a mesma. O dano contra branco aqui é ligeiramente menor do que se a pedra C estivesse em D, Branco consegue ainda fazer algum uso de C com algo como E no futuro.},
  answer text two = {\emph{Incorreto.} Uma das maneiras com que Branco poderia ter se defendido. Cortar com algo como A não é um erro em absoluto, mas é bastante circunstancial.},
]

\problemAnswerDiagram[%
  sgf folder      = liberdades/pedras_de_corte,
  sgf filename    = pedras_de_corte.10,
  problem text    = {Branco 6 parece ser uma variação agressiva, mas é joseki?},
  answer text one = {\emph{Correto.} Branco A é um pouco demais, os josekis nesta situação são ou B ou C. Continuar com a captura leva a uma forma de keima quebrado ou atravessado para Branco. Além disso, as pedras brancas à direita perderam liberdades e estão bastante pressionadas.},
  answer text two = {\emph{Variação.} Esta é uma das maneiras com que Branco poderia ter revertido para algo perto de parelho.},
]

\problemAnswerDiagram[%
  sgf folder      = liberdades/pedras_de_corte,
  sgf filename    = pedras_de_corte.11,
  problem text    = {Um movimento aqui faz uma diferença colossal.},
  answer text one = {\emph{Correto.} Preto salva uma pedra, divide os e pressiona dois grupos brancos, e ainda reduz o território branco. Note que atravessamos o keima branco.},
  answer text two = {\emph{Variação.} Se Branco puder, cortar em 1 é garantir muito território e segurança, além de servir de ataque indireto contra o grupo Preto, que efetivamente não faz nada de produtivo.},
]

\problemAnswerDiagram[%
  sgf folder      = liberdades/pedras_de_corte,
  sgf filename    = pedras_de_corte.12,
  problem text    = {Preto joga um possível tesuji em 1, e Branco discorda, tentando cortar. Se é um tesuji, o que fazer então?},
  answer text one = {\emph{Correto.} Preto pode conectar com uma das pedras do grupo, A ou B, e, assim, atravessar a forma branca completamente. Preto C é um tesuji muito útil quando queremos fugir de cercos.},
  answer text two = {\emph{Incorreto.} Preto é cortado desta maneira, e terá que manejar ambos os grupos separadamente.},
]

\clearedpage
\clearedpage