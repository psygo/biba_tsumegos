\chapter{Atravessar}

\emptypage

\problemAnswerDiagram[%
  sgf folder      = liberdades/atravessar,
  sgf filename    = atravessar.1,
  problem text    = {Este problema é a essência das lutas no Go.},
  answer text one = {\emph{Correto.} Esta resposta é tão óbvia quanto útil. E, com alguma experiência, veremos que ela aparece indiretamente com bastante frequência em lutas.},
  answer text two = {\emph{Incorreto.} O ponto-chave é o mesmo para Branco. Não somente os grupos brancos se conectam, mas os grupos pretos se desconectam, e perdem 1 liberdade cada um.},
]

\problemAnswerDiagram[%
  sgf folder               = liberdades/atravessar,
  sgf filename             = atravessar.3,
  problem text             = {Esta é uma variação um pouco rara de um dos josekis mais complexos do Go atual, em que Branco corta em 13, ao invés de continuar empurrando em A. Como proceder?},
  answer text one          = {\emph{Correto.} Preto deveria espremer as pedras de corte brancas, minimizando o canto e dificultando a forma branca. Não se preocupe tanto com os movimentos após Preto 1, é uma sequência muito avançada e com outras possibilidades.},
  answer text two          = {\emph{Incorreto.} Salvar as pedras coloca mais pressão em algumas das pedras brancas, mas ter sua forma atravessada é um dano grande demais.},
  answer diagram clip vert = 10 ,
]

\problemAnswerDiagram[%
  sgf folder      = liberdades/atravessar,
  sgf filename    = atravessar.2,
  problem text    = {Uma variação do problema de atravessar o adversário.},
  answer text one = {\emph{Correto.} Se Preto tivesse uma pedra em A ao invés de B, jogaria em B de qualquer maneira, ou seja, a forma final seria a mesma. O dano contra branco aqui é um pouco menor do que se a pedra C estivesse em D, pois Branco consegue ainda fazer algum uso de C com algo como E no futuro.},
  answer text two = {\emph{Incorreto.} Uma das maneiras com que Branco poderia ter se defendido. Tentar cortar com algo como A não é um erro em absoluto, mas é bastante circunstancial.},
]

\problemAnswerDiagram[%
  sgf folder      = liberdades/atravessar,
  sgf filename    = atravessar.4,
  problem text    = {Branco 6 parece ser uma variação agressiva, mas é joseki?},
  answer text one = {\emph{Correto.} Branco A é um pouco demais, os josekis nesta situação teriam sido ou B ou C. Continuar com a captura leva a uma forma de keima quebrado ou atravessado para Branco. Além disso, as pedras brancas à direita perderam liberdades e estão bastante pressionadas.},
  answer text two = {\emph{Variação.} Esta é uma das maneiras com que Branco poderia ter revertido para algo perto de parelho.},
]

\problemAnswerDiagram[%
  sgf folder      = liberdades/atravessar,
  sgf filename    = atravessar.5,
  problem text    = {Um movimento aqui faz uma diferença colossal.},
  answer text one = {\emph{Correto.} Preto salva uma pedra, divide e pressiona os dois grupos brancos, e ainda reduz o território branco. Note que atravessamos o keima branco.},
  answer text two = {\emph{Variação.} Se Branco puder, cortar em 1 é garantir muito território e segurança, além de servir de ataque indireto contra o grupo Preto, o que efetivamente não faz nada de produtivo.},
]

\problemAnswerDiagram[%
  sgf folder      = liberdades/atravessar,
  sgf filename    = atravessar.6,
  problem text    = {Preto joga um possível tesuji em 1, e Branco discorda, tentando cortar. Se é um tesuji, o que fazer então?},
  answer text one = {\emph{Correto.} Preto pode conectar com uma das pedras do grupo, A ou B, e, assim, atravessar a forma branca completamente. Preto C é um tesuji muito útil quando queremos fugir de cercos.},
  answer text two = {\emph{Incorreto.} Preto é cortado desta maneira, e terá que manejar ambos os grupos separadamente.},
]

\problemAnswerDiagram[%
  sgf folder               = liberdades/atravessar,
  sgf filename             = atravessar.7,
  problem text             = {Preto gostaria de sair para o centro idealmente, pois também seria uma maneira de contra-atacar as pedras marcadas.},
  answer text one          = {\emph{Correto.} Branco não consegue cortar diretamente por 3, pois perderia a corrida de liberdades. Jogar em 2 é uma opção. Preto escapa, e Branco continua defendendo os dois lados, não há o que reclamar como Branco.},
  answer text two          = {\emph{Variação.} Como em exercícios anteriores, se Branco cortar por 2, Preto atravessa a forma branca. Porém, aqui, isto pode ser uma opção para Branco, e talvez a melhor, dado que já está vivo à direita, e a pedra preta marcada é bastante danificada. Branco 8 em A também é uma opção.},
  answer diagram clip vert = 10,
]

\problemAnswerDiagram[%
  sgf folder               = liberdades/atravessar,
  sgf filename             = atravessar.8,
  problem text             = {Branco acaba de cortar em 1, como responder?},
  answer text one          = {\emph{Correto.} Cortar foi um erro grande, pois, com 1 a 5, atravessamos a forma branca. E a pedra A ainda representa um problema remanescente para Branco.},
  answer text two          = {\emph{Variação.} Na grande maioria dos casos, atravessar o adversário será a resposta, mas tomar o canto desta maneira pode ser uma boa jogada também. Para Branco, teria sido melhor proteger o canto com A, ao invés de ter cortado em B.},
  answer diagram clip vert = 9,
]

\problemAnswerDiagram[%
  sgf folder               = liberdades/atravessar,
  sgf filename             = atravessar.9,
  problem text             = {Branco tenta salvar uma pedra para minimizar o território preto na região, e criar algum problema futuro no canto. Mas é um pouco demais. Como puni-lo?},
  answer text one          = {\emph{Correto.} Se Branco tentar salvar tudo, ficará com nada. Branco morre à direita, e a outra pedra no lado direito é severamente danificada.},
  answer text two          = {\emph{Variação.} Branco pode salvar o grupo à direita, mas o custo de um ponnuki no exterior é demasiado grande.},
  answer diagram clip vert = 10,
]

\problemAnswerDiagram[%
  sgf folder      = liberdades/atravessar,
  sgf filename    = atravessar.10,
  problem text    = {O contato em 3 é um joseki circunstancial, para o caso em que Branco queira se estabilizar rapidamente. É uma sequência que só passou a ser viável para Preto após as IAs modernas.},
  answer text one = {\emph{Correto.} Preto joga da maneira mais severa possível. Branco faz o atari de 4 se Preto não conseguir lutar o ko, que é imenso. A pedra preta marcada foi bastante danificada, mas ela pode ser resgatada com A, e o grupo Branco inteiro flutuará, instável.},
  answer text two = {\emph{Incorreto.} Uma resposta preta passiva. Branco agora pode disparar um ko, ou simplesmente estender em 3 caso haja espaço para viver sem ko.},
]

\clearedpage
\clearedpage