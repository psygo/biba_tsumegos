\chapter{Captura na Segunda Linha}

\emptypage

\problemAnswerDiagram
  {liberdades/captura_na_segunda_linha}
  {captura_na_segunda_linha.1}
  {Localmente, uma situação assim deveria ser o equivalente a alarmes soando em uma base militar.}
  {\emph{Correto.} Preto estabiliza seu grupo e fragiliza completamente as pedras brancas.}
  {\emph{Incorreto.} Branco reverte a situação, e é agora Preto quem está desmoronando.}

\problemAnswerDiagram
  {liberdades/captura_na_segunda_linha}
  {captura_na_segunda_linha.2}
  {Um problema similar ao anterior. Você jogaria no mesmo lugar?}
  {\emph{Correto.} Desta vez, capturamos por fora, pois, caso contrário...}
  {\emph{Incorreto.} Ao fazer atari por fora, as pedras pretas ficam em atari.}

\problemAnswerDiagram
  {liberdades/captura_na_segunda_linha}
  {captura_na_segunda_linha.3}
  {Esta é a situação que gera, com frequência, os cenários dos 2 problemas anteriores. Este é um padrão bastante comum no Go.}
  {\emph{Correto.} Preto captura a pedra da segunda linha pois ela não tem como estender suas liberdades. Este padrão é um dos principais motivos por invasões na segunda linha raramente funcionarem.}
  {\emph{Variação.} Branco não tem como salvar suas pedras.}

\problemAnswerDiagram
  {liberdades/captura_na_segunda_linha}
  {captura_na_segunda_linha.5}
  {Branco parece ter se esquecido de algo talvez.}
  {\emph{Correto.} Capturar estas pedras é grande em termos de pontos, e também garante a vida do grupo preto no canto.}
  {\emph{Incorreto.} Não há como cortar nada com Preto 1.}

\problemAnswerDiagram
  {liberdades/captura_na_segunda_linha}
  {captura_na_segunda_linha.4}
  {Aqui talvez seria mais uma captura na terceira linha, mas será necessário utilizar o que vimos anteriormente de todo modo.}
  {\emph{Correto.} Estritamente, Pre-to não precisa capturar, mas é um movimento de extrema valia localmente.}
  {\emph{Variação.} Se Branco tiver a chance, salvar as pedras assim enfraquece muito Preto.}

\clearedpage
\clearedpage
