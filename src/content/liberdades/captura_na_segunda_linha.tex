\chapter{Captura na Segunda Linha}

\emptypage

\problemAnswerDiagram[%
  sgf folder      = liberdades/captura_na_segunda_linha,
  sgf filename    = captura_na_segunda_linha.1,
  problem text    = {Localmente, uma situação assim deveria ser o equivalente a alarmes soando em uma base militar.},
  answer text one = {\emph{Correto.} Preto estabiliza seu grupo e fragiliza completamente as pedras brancas.},
  answer text two = {\emph{Incorreto.} Branco reverte a situação, e é agora Preto quem está desmoronando.},
]

\problemAnswerDiagram[%
  sgf folder      = liberdades/captura_na_segunda_linha,
  sgf filename    = captura_na_segunda_linha.2,
  problem text    = {Um problema similar ao anterior. Você jogaria no mesmo lugar?},
  answer text one = {\emph{Correto.} Desta vez, capturamos por fora, pois, caso contrário...},
  answer text two = {\emph{Incorreto.} Ao fazer atari por fora, as pedras pretas colocam sob atari.},
]

\problemAnswerDiagram[%
  sgf folder      = liberdades/captura_na_segunda_linha,
  sgf filename    = captura_na_segunda_linha.3,
  problem text    = {Esta é a situação que gera, com frequência, os cenários dos 2 problemas anteriores. Este é um padrão bastante comum no Go.},
  answer text one = {\emph{Correto.} Preto captura a pedra da segunda linha pois ela não tem como estender suas liberdades. Este padrão é um dos principais motivos por invasões na segunda linha raramente funcionarem.},
  answer text two = {\emph{Variação.} Branco não tem como salvar suas pedras.},
]

\problemAnswerDiagram[%
  sgf folder      = liberdades/captura_na_segunda_linha,
  sgf filename    = captura_na_segunda_linha.5,
  problem text    = {Branco parece ter se esquecido de algo talvez.},
  answer text one = {\emph{Correto.} Capturar estas pedras é grande em termos de pontos, e também garante a vida do grupo preto no canto.},
  answer text two = {\emph{Incorreto.} Não há como cortar ou capturar nada com Preto 1, além de essencialmente ajudarmos a melhorar a forma branca com uma ``cabeçada'' contra seu muro.},
]

\problemAnswerDiagram[%
  sgf folder      = liberdades/captura_na_segunda_linha,
  sgf filename    = captura_na_segunda_linha.4,
  problem text    = {Esta talvez estaria mais para uma captura na terceira linha, mas será necessário utilizar o que vimos anteriormente de todo modo.},
  answer text one = {\emph{Correto.} Estritamente, Pre-to não precisa capturar, mas é um movimento de extrema valia localmente.},
  answer text two = {\emph{Variação.} Se Branco tiver a chance, salvar suas pedras assim enfraquece muito Preto.},
]

\clearedpage
\clearedpage
