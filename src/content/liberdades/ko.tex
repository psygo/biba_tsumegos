\chapter{Ko}

\emptypage

\problemAnswerDiagram[%
  sgf folder      = liberdades/ko,
  sgf filename    = ko.1,
  problem text    = {Preto pode capturar a pedra marcada? Se sim, Branco pode recapturá-la?},
  answer text one = {\emph{Correto.} Sim, é possível capturar a pedra marcada. E Branco não pode imediatamente recapturá-la pois, pela regra do \emph{\gls{ko}}, o tabuleiro não pode ser repetido. Ko vem originalmente do sânscrito, da Índia, através do budismo, e significa ``ciclo infinito''.},
  answer text two = {\emph{Variação.} Se tiver a chance, Branco pode conectar todas as suas pedras, o que fragiliza a posição preta consideravelmente.},
]

\problemAnswerDiagram[%
  sgf folder      = liberdades/ko,
  sgf filename    = ko.4,
  problem text    = {O que Preto pode fazer de melhor no canto?},
  answer text one = {\emph{Correto.} Preto pode criar um ko para possivelmente capturar boa parte do canto branco, além de desestabilizar o grupo como um todo. Note que Preto precisa sacrificar uma pedra para disparar o ko, ou seja, Branco possui uma vantagem neste ko, já que captura primeiro.},
  answer text two = {\emph{Incorreto.} Branco se protege e está basicamente vivo, e Preto não ganha nada.},
]

\problemAnswerDiagram[%
  sgf folder      = liberdades/ko,
  sgf filename    = ko.7,
  problem text    = {As pedras marcadas estão bastante isoladas. Elas podem viver separadamente. Mas qual seria uma outra alternativa?},
  answer text one = {\emph{Correto.} Preto pode tentar se conectar em ko, em último caso.},
  answer text two = {\emph{Incorreto.} Jogar em 1 pode ajudar a garantir espaço vital na região marcada, mas elimina a possibilidade de ko, o que não é recomendado, a não ser que muito necessário.},
]

\problemAnswerDiagram[%
  sgf folder      = liberdades/ko,
  sgf filename    = ko.6,
  problem text    = {Ainda não discutimos vida ou morte a fundo. Mas Preto precisa de dois olhos para viver.},
  answer text one = {\emph{Correto.} Preto precisa bloquear o olho do canto, e é possível fazê-lo com um ko. Se ele ganhar o ko, conectando em 2, terá um olho em A, e outro em B.},
  answer text two = {\emph{Incorreto.} Se Preto simplesmente conectar, não conseguirá fazer um segundo olho no canto.},
]

\problemAnswerDiagram[%
  sgf folder      = liberdades/ko,
  sgf filename    = ko.10,
  problem text    = {Esta é uma forma que surge com alguma frequência em partidas de jogadores avançados e experientes.},
  answer text one = {\emph{Correto.} Preto perde muito pouco ao sacrificar uma pedra para iniciar um ko. Mas Branco, por outro lado, pode morrer por inteiro.},
  answer text two = {\emph{Incorreto.} É boa técnica reduzir por fora, mas Branco aproveita para corrigir seu problema. O movimento 7 é somente ilustrativo, Preto está gastando uma ameaça de ko ao jogá-lo.},
]

\problemAnswerDiagram[%
  sgf folder      = liberdades/ko,
  sgf filename    = ko.8,
  problem text    = {Uma forma bem comum em partidas.},
  answer text one = {\emph{Correto.} Preto precisa de um ko para conseguir viver. Branco não consegue eliminar a possibilidade de um ko pois A seria auto-atari. E B levaria a um outro ko a partir da captura de A por Preto, um ko que teria um potencial de perda maior para Branco.},
  answer text two = {\emph{Incorreto.} Se Preto conectar, não conseguirá um segundo olho na lateral.},
]

\problemAnswerDiagram[%
  sgf folder      = liberdades/ko,
  sgf filename    = ko.5,
  problem text    = {Há somente uma esperança para viver.},
  answer text one = {\emph{Correto.} Capturando, Preto consegue um ko para ajudá-lo a criar mais um olho no canto.},
  answer text two = {\emph{Incorreto.} Se Branco puder jogar primeiro, 1 mataria o grupo preto inteiro.},
]

\problemAnswerDiagram[%
  sgf folder      = liberdades/ko,
  sgf filename    = ko.9,
  problem text    = {Cuidado para não fazer algo desnecessário.},
  answer text one = {\emph{Correto.} Ao jogar em 1, Preto elimina a necessidade de um ko, evitando complicações.},
  answer text two = {\emph{Incorreto.} Agora, Branco te-rá a oportunidade de disparar o ko.},
  answer diagram clip vert = 9,
]

\problemAnswerDiagram[%
  sgf folder      = liberdades/ko,
  sgf filename    = ko.2,
  problem text    = {Este é um joseki de invasão de canto bastante avançado. Preto tem basicamente 2 opções. Qual é a melhor?},
  answer text one = {\emph{Correto.} Quase sempre, jogar um ko aqui será melhor. Não jogar o ko é um erro que mesmo jogadores com alguns anos de experiência farão.},
  answer text two = {\emph{Incorreto.} Se Branco 2 for necessário, Preto pode viver desta maneira, porém, Branco fica muito forte no exterior, e ainda preserva o \emph{\gls{sente}}, isto é, a iniciativa.},
]

\problemAnswerDiagram[%
  sgf folder      = liberdades/ko,
  sgf filename    = ko.3,
  problem text    = {Esta é uma situação bastante peculiar. Quem está vivo? Quem será capturado?},
  answer text one = {\emph{Correto.} Este é um duplo-ko. Não é possível capturar Preto pois há duas opções de captura em 1 e 5, para uma liberdade extra. Kos duplos são conhecidos como ``fábricas de ameaças'', pois Branco pode sempre efetuar uma captura interna como ameaça de um outro ko no tabuleiro.},
  answer text two = {\emph{Variação.} Se Preto sentir que limpar a situação localmente, perdendo um movimento em outro lugar no tabuleiro mas eliminando as infinitas ameaças de ko, for melhor, pode capturar tudo diretamente, apesar de serem circunstâncias bastante raras.},
]