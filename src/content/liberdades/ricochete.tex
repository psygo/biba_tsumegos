\chapter{Ricochete}

\emptypage

\problemAnswerDiagram[%
  sgf folder      = liberdades/sacrificio,
  sgf filename    = sacrificio.1,
  problem text    = {Já entramos em técnicas de captura bastante avançadas. E este padrão é devastador localmente caso consigamos aplicá-lo.},
  answer text one = {\emph{Correto.} Com um sacrifício, forçamos Branco a se colocar em atari. Esta técnica, em que sacrificamos uma pedra para deixar o adversário imediatamente em atari, é chamada de ``snapback'' em inglês, o que pode ser traduzido talvez como ``ricochete'' em português.},
  answer text two = {\emph{Continuação.} Em seguida, capturamos diretamente. Tipicamente, esta técnica é tão potente que Branco deveria ter jogado em A ao invés de B, caso B tenha sido a última jogada branca antes de A.},
]

\problemAnswerDiagram[%
  sgf folder      = liberdades/sacrificio,
  sgf filename    = sacrificio.2,
  problem text    = {Branco optou por focar em outros lugares no tabuleiro. Localmente, qual seria a melhor continuação para Preto?},
  answer text one = {\emph{Correto.} Preto pode capturar duas pedras brancas através de um aparente sacrifício inicial. Note também que a captura termina com a pedra A possibilitando uma outra futura captura em B ainda.},
  answer text two = {\emph{Variação.} Localmente, po-de-se defender com 1 ou A, mas 1 geralmente leva a mais forma de olho localmente, o que pode ser importante no futuro.},
]

\problemAnswerDiagram[%
  sgf folder      = liberdades/sacrificio,
  sgf filename    = sacrificio.6,
  problem text    = {Branco achou que já tinha bastante forma de olho, o suficiente para pelo menos 2 olhos.},
  answer text one = {\emph{Correto.} Ao capturar com 1, o grupo à direita não possui espaço suficiente para 2 olhos, então o grupo inteiro morre.},
  answer text two = {\emph{Variação.} Se Branco tivesse respondido com uma conexão, o resultado ainda seria um ko, Branco não viveria incondicionalmente.},
]

\problemAnswerDiagram[%
  sgf folder      = liberdades/sacrificio,
  sgf filename    = sacrificio.3,
  problem text    = {Um problema não muito simples, mas, com boa técnica, conseguiremos simplificá-lo drasticamente.},
  answer text one = {\emph{Correto.} Em problemas simétricos, o eixo de simetria é sempre um excelente indício. Neste caso, ao jogar no meio conseguimos aplicar um sacrifício por quaisquer dos lados.},
  answer text two = {\emph{Incorreto.} Preto não consegue conectar em A pois seria suicídio, portanto, os dois olhos já estão configurados.},
]

\problemAnswerDiagram[%
  sgf folder      = liberdades/sacrificio,
  sgf filename    = sacrificio.4,
  problem text    = {Uma situação um pouco mais delicada. Será que a mesma técnica funciona?},
  answer text one = {\emph{Correto.} Aqui, é melhor capturar diretamente, pois Preto possui problemas de liberdades no exterior com a forma marcada.},
  answer text two = {\emph{Subótimo.} Caso Preto foque em otimizar o fim de jogo, Branco consegue movimentos forçados no centro, o que será muito provavelmente mais valioso do que o lucro preto no topo.},
]


\problemAnswerDiagram[%
  sgf folder      = liberdades/sacrificio,
  sgf filename    = sacrificio.5,
  problem text    = {Se você conseguir resolver este problema em uma partida real, com certeza, os espectadores ficarão impressionados.},
  answer text one = {\emph{Correto.} Branco não consegue resgatar suas pedras se Preto cortar em 1.},
  answer text two = {\emph{Incorreto.} Branco toma para si o ponto-chave para a captura.},
]

\problemAnswerDiagram[%
  sgf folder      = liberdades/sacrificio,
  sgf filename    = sacrificio.7,
  problem text    = {O selo clássico de maestria da técnica de ricochete.},
  answer text one = {\emph{Correto.} Um duplo ricochete. Se Branco capturar em A, ainda possuirá somente 1 liberdade e será recapturado. O mesmo vale para o outro grupo, a partir de B.},
  answer text two = {\emph{Incorreto.} Branco conseguirá capturar as pedras de corte. Conectar em A também funcionaria para Branco.},
]

\problemAnswerDiagram[%
  sgf folder               = liberdades/sacrificio,
  sgf filename             = sacrificio.8,
  problem text             = {Dependendo de onde começamos, podemos chegar a um erro.},
  answer text one          = {\emph{Correto.} Branco não consegue salvar o grupo de pedras de corte.},
  answer text two          = {\emph{Incorreto.} Branco não pode salvar as pedras marcadas, mas isso pouco importa, pois as pedras de corte de maior valor estão mais à direita. Não é um fracasso total, mas, certamente, uma sequência subótima.},
  answer diagram clip vert = 10,
]

\problemAnswerDiagram[%
  sgf folder      = liberdades/sacrificio,
  sgf filename    = sacrificio.9,
  problem text    = {Depois de Preto jogar 1, Branco discorda e tenta desconectar os grupos. Como prosseguir?},
  answer text one = {\emph{Correto.} Branco não deveria tentar uma desconexão. A sequência acaba em um ricochete.},
  answer text two = {\emph{Variação.} O máximo que Branco consegue é resgatar a pedra A somente.},
]

\problemAnswerDiagram[%
  sgf folder      = liberdades/sacrificio,
  sgf filename    = sacrificio.10,
  problem text    = {Branco está tentando reduzir o território preto do canto. Qual é a resposta ótima?},
  answer text one = {\emph{Correto.} Bloquear em 1 utiliza o ricochete como proteção, e a pedra marcada ainda fica em atari.},
  answer text two = {\emph{Incorreto.} Uma resposta assim seria subótima na imensa maioria dos casos. Isso só seria uma opção caso Preto estivesse com problemas de liberdades no exterior.},
]