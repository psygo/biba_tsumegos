\chapter{Resgate}

\emptypage

\problemAnswerDiagram[%
  sgf folder      = liberdades/resgate,
  sgf filename    = resgate.1,
  problem text    = {Toda a região está em um estado bastante caótico. Como estabilizá-la?},
  answer text one = {\emph{Correto.} Preto conecta suas pedras e deixa Branco com os pontos de corte A e B a serem gerenciados.},
  answer text two = {\emph{Incorreto.} Se Preto 1, Branco pode capturar duas pedras de corte e atravessar Preto completamente.},
]

\problemAnswerDiagram[%
  sgf folder      = liberdades/resgate,
  sgf filename    = resgate.2,
  problem text    = {Há mais de um grupo preto sob \emph{atari} --- atari é o termo para um grupo que está com somente 1 liberdade, e a inspiração para o console de videogame. Como melhorar a situação?},
  answer text one = {\emph{Correto.} Conectar aqui expõe cortes na forma branca, e mantém o status do grupo como não vivo, além de resgatar pontos.},
  answer text two = {\emph{Incorreto.} Há situações em que é realmente melhor capturar diretamente estas pedras, frequentemente quando há uma escassez de liberdades no exterior preto. Porém, aqui, Branco captura pedras que representam pontos e solução de problemas.},
]

\problemAnswerDiagram[%
  sgf folder      = liberdades/resgate,
  sgf filename    = resgate.3,
  problem text    = {Preto precisa se manter forte.},
  answer text one = {\emph{Correto.} Com 1, Preto captura uma pedra branca, mantém-se conectado, e ainda danifica a pedra A.},
  answer text two = {\emph{Incorreto.} Branco captura tudo no topo, e Preto possui um grupo que não somente não faz pontos como também não está vivo.},
]

\problemAnswerDiagram[%
  sgf folder      = liberdades/resgate,
  sgf filename    = resgate.4,
  problem text    = {O que Preto pode fazer de melhor à direita?},
  answer text one = {\emph{Correto.} Branco não consegue jogar em A diretamente pois seria auto-atari. E outras tentativas também não funcionam.},
  answer text two = {\emph{Incorreto.} Diminuir as liberdades brancas a partir do exterior também é uma boa técnica geralmente, mas não aqui.},
]

\problemAnswerDiagram[%
  sgf folder      = liberdades/resgate,
  sgf filename    = resgate.5,
  problem text    = {Quem jogar primeiro no canto lucrará um bocado.},
  answer text one = {\emph{Correto.} Preto conecta, salva suas pedras e ainda captura boa parte do canto. Branco 2 não agrega muito valor, e é perder uma potencial ameaça de ko.},
  answer text two = {\emph{Incorreto.} Se Preto tentar capturar por fora, Branco limpa seus problemas.},
]

\problemAnswerDiagram[%
  sgf folder      = liberdades/resgate,
  sgf filename    = resgate.6,
  problem text    = {Qual é a sequência ótima aqui?},
  answer text one = {\emph{Correto.} Preto resgata sua pedra, e, ao impossibilitar Branco de conectar em 3, conecta seus grupos.},
  answer text two = {\emph{Incorreto.} O atari de 1 parece gerar benefícios similares, mas Branco captura a pedra mais importante e resolve seus problemas.},
]

\problemAnswerDiagram[%
  sgf folder               = liberdades/resgate,
  sgf filename             = resgate.7,
  problem text             = {Branco precisa de algo urgente. Como Preto pode preveni-lo?},
  answer text one          = {\emph{Correto.} Ao conectar, Preto automaticamente captura as pedras brancas. Se Branco perceber que não há mais nada a fazer, é melhor não jogar 2 ou 4.},
  answer text two          = {\emph{Incorreto.} Cercar por fora não funcionará.},
  answer diagram clip vert = 10,
]

\problemAnswerDiagram[%
  sgf folder      = liberdades/resgate,
  sgf filename    = resgate.8,
  problem text    = {Uma ilustração das anomalias do canto.},
  answer text one = {\emph{Correto.} Preto se conecta, e Branco não pode diretamente jogar nem em A, nem em B. Este tipo de impossibilidade de aproximação direta acontece com frequência no canto.},
  answer text two = {\emph{Incorreto.} Branco toma a resposta para si. O bom movimento do seu adversário é frequentemente a sua melhor jogada, ou, pelo menos, uma boa opção.},
]

\problemAnswerDiagram[%
  sgf folder      = liberdades/resgate,
  sgf filename    = resgate.9,
  problem text    = {Para resolver problemas mútuos de liberdades, o princípio-chave é pressão.},
  answer text one = {\emph{Correto.} Ao se conectar, Preto torna efetivamente impossível uma conexão no canto e previne o atari de A.},
  answer text two = {\emph{Variação.} Preto pode também capturar o ko no canto, mas, no final, vai precisar voltar e jogar em A. Branco pode seguir a partida normalmente, jogando um movimento qualquer no tabuleiro, e voltar para capturar o ko de volta em B.},
]

\problemAnswerDiagram[%
  sgf folder               = liberdades/resgate,
  sgf filename             = resgate.10,
  problem text             = {Se Preto conseguir resgatar algo, o que acontecerá?},
  answer text one          = {\emph{Correto.} Preto resgata a pedra de corte, e Branco não tem mais escapatória. Agora, inclusive, o grupo branco do topo precisará viver imediatamente.},
  answer text two          = {\emph{Incorreto.} Branco captura a pedra que importa, e Preto não possui espaço suficiente para viver à direita.},
  answer diagram clip vert = 9,
]

\clearedpage
\clearedpage