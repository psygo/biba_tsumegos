\chapter{Resgate}

\emptypage

\problemAnswerDiagram[%
  sgf folder      = liberdades/resgate,
  sgf filename    = resgate.1,
  problem text    = {Toda a região está em um estado bastante caótico. Como organizá-la?},
  answer text one = {\emph{Correto.} Preto conecta suas pedras e deixa Branco com os pontos de corte A e B a serem gerenciados.},
  answer text two = {\emph{Incorreto.} Branco pode capturar duas pedras de corte e atravessar Preto completamente.},
]

\problemAnswerDiagram[%
  sgf folder      = liberdades/resgate,
  sgf filename    = resgate.2,
  problem text    = {Há mais de um grupo preto sob atari. Como melhorar a situação?},
  answer text one = {\emph{Correto.} Conectar aqui expõe cortes na forma branca, e mantém o status do grupo como não vivo, além de resgatar pontos.},
  answer text two = {\emph{Incorreto.} Há situações em que é realmente melhor capturar diretamente estas pedras, porém, aqui, Branco captura pedras de corte que representam pontos e problemas.},
]

\problemAnswerDiagram[%
  sgf folder      = liberdades/resgate,
  sgf filename    = resgate.3,
  problem text    = {Preto precisa se manter forte.},
  answer text one = {\emph{Correto.} Com 1, Preto captura uma pedra branca, mantém-se conectado, e ainda danifica a pedra A.},
  answer text two = {\emph{Incorreto.} Branco captura tudo no topo, e Preto possui um grupo que não faz pontos e que também não está vivo.},
]

\clearedpage
\clearedpage