\chapter{Captura Direta}

\emptypage

\problemAnswerDiagram[%
  sgf folder      = liberdades/captura,
  sgf filename    = captura.1,
  problem text    = {É possível capturar a pedra branca? (Em livros de tsumegos, costuma-se padronizar sempre Preto a jogar.)},
  answer text one = {\emph{Correto.} Sim, é possível capturar a pedra branca diretamente, já que ela só possui 1 liberdade. (Note que a pedra branca será retirada do tabuleiro.)},
  answer text two = {\emph{Variação.} Mesmo se Branco tentar resgatar ou contra-capturar a pedra preta, sua pedra do canto ainda possui somente 1 liberdade e pode ser capturada. (Isso não quer dizer que Branco 1 seja uma jogada ruim em absoluto.)},
]

\problemAnswerDiagram[%
  sgf folder      = liberdades/captura,
  sgf filename    = captura.2,
  problem text    = {E se a pedra branca estiver na lateral ao invés do canto, ainda é possível capturá-la?},
  answer text one = {\emph{Correto.} Branco não deveria jogar 2, pois não há para onde fugir. Suas pedras terão sempre 1 liberdade, e vão colidir com o canto superior esquerdo no final.},
  answer text two = {\emph{Variação.} Preto pode sempre fazer \emph{tenuki}, isto é, ignorar movimentos locais e jogar em outro lugar, mas então será Branco quem poderá capturar a pedra preta.},
]

\problemAnswerDiagram[%
  sgf folder      = liberdades/captura,
  sgf filename    = captura.3,
  problem text    = {Agora, a pedra branca está no centro. Ela ainda pode ser capturada?},
  answer text one = {\emph{Correto.} A captura da pedra branca configura uma forma chamada \emph{ponnuki}, que é o número mínimo de pedras para se capturar uma pedra adversária (sem ser no canto ou na lateral), quando ela não está nem no canto e nem na borda.},
  answer text two = {\emph{Variação.} Ao escapar com esta pedra, Branco vai de 1 liberdade para 3, o que é bastante eficiente, isto é, 2 liberdades por movimento; em lutas, liberdades são talvez o bem mais crucial. Desta maneira, Branco também expõe os cortes A e B no exterior preto.},
]

\problemAnswerDiagram[%
  sgf folder      = liberdades/captura,
  sgf filename    = captura.8,
  problem text    = {Há uma pedra branca solitária no topo. O que Preto pode fazer com ela?},
  answer text one = {\emph{Correto.} Preto pode capturá-la e, assim, resolver o problema do corte que existia em A.},
  answer text two = {\emph{Incorreto.} Preto poderia ter capturado a pedra, o que resolveria o corte de uma maneira mais eficiente, sem jogar em seu próprio território. E esta defesa dá chances para que Branco conecte todas as suas pedras mais tarde.},
]

\problemAnswerDiagram[%
  sgf folder      = liberdades/captura,
  sgf filename    = captura.9,
  problem text    = {Capturas no canto são geralmente de extremo valor, pois podem garantir não somente território, mas, também, o espaço vital de grupos.},
  answer text one = {\emph{Correto.} A captura da pedra do canto limpa o a região para Preto, e ainda deixa a pedra A com somente 1 liberdade.},
  answer text two = {\emph{Variação.} Localmente pelo menos, não jogar aqui como Preto ajuda Branco a se estabilizar, enquanto contra-ataca severamente.},
]

\problemAnswerDiagram[%
  sgf folder      = liberdades/captura,
  sgf filename    = captura.7,
  problem text    = {O grupo branco mais ao topo possui 2 pedras. Isso muda algo em relação à possibilidade de captura?},
  answer text one = {\emph{Correto.} Apesar de o grupo branco ter mais pedras, ele ainda possui somente 1 liberdade. Esta forma é conhecida como ``casco de tartaruga'', que é o número mínimo de pedras necessário para se capturar 2 pedras. Também era possível capturar com A.},
  answer text two = {\emph{Variação.} Ao fugir, Branco não somente resgata suas pedras como expõe múltiplas fraquezas no exterior preto.},
]

\problemAnswerDiagram[%
  sgf folder      = liberdades/captura,
  sgf filename    = captura.10,
  problem text    = {Ambos os lados estão, localmente, em uma situação crítica.},
  answer text one = {\emph{Correto.} Preto captura dois pontos, praticamente garante o canto, e a pedra marcada é indiretamente engolida.},
  answer text two = {\emph{Variação.} Branco reverte pa-ra uma captura para si, e o grupo preto agora está flutuando e instável.},
]

\problemAnswerDiagram[%
  sgf folder      = liberdades/captura,
  sgf filename    = captura.23,
  problem text    = {Há mais de 1 grupo preto prestes a ser capturado.},
  answer text one = {\emph{Correto.} Preto captura as principais pedras de corte e resolve todos os seus problemas.},
  answer text two = {\emph{Incorreto.} Resgatar pedras diretamente é muitas vezes o melhor movimento. Mas por que resgatar se podemos corrigir capturando diretamente?},
]

\problemAnswerDiagram[%
  sgf folder      = liberdades/captura,
  sgf filename    = captura.24,
  problem text    = {Preto já está basicamente seguro, mas uma captura nesta região não somente representaria muitos pontos, como também um ataque.},
  answer text one = {\emph{Correto.} Preto captura tudo basicamente.},
  answer text two = {\emph{Incorreto.} Se Branco tiver a chance, com 1, ele basicamente estará vivo, pois a pedra A é engolida automaticamente. Além disso, o corte em B agora está disponível.},
]

\problemAnswerDiagram[%
  sgf folder      = liberdades/captura,
  sgf filename    = captura.4,
  problem text    = {É possível capturar a pedra branca A?},
  answer text one = {\emph{Correto.} Sim, pois não há para onde fugir. A pedra branca colidirá contra o canto, onde não conseguirá mais estender suas liberdades. Como Branco não consegue resgatar as pedras, é melhor diminuir o prejuízo com A, ao invés de 2.},
  answer text two = {\emph{Variação.} Se Branco quiser e puder, pode consertar o problema com 1 ou A.},
]

\clearedpage
\clearedpage