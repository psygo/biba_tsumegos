\chapter{Captura}

\emptypage

\problemAnswerDiagram
  {liberdades/captura}
  {captura.1}
  {É possível capturar a pedra branca? (Em livros de tsumegos, costuma-se padronizar sempre Preto a jogar.)}
  {\emph{Correto.} Sim, é possível capturar a pedra branca diretamente, já que ela só possui 1 liberdade. (Note que a pedra branca será retirada do tabuleiro.)}
  {\emph{Variação.} Mesmo se Branco tentar resgatar ou contra-capturar a pedra preta, sua pedra do canto ainda possui somente 1 liberdade e pode ser capturada.}
  
\problemAnswerDiagram
  {liberdades/captura}
  {captura.2}
  {E se a pedra branca estiver no lado ao invés do canto, ainda é possível capturá-la?}
  {\emph{Correto.} Branco não deveria jogar 2, pois não há para onde fugir. Suas pedras terão sempre 1 liberdade, e vão colidir com o canto esquerdo no final.}
  {\emph{Variação.} Preto pode sempre fazer \emph{tenuki}, isto é, ignorar movimentos locais e jogar em outro lugar, mas aí é Branco quem poderá capturar a pedra preta.}

\problemAnswerDiagram
  {liberdades/captura}
  {captura.3}
  {Agora a pedra branca está no centro. Ela ainda pode ser capturada?}
  {\emph{Correto.} A captura da pedra branca configura uma forma chamada \emph{ponnuki}, que é o número mínimo de pedras para se capturar uma pedra adversária, quando ela não está nem no canto e nem na borda. (Note que a pedra branca será retirada do tabuleiro.)}
  {\emph{Variação.} Ao escapar com esta pedra, Branco vai de 1 liberdade para 3, o que é bastante eficiente, isto é, 2 liberdades por movimento; em lutas, liberdades são talvez o bem mais crucial. Desta maneira, Branco também expõe os cortes A e B no exterior preto.}

\problemAnswerDiagram
  {liberdades/captura}
  {captura.8}
  {Há uma pedra branca solitária no topo. O que Preto pode fazer com ela?}
  {\emph{Correto.} Preto pode capturá-la e, assim, resolver o problem do corte que existia em A.}
  {\emph{Incorreto.} Preto poderia ter capturado a pedra, o que resolveria o corte de uma maneira mais eficiente, sem jogar em seu próprio território. E esta defesa dá chances para que Branco conecte todas as suas pedras mais tarde.}

\problemAnswerDiagram
  {liberdades/captura}
  {captura.9}
  {Capturas no canto são geralmente de extremo valor pois podem garantir não somente território, mas, também, o espaço vital de grupos, como veremos mais tarde nos exercícios de vida ou morte deste livro.}
  {\emph{Correto.} A captura da pedra do canto limpa o canto para Preto, e ainda deixa a pedra A com somente 1 liberdade.}
  {\emph{Incorreto.} Localmente pelo menos, não jogar aqui como Preto ajuda Branco a se estabilizar localmente, enquanto contra-ataca severamente.}

\problemAnswerDiagram
  {liberdades/captura}
  {captura.7}
  {O grupo branco mais ao topo possui 2 pedras. Isso muda algo em relação à possibilidade de captura?}
  {\emph{Correto.} Apesar de o grupo branco ter mais pedras, ele ainda possui somente 1 liberdade. Esta forma é conhecida como ``casco de tartaruga'', que é o número mínimo de pedras necessário para se capturar 2 pedras. Também era possível capturar com A. (Note que ambas as pedras brancas serão retiradas do tabuleiro.)}
  {\emph{Variação.} Ao fugir, Branco não somente resgata suas pedras como expõe múltiplas fraquezas no exterior preto.}

\problemAnswerDiagram
  {liberdades/captura}
  {captura.10}
  {Ambos os lados estão, localmente, em uma situação crítica.}
  {\emph{Correto.} Preto captura dois pontos, praticamente garante o canto, e a pedra marcada é indiretamente engolida.}
  {\emph{Variação.} Branco reverte para uma captura para si, e o grupo preto agora está flutuando e instável.}

\problemAnswerDiagram
  {liberdades/captura}
  {captura.23}
  {Há mais de 1 grupo preto prestes a ser capturado.}
  {\emph{Correto.} Preto captura as principais pedras de corte e resolve todos os seus problemas.}
  {\emph{Incorreto.} Resgatar pedras diretamente é muitas vezes o melhor momento. Mas por que resgatar se podemos corrigir capturando diretamente?}

\problemAnswerDiagram
  {liberdades/captura}
  {captura.24}
  {Preto já está basicamente seguro, mas uma captura nesta região não somente seriam pontos, mas, também, um possível ataque contra Branco.}
  {\emph{Correto.} Preto captura tudo basicamente.}
  {\emph{Incorreto.} Se Branco tiver a chance, com 1, ele basicamente estará vivo, pois a pedra A é engolida automaticamente. Além disso, o corte em B agora é possível.}

\clearedpage
\clearedpage

% \chapter{Conectar e Morrer}

% \emptypage

% \problemAnswerDiagram
%   {liberdades/captura}
%   {captura.17}
%   {Branco parece seguro, mas há um problema gigante em sua forma.}
%   {\emph{Correto.} Se Branco conectar em 2, é capturado. Esta técnica é conhecida como ``connect and die'', ou ``conectar e morrer'', apesar de que muitos diriam que é mais um problema de falta de liberdades.}
%   {\emph{Incorreto.} Em algum momento, Branco pode jogar em 1 para viver.}

% \problemAnswerDiagram
%   {liberdades/captura}
%   {captura.18}
%   {Preto precisa de um milagre, e rápido.}
%   {\emph{Correto.} Branco não consegue conectar em 3, pois a pedra preta em A garante um auto-atari. Este é um exemplo muito mais emblemático de ``conectar e morrer''.}
%   {\emph{Incorreto.} Capturar em 2 corrige os problemas para Branco.}

% \problemAnswerDiagram
%   {liberdades/captura}
%   {captura.16}
%   {Mais uma situação extremamente suspeita em termos de liberdades.}
%   {\emph{Correto.} Branco não consegue conectar em 3 diretamente pois seria auto-atari!}
%   {\emph{Incorreto.} Caso Preto não perceba o problema, Branco toma o bom movimento preto para si. Branco A também funcionaria, mas essencialmente tem o mesmo efeito no canto nesta situação, além de retirar uma liberdade das pedras pretas no exterior.}

% \chapter{Corridas de Captura}

% \emptypage
