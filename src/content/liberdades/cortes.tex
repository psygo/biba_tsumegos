\chapter{Cortes}

\emptypage

\problemAnswerDiagram[%
  sgf folder      = liberdades/cortes,
  sgf filename    = cortes.1,
  problem text    = {Salvar as pedras pretas não somente é uma quantia considerável de pontos, mas uma maneira de contra-atacar.},
  answer text one = {\emph{Correto.} Com 2 e 4, Branco consegue espremer Preto --- em inglês, esta técnica é conhecida como ``squeeze'' ---, mas isso ainda não corrige os cortes de A a E.},
  answer text two = {\emph{Incorreto.} Preto 2 parece ser uma rede --- veremos esta técnica um pouco mais à frente ---, e uma forma mais bonita e eficiente, mas as pedras pretas não possuem liberdades suficientes para capturar em rede.},
]

\problemAnswerDiagram[%
  sgf folder      = liberdades/cortes,
  sgf filename    = cortes.2,
  problem text    = {Preto precisa proteger dois lados ao mesmo tempo. É mesmo possível?},
  answer text one = {\emph{Correto.} As pedras pretas A possuem liberdades o suficiente.},
  answer text two = {\emph{Incorreto.} Branco não só garante o canto como o mata o grupo inteiro.},
]

\problemAnswerDiagram[%
  sgf folder      = liberdades/cortes,
  sgf filename    = cortes.3,
  problem text    = {Uma situação bastante confusa, com múltiplos grupos cortados.},
  answer text one = {\emph{Correto.} Com esta captura, Preto gera liberdades para o grupo A, que estava em estado crítico, e também basicamente captura as pedras marcadas.},
  answer text two = {\emph{Incorreto.} Primeiramente, jogar em 1 é auto-atari --- quando o próprio jogador se põe em atari, ``self-atari'' em inglês --- nas pedras A. Mas Branco pode ir além e capturar as pedras do topo.},
]

\problemAnswerDiagram[%
  sgf folder      = liberdades/cortes,
  sgf filename    = cortes.4,
  problem text    = {Como Preto pode salvar as 3 pedras mais ao topo?},
  answer text one = {\emph{Correto.} Ao pular, conectamos os dois grupos.},
  answer text two = {\emph{Incorreto.} Este outro pulo parece similar, mas efetivamente falhará em conectar os grupos.},
]

\problemAnswerDiagram[%
  sgf folder      = liberdades/cortes,
  sgf filename    = cortes.5,
  problem text    = {Há mais de uma maneira de salvar as pedras pretas.},
  answer text one = {\emph{Correto.} Conectamos e capturamos 3 pedras brancas, um resultado excelente.},
  answer text two = {\emph{Subótimo.} Por que conectar simplesmente se podemos conectar capturando o adversário?},
]

\problemAnswerDiagram[%
  sgf folder      = liberdades/cortes,
  sgf filename    = cortes.6,
  problem text    = {Uma técnica excelente para se desconectar pedras.},
  answer text one = {\emph{Correto.} Em japonês, este movimento em que se joga entre as pedras adversárias se chama \emph{warikomi}, ou, em inglês, ``wedge''. O objetivo é criar dois pontos de corte, A e B, de uma vez só. Note, porém, que é preciso muita força externa para que isso funcione.},
  answer text two = {\emph{Incorreto.} Às vezes, isto funciona de maneira parecida com o warikomi, mas, como Branco já possui a boca-de-tigre preparada à esquerda, não conseguiremos cortar depois da conexão em Branco 2.},
]

\problemAnswerDiagram[%
  sgf folder      = liberdades/cortes,
  sgf filename    = cortes.7,
  problem text    = {Conectamos o nosso ponto fraco? Como? Ou será que há algo que podemos capturar à direita?},
  answer text one = {\emph{Correto.} A conexão com boca-de-tigre, além de conectar, retira uma liberdade da pedra preta marcada.},
  answer text two = {\emph{Subótimo.} É correto conectar, mas esta conexão não tem pressão no adversário.},
]

\problemAnswerDiagram[%
  sgf folder      = liberdades/cortes,
  sgf filename    = cortes.8,
  problem text    = {A resposta é devastadora.},
  answer text one = {\emph{Correto.} Preto basicamente joga um warikomi. Ao capturar as pedras à direita, não será mais necessário viver separadamente com o grupo do topo.},
  answer text two = {\emph{Incorreto.} Por que continuar fugindo ou batalhando contra o grupo branco se podemos viver capturando? Agora, Branco terá a opção de se conectar com A, B ou algo equivalente.},
]

\problemAnswerDiagram[%
  sgf folder      = liberdades/cortes,
  sgf filename    = cortes.9,
  problem text    = {Jogar em 1 foi um tesuji. Você consegue concluir a sequência?},
  answer text one = {\emph{Correto.} Preto se conecta por cima, e Branco nem sequer está vivo no topo ainda.},
  answer text two = {\emph{Incorreto.} Preto não possui espaço para viver no canto.},
]

\problemAnswerDiagram[%
  sgf folder      = liberdades/cortes,
  sgf filename    = cortes.10,
  problem text    = {Dois grupos pretos em estado crítico. Qual é a melhor maneira de melhorar a situação?},
  answer text one = {\emph{Correto.} Preto pode se conectar, resgatando o canto e garantindo pelo menos 1 olho.},
  answer text two = {\emph{Incorreto.} Preto não possui espaço suficiente no canto para viver.},
]

\clearedpage
\clearedpage