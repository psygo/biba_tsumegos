\chapter{Suicídio}

\emptypage

\problemAnswerDiagram[%
  sgf folder      = liberdades/suicidio,
  sgf filename    = suicidio.1,
  problem text    = {Preto pode conectar suas pedras?},
  answer text one = {\emph{Correto.} Conectar as pedras zeraria as liberdades de todas as pedras do grupo, o que as automaticamente capturaria, ou seja, seria suicídio, uma jogada inválida.},
  answer text two = {\emph{Variação.} Branco tem sempre a opção de capturar ambas as pedras.},
]

\problemAnswerDiagram[%
  sgf folder      = liberdades/suicidio,
  sgf filename    = suicidio.2,
  problem text    = {Preto pode capturar as pedras brancas?},
  answer text one = {\emph{Correto.} Note que não é suicídio jogar em 1, pois a regra da captura possui precedência. Isto é, primeiro aplicamos a regra da captura, se possível, e, só depois, examinamos se é suicídio. E segue que, se algo for capturado, haverá pelo menos uma liberdade.},
  answer text two = {\emph{Variação.} Mais tarde, se Branco conseguir pedras no exterior, quem pode ser capturado é o Preto! Antes de 1, Preto poderia finalmente capturar o canto, e, dessa maneira, evitar de ser capturado.},
]

\problemAnswerDiagram[%
  sgf folder      = liberdades/suicidio,
  sgf filename    = suicidio.4,
  problem text    = {Preto pode capturar o grupo branco?},
  answer text one = {\emph{Correto.} Não é possível capturar o grupo branco pois tanto A quanto B excluem um ao outro por suicídio. Isto é, Preto precisa jogar em A e B para capturar, mas possui somente 1 movimento por turno, e, separadamente, ambos são suicídios.},
  answer text two = {\emph{Variação.} Aqui, Branco possui somente 1 espaço cercado, ou 1 olho. Branco precisa de 2 espaços dissociados pelo menos, ou 2 olhos, para conseguir viver.},
]

\problemAnswerDiagram[%
  sgf folder      = liberdades/suicidio,
  sgf filename    = suicidio.5,
  problem text    = {É possível capturar as pedras brancas?},
  answer text one = {\emph{Correto.} Preto pode capturar as pedras brancas o que gera mais de zero liberdades. Nas regras japonesas, os pontos finais são as intersecções cercadas e as capturas, então, ter de capturar em 1, a não ser que necessário, é preencher o seu próprio território, resultando em 1 ponto a menos.},
  answer text two = {\emph{Variação.} Mesmo se Branco capturar, já está morto, pois não consegue dissociar seu espaço interno para criar os necessários 2 olhos.},
]

\problemAnswerDiagram[%
  sgf folder      = liberdades/suicidio,
  sgf filename    = suicidio.3,
  problem text    = {Esta posição é inspirada em um incidente de 2003, no qual um dos maiores jogadores da história, o sul-coreano Cho Hunhyun 9p, cometeu um erro raríssimo. É possível jogar em A?},
  answer text one = {\emph{Correto.} Idealmente, Preto capturaria em 1, o que seria um ko, e que exercitaremos mais à frente. Essa captura ajuda Preto a depois jogar em A para capturar o topo.},
  answer text two = {\emph{Incorreto.} Cho Hunhyun 9p acidentalmente jogou em A na final do título Kiseong, contra Choi Cheolhan 5p, outra lenda do Go, e acabou perdendo, por jogar um movimento ilegal.},
]