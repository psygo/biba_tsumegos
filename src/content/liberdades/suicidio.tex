\chapter{Suicídio}

\emptypage

\problemAnswerDiagram
  {liberdades/suicidio}
  {suicidio.1}
  {Preto pode conectar suas pedras?}
  {\emph{Correto.} Conectar as pedras zeraria as liberdades de todas as pedras do grupo, o que as automaticamente capturaria, ou seja, seria suicídio, uma jogada inválida.}
  {\emph{Variação.} Branco tem sempre a opção de capturar ambas as pedras.}

\problemAnswerDiagram
  {liberdades/suicidio}
  {suicidio.2}
  {Preto pode capturar as pedras brancas?}
  {\emph{Correto.} Note que não é suicídio jogar em 1, pois a regra da captura possui precedência. Isto é, primeiro aplicamos a regra da captura, se possível, e, só depois, examinamos se é suicídio. E segue que, se algo for capturado, haverá mais de uma liberdade.}
  {\emph{Variação.} Mais tarde, se Branco conseguir pedras no exterior, quem pode ser capturado é o Preto! Antes de 1, Preto poderia finalmente capturar o canto, e, dessa maneira, evitar de ser capturado.}

\clearedpage
\clearedpage