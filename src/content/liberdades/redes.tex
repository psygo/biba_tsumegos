\chapter{Redes}

\emptypage

\problemAnswerDiagram[%
  sgf folder      = liberdades/redes,
  sgf filename    = redes.1,
  problem text    = {É possível capturar a pedra branca incondicionalmente? Tente começar com os movimentos mais justos possíveis e, depois, extrapole para jogadas mais distantes.},
  answer text one = {\emph{Correto.} Redes já são um tópico bastante complexo para quem está começando. Com o movimento 1, Preto aprisiona a pedra branca, não há escapatória, e nem como ganhar mais liberdades. Em japonês, o termo para rede é \emph{geta}; em inglês, \emph{net}.},
  answer text two = {\emph{Incorreto.} Preto 1 é uma captura em escada, o que depende do resto do tabuleiro. É preferível algo mais incondicional, como uma rede.},
]

\problemAnswerDiagram[%
  sgf folder      = liberdades/redes,
  sgf filename    = redes.4,
  problem text    = {O grupo preto mais abaixo está um pouco pressionado pelo grupo branco do canto, isso muda a rede a ser aplicada?},
  answer text one = {\emph{Correto.} O grupo preto possui 3 liberdades, o que é o suficiente para se garantir uma rede padrão.},
  answer text two = {\emph{Variação.} Outra rede que funciona, Branco não consegue fugir.},
]

\problemAnswerDiagram[%
  sgf folder      = liberdades/redes,
  sgf filename    = redes.3,
  problem text    = {A parede branca à esquerda parece dificultar as coisas para Preto. Ou não?},
  answer text one = {\emph{Correto.} Não muda nada em relação à rede do problema \ref{redes.1}.},
  answer text two = {\emph{Incorreto.} Não há escada!},
]

\problemAnswerDiagram[%
  sgf folder      = liberdades/redes,
  sgf filename    = redes.7,
  problem text    = {Algo não parece estar certo na forma branca.},
  answer text one = {\emph{Correto.} A rede não funciona! Preto resgata sua pedra, retira o segundo olho branco e, assim, mata o grupo todo.},
  answer text two = {\emph{Incorreto.} Preto presenteia Branco com o necessário segundo olho.},
]

\problemAnswerDiagram[%
  sgf folder      = liberdades/redes,
  sgf filename    = redes.8,
  problem text    = {Preto possui vários cortes e mais de um grupo, uma situação crítica.},
  answer text one = {\emph{Correto.} Preto captura as pedras de corte, conectando seus grupos e corrigindo as falhas de A e B, além de conectar o grupo C e isolar o grupo branco D, revertendo a situação para uma crise branca agora.},
  answer text two = {\emph{Variação.} Se Branco tiver a chance, reforçar com 1, demole todas as esperanças pretas localmente.},
]

\problemAnswerDiagram[%
  sgf folder      = liberdades/redes,
  sgf filename    = redes.6,
  problem text    = {Uma técnica de fim de jogo de grande utilidade. Estas leituras de primeira versus segunda linha podem ser bastante desafiadoras até mesmo para jogadores muito avançados.},
  answer text one = {\emph{Correto.} Jogando em 1, Preto faz com que a borda do tabuleiro sirva de parede em uma rede.},
  answer text two = {\emph{Incorreto.} Este tipo de atari dos dois lados é algo possível em certas sequências, mas, aqui, Preto ficará sob atari por ambos os lados.},
]

\problemAnswerDiagram[%
  sgf folder      = liberdades/redes,
  sgf filename    = redes.11,
  problem text    = {Parece ser uma rede mais complexa, mas é a mesma família problemas dos problemas anteriores.},
  answer text one = {\emph{Correto.} O \emph{tesuji}, isto é, o movimento brilhante, é muito parecido com os anteriores.},
  answer text two = {\emph{Incorreto.} Branco pode fugir.},
]

\problemAnswerDiagram[%
  sgf folder      = liberdades/redes,
  sgf filename    = redes.9,
  problem text    = {Branco adicionou uma troca de 1 por 2 em algum momento. Será que uma rede ainda é possível?},
  answer text one = {\emph{Correto.} Sim. Não é fácil visualizar. Mas o exterior preto permanece forte o suficiente para capturar as duas pedras brancas. Novamente, a sugestão é começar pelas jogadas mais justas possíveis e extrapolar gradualmente até encontrar uma solução.},
  answer text two = {\emph{Incorreto.} Branco pode escapar. No caso, escapar com A é melhor, pois escapar com 1 leva a um ``squeeze'' ou ``espremida'', isto é, Preto sacrifica as pedras para obter uma força bastante considerável no exterior.},
]

\problemAnswerDiagram[%
  sgf folder      = liberdades/redes,
  sgf filename    = redes.10,
  problem text    = {Se Preto encontrar a solução, quais são os problemas que ele resolve?},
  answer text one = {\emph{Correto.} Capturar as pedras de corte é praticamente um milagre em uma situação como esta.},
  answer text two = {\emph{Incorreto.} Preto 1 é um movimento simples e, em geral, é boa técnica atravessar o \emph{keima} --- movimento do cavaleiro, ou o ``L'' do cavalo no xadrez --- de A para B do adversário, mas, desta vez, não funciona.},
]

\problemAnswerDiagram[%
  sgf folder      = liberdades/redes,
  sgf filename    = redes.2,
  problem text    = {Capturar pedras de corte é frequentemente algo de suma importância no Go.},
  answer text one = {\emph{Correto.} Branco não consegue aumentar suas liberdades ou utilizar fraquezas no exterior preto para poder escapar. Note que, se Branco tivesse uma pedra em A por exemplo, poderia usar o atari de B para escapar.},
  answer text two = {\emph{Incorreto.} Seja esta escada ou a de A, ambas são capturas condicionais, o que é quase sempre subótimo.},
]

\problemAnswerDiagram[%
  sgf folder      = liberdades/redes,
  sgf filename    = redes.5,
  problem text    = {Esta já é uma rede bem mais avançada.},
  answer text one = {\emph{Correto.} O importante é que Branco não vai conseguir mais de 2 liberdades com a pedra de corte, o que é menos do que as pedras pretas que o cercam.},
  answer text two = {\emph{Incorreto.} Às vezes, 1 pode ser a rede ótima, mas, neste caso, Branco está muito forte no exterior.},
]

\problemAnswerDiagram[%
  sgf folder      = liberdades/redes,
  sgf filename    = redes.12,
  problem text    = {Jogadores muito avançados vão ou errar ou tomar um tempo considerável aqui.},
  answer text one = {\emph{Correto.} Branco não tem para onde correr.},
  answer text two = {\emph{Incorreto.} Esta rede parece funcionar também, mas Preto está muito frágil no exterior em relação ao grupo A.},
]

\problemAnswerDiagram[%
  sgf folder      = liberdades/redes,
  sgf filename    = redes.14,
  problem text    = {Mais um padrão de rede muito comum.},
  answer text one = {\emph{Correto.} Branco conecta seus grupos e se torna muito mais forte. Branco não deveria tentar escapar, se não há como fazê-lo.},
  answer text two = {\emph{Incorreto.} A rede de Preto 3 parece funcionar, mas Preto não possui muitas liberdades como grupo A.},
]

\problemAnswerDiagram[%
  sgf folder      = liberdades/redes,
  sgf filename    = redes.13,
  problem text    = {Esta é uma sequência relativamente comum e parelha se jogada na hora certa. Entretanto, se Preto encontrar o melhor movimento, seu exterior ficará muito forte.},
  answer text one = {\emph{Correto.} O atari de Branco 4 é um pouco alarmante, mas não há escapatória.},
  answer text two = {\emph{Incorreto.} Em 99\% dos casos, capturar em escada, ou seja, uma captura condicional, não é ideal, isto é, se uma captura incondicional for possível.},
]

\problemAnswerDiagram[%
  sgf folder      = liberdades/redes,
  sgf filename    = redes.15,
  problem text    = {Depois de uma sequência padrão de invasão de canto, há 2 redes possíveis. Qual é a melhor?},
  answer text one = {\emph{Correto.} A diferença é bastante sutil e avançada. Mas a rede com Preto 1 aqui é melhor pois a forma preta é mais forte e flexível. E é ainda possível capturar em escada em último caso também.},
  answer text two = {\emph{Subótimo.} Esta rede também funciona, mas é menos justa.},
]

\problemAnswerDiagram[%
  sgf folder      = liberdades/redes,
  sgf filename    = redes.16,
  problem text    = {Este é um dos poucos tesujis que possui nome próprio.},
  answer text one = {\emph{Correto.} Em inglês, o nome desta rede é ``crane's nest'', ou ``ninho do grou'', expressão que vem da China. Um dos principais aspectos para que esta rede funcione é ter as pedras A e B.},
  answer text two = {\emph{Incorreto.} Se Preto fizer 1, Branco resgata as pedras de corte, e as pedras A morrem.},
]

\problemAnswerDiagram[%
  sgf folder      = liberdades/redes,
  sgf filename    = redes.17,
  problem text    = {Branco está bem forte, mas será que tem o que é preciso para impedir que Preto salve seu grupo cortado?},
  answer text one = {\emph{Correto.} É possível capturar a pedra de corte com uma rede após um primeiro atari. Branco não deveria tentar escapar com a pedra, dado que não funciona, seria melhor jogar em A para salvar as pedras do canto e manter a iniciativa.},
  answer text two = {\emph{Incorreto.} A pedra branca A garante uma vitória na corrida de captura, Branco possui 4 liberdades, 1 a mais que Preto.},
]

\problemAnswerDiagram[%
  sgf folder      = liberdades/redes,
  sgf filename    = redes.18,
  problem text    = {Certamente, esta rede é difícil até mesmo para quem está chegando a níveis dan.},
  answer text one = {\emph{Correto.} Uma outra rede em duas etapas.},
  answer text two = {\emph{Incorreto.} Este é um bom primeiro instinto, mas é preciso checar antes se realmente funciona.},
]

\problemAnswerDiagram[%
  sgf folder               = liberdades/redes,
  sgf filename             = redes.19,
  problem text             = {Um momento crítico na partida. Quando algo assim aparecer, é o momento de utilizar o tempo no relógio.},
  answer text one          = {\emph{Correto.} Após 1, o movimento crucial é estender em 5, Branco não tem como escapar.},
  answer text two          = {\emph{Incorreto.} 1, 3 e 5 parecem funcionar por colocar maior pressão, mas Preto termina com muitas fraquezas, e a pedra marcada faz a diferença.},
  answer diagram clip vert = 10,
]

% From [this game of mine against TakumiGo](https://online-go.com/game/68551561)
\problemAnswerDiagram[%
  sgf folder               = liberdades/redes,
  sgf filename             = redes.20,
  problem text             = {Esta é uma variação do joseki ``Pequena Avalanche'', um dos mais complexos no Go. Se Branco possui a escada, pode estender em 12. Branco 28 a 32 foram erros enormes, como punir?},
  answer text one          = {\emph{Correto.} Há ainda um ko em A, mas é algo inganhável, na imensa maioria das vezes, especialmente no início de partida, pois não haverá nada de tamanho similar no tabuleiro. Esta é uma sequência difícil até mesmo para jogadores no topo da escala amadora, pelo menos a partir de Branco 28.},
  answer text two          = {\emph{Incorreto.} Preto 1 parece criar uma ameaça de ambos os lados, no entanto, há um ko no canto a ser resolvido. Branco nem sequer precisa disparar o ko imediatamente com 4, jogar em A é o suficiente.},
  answer diagram clip vert = 9,
]

\clearedpage
\clearedpage