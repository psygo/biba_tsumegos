\chapter{Redes}

\emptypage

\problemAnswerDiagram[%
  sgf folder      = liberdades/redes,
  sgf filename    = redes.1,
  problem text    = {É possível capturar a pedra branca incondicionalmente com somente um movimento? Tente começar com os movimentos mais justos possíveis e, depois, extrapole para movimentos mais distantes.},
  answer text one = {\emph{Correto.} Redes já são um tópico bastante complexo para quem está começando. Com o movimento 1, Preto aprisiona a pedra branca, não há escapatória, e nem como ganhar mais liberdades. Em japonês, o termo para redes é \emph{geta}.},
  answer text two = {\emph{Correto.} Redes já são um tópico bastante complexo para quem está começando. Com o movimento 1, Preto aprisiona a pedra branca, não há escapatória, e nem como ganhar mais liberdades. Em japonês, o termo para redes é \emph{geta}.},
]

\problemAnswerDiagram[%
  sgf folder      = liberdades/redes,
  sgf filename    = redes.4,
  problem text    = {O grupo preto mais abaixo está um pouco pressionado pelo grupo branco do canto, isso muda a rede a ser aplicada?},
  answer text one = {\emph{Correto.} O grupo preto possui 3 liberdades, o que é o suficiente para se garantir uma rede padrão.},
  answer text two = {\emph{Incorreto.} No geral, 1 é incorreto, pois acaba em uma escada que vai para o lado. Mas escadas indo para o lado podem ser ocasionalmente melhores do que uma rede.},
]

\problemAnswerDiagram[%
  sgf folder      = liberdades/redes,
  sgf filename    = redes.3,
  problem text    = {A parede branca à esquerda parece dificultar as coisas para o Preto. Ou não?},
  answer text one = {\emph{Correto.} Não muda em nada em relação à rede do problema 1.},
  answer text two = {\emph{Incorreto.} Não há escada!},
]

\problemAnswerDiagram[%
  sgf folder      = liberdades/redes,
  sgf filename    = redes.7,
  problem text    = {Algo não parece estar certo na forma branca.},
  answer text one = {\emph{Correto.} A rede não funciona! Preto resgata sua pedra, retira o segundo olho branco e, assim, mata o grupo todo.},
  answer text two = {\emph{Incorreto.} Preto presenteia Branco com o necessário segundo olho.},
]

\problemAnswerDiagram[%
  sgf folder      = liberdades/redes,
  sgf filename    = redes.8,
  problem text    = {Preto possui vários cortes e mais de um grupo, uma situação crítica.},
  answer text one = {\emph{Correto.} Preto captura as pedras de corte, conectando seus grupos e corrigindo as falhas de A, B e C, o que reverte a situação para uma crise branca agora.},
  answer text two = {\emph{Incorreto.} Se Branco tiver a chance, reforçar a com 1, demole todas as esperanças pretas localmente.},
]

\problemAnswerDiagram[%
  sgf folder      = liberdades/redes,
  sgf filename    = redes.11,
  problem text    = {Parece ser uma rede mais complexa, mas é a mesma família problemas dos problemas anteriores.},
  answer text one = {\emph{Correto.} O \emph{tesuji}, isto é, o movimento brilhante, é muito parecido com os anteriores.},
  answer text two = {\emph{Incorreto.} Branco pode fugir.},
]

\problemAnswerDiagram[%
  sgf folder      = liberdades/redes,
  sgf filename    = redes.9,
  problem text    = {Branco adicionou uma troca de 1 por 2 em algum momento. Será que uma rede ainda é possível?},
  answer text one = {\emph{Correto.} Sim. Não é fácil ver. Mas o exterior preto permanece forte o suficiente para capturar as duas pedras brancas.},
  answer text two = {\emph{Incorreto.} Branco pode escapar. No caso, escapar com A é melhor, pois escapar com 1 leva a um ``squeeze'' ou ``espremida'', isto é, Preto sacrifica as pedras para obter uma força bastante considerável no exterior.},
]

\problemAnswerDiagram[%
  sgf folder      = liberdades/redes,
  sgf filename    = redes.10,
  problem text    = {\emph{Correto.} Capturar as pedras de corte é praticamente um milagre em uma situação como esta.},
  answer text one = {\emph{Correto.} Capturar as pedras de corte é praticamente um milagre em uma situação como esta.},
  answer text two = {\emph{Incorreto.} Branco 1 é um movimento simples e, em geral, é boa técnica atravessar o \emph{keima} --- movimento do cavaleiro, ou o ``L'' do cavalo no xadrez --- de A para B do adversário, mas, desta vez, não funciona.},
]

\problemAnswerDiagram[%
  sgf folder      = liberdades/redes,
  sgf filename    = redes.2,
  problem text    = {Capturar pedras de corte é frequentemente algo de extrema importância no Go.},
  answer text one = {\emph{Correto.} Novamente, Preto não consegue aumentar suas liberdades ou utilizar fraquezas no exterior preto para poder escapar.},
  answer text two = {\emph{Incorreto.} Seja esta escada ou a de A, ambas são capturas condicionais, o que é quase sempre subótimo.},
]

\problemAnswerDiagram[%
  sgf folder      = liberdades/redes,
  sgf filename    = redes.5,
  problem text    = {Esta já é uma rede bem mais avançada.},
  answer text one = {\emph{Correto.} O importante é que Branco não vai conseguir mais de 2 liberdades com a pedra de corte, o que é menos do que as pedras pretas que a cercam.},
  answer text two = {\emph{Incorreto.} Às vezes, 1 pode sera a rede ótima, mas, neste caso, Branco está muito forte no exterior.},
]

\problemAnswerDiagram[%
  sgf folder      = liberdades/redes,
  sgf filename    = redes.12,
  problem text    = {Jogadores muito avançados vão ou errar ou tomar um tempo considerável aqui.},
  answer text one = {\emph{Correto.} Branco não tem escapatória.},
  answer text two = {\emph{Incorreto.} Esta rede parece funcionar também, mas Preto está muito frágil no exterior.},
]

\problemAnswerDiagram[%
  sgf folder      = liberdades/redes,
  sgf filename    = redes.14,
  problem text    = {Mais uma forma muito comum.},
  answer text one = {\emph{Correto.} Branco conecta seus grupos e se torna muito mais forte. Branco não deveria tentar escapar, se não há como fazê-lo.},
  answer text two = {\emph{Incorreto.} A rede de Preto 3 parece funcionar, mas Preto não possui muitas liberdades como grupo A.},
]

\problemAnswerDiagram[%
  sgf folder      = liberdades/redes,
  sgf filename    = redes.13,
  problem text    = {Esta é uma sequência bem comum e parelha se jogada na hora certa, pois, se Preto encontrar o melhor movimento, seu exterior ficará muito forte.},
  answer text one = {\emph{Correto.} O atari de Branco 4 é um pouco alarmante, mas não há escapatória.},
  answer text two = {\emph{Incorreto.} Em 99\% dos casos, capturar em escada, ou seja, uma captura condicional, não é ideal, isto é, se uma captura incondicional for possível.},
]

\problemAnswerDiagram[%
  sgf folder      = liberdades/redes,
  sgf filename    = redes.15,
  problem text    = {Depois de uma sequência padrão de invasão de canto, há 2 redes possíveis. Qual é a melhor?},
  answer text one = {\emph{Correto.} A diferença é bastante sutil e avançada. Mas a rede com Preto 1 aqui é melhor pois a forma preta é mais forte e flexível, dado que é possível capturar em escada em último caso também. Esta forma também é mais expansiva.},
  answer text two = {\emph{Subótimo.} Esta rede também funciona, mas é mais contraída.},
]

\problemAnswerDiagram[%
  sgf folder      = liberdades/redes,
  sgf filename    = redes.16,
  problem text    = {Este é um dos poucos tesujis que possui nome próprio.},
  answer text one = {\emph{Correto.} Em inglês, o nome desta rede é ``crane's nest'', ou ``ninho da garça''. Um dos principais aspectos para que esta rede funcione é ter as pedras A e B.},
  answer text two = {\emph{Incorreto.} Se Preto fizer 1, Branco resgata as pedras de corte, e as pedras A morrem.},
]

\clearedpage
\clearedpage