\chapter{Corridas de Captura}

\emptypage

\problemAnswerDiagram[%
  sgf folder      = liberdades/corridas_de_captura,
  sgf filename    = corridas_de_captura.1,
  problem text    = {Corridas de captura --- ou, em japonês, \emph{semeai} --- são provavelmente o tópico de leitura mais complexo no Go. Porém, há princípios que simplificam drasticamente sua complexidade.},
  answer text one = {\emph{Correto.} Manter-se conectado aumenta o número de liberdades para Preto e, assim, pode-se ganhar a corrida.},
  answer text two = {\emph{Incorreto.} Se simplesmente tentarmos diminuir as liberdades brancas por fora, o que é um outro e muito bom princípio, desta vez, não funcionará. É preciso checar se o princípio funciona, ao invés de jogá-lo diretamente.},
]

\problemAnswerDiagram[%
  sgf folder               = liberdades/corridas_de_captura,
  sgf filename             = corridas_de_captura.2,
  problem text             = {Aqui, temos um outro princípio muito importante. Qual?},
  answer text one          = {\emph{Correto.} Este é o princípio de ``olho vs sem olho''. Em geral, o grupo que possui olho guarda as liberdades internas para si, garantindo uma grande vantagem na corrida. Por exemplo, Branco não pode se aproximar em A depois de Preto 1.},
  answer text two          = {\emph{Variação.} Se Branco puder jogar, pode ganhar a corrida fazendo retirando o olho ao fazer atari.},
  answer diagram clip vert = 10,
]

\problemAnswerDiagram[%
  sgf folder      = liberdades/corridas_de_captura,
  sgf filename    = corridas_de_captura.3,
  problem text    = {Uma possível armadilha para quem está vendo isto pela primeira vez.},
  answer text one = {\emph{Correto.} Preto impossibilita que Branco jogue A ou B, pois seria auto-atari. Este tipo de movimento que trava o adversário com atari é outra técnica muito útil.},
  answer text two = {\emph{Incorreto.} Ao bloquear com 2, Branco põe Preto em atari.},
]

\problemAnswerDiagram[%
  sgf folder      = liberdades/corridas_de_captura,
  sgf filename    = corridas_de_captura.4,
  problem text    = {Tente combinar múltiplas das técnicas que vimos até aqui.},
  answer text one = {\emph{Correto.} Em primeiro lugar, vemos que Preto possui um olho, ou seja, as liberdades internas são mais provavelmente pretas, mas só se jogarmos algo para reduzir as liberdades brancas. Note que parece restar um ko, mas este ko pode ser clareado no final da partida, é um privilégio preto.},
  answer text two = {\emph{Incorreto.} Preto 1 parece similar à resposta, mas Branco bloqueia e chegamos a um impasse local, uma vida mútua, o que veremos mais à frente. Preto não poderá jogar em A, ambos os grupos permanecem no tabuleiro, sem fazer pontos.},
]

\problemAnswerDiagram[%
  sgf folder      = liberdades/corridas_de_captura,
  sgf filename    = corridas_de_captura.5,
  problem text    = {Precisamos de uma heurística nova?},
  answer text one = {\emph{Correto.} Ao fazer um olho, reduzimos o problema a ``olho vs sem olho''. Branco não consegue se aproximar por lugar nenhum, Preto pode capturar quando quiser.},
  answer text two = {\emph{Incorreto.} Saber que há a possibilidade de ko é uma técnica muito útil, mas desnecessária e errada aqui. Além disso, este é um ko em que Branco captura primeiro e, se Preto perder, Branco captura em A, fazendo olho e colocando B sob atari.},
]

\problemAnswerDiagram[%
  sgf folder      = liberdades/corridas_de_captura,
  sgf filename    = corridas_de_captura.6,
  problem text    = {Já vimos a técnica de captura a ser aplicada, mas não no contexto de corridas de captura.},
  answer text one = {\emph{Correto.} Preto captura e recaptura para configurar um olho falso branco.},
  answer text two = {\emph{Continuação.} Já que somente Preto possui olho, a corrida é garantida para Preto.},
]

\problemAnswerDiagram[%
  sgf folder      = liberdades/corridas_de_captura,
  sgf filename    = corridas_de_captura.7,
  problem text    = {Parece ser uma corrida bastante específica, mas este tipo de forma é bastante recorrente em partidas.},
  answer text one = {\emph{Correto.} Preto garante um olho para expandir suas liberdades e começar a reduzir as liberdades brancas por fora.},
  answer text two = {\emph{Incorreto.} Se Preto tentar reduzir as liberdades diretamente, é colocado em atari. Jogar em A parece um ko, mas Preto será capturado diretamente.},
]

\problemAnswerDiagram[%
  sgf folder      = liberdades/corridas_de_captura,
  sgf filename    = corridas_de_captura.8,
  problem text    = {Uma das técnicas de capítulos anteriores.},
  answer text one = {\emph{Correto.} Com um sacrifício, forçamos Preto a basicamente colidir com a lateral, reduzindo uma liberdade e garantindo a corrida.},
  answer text two = {\emph{Incorreto.} Sem o sacrifício, Branco acaba com uma liberdade extra.},
]

\problemAnswerDiagram[%
  sgf folder      = liberdades/corridas_de_captura,
  sgf filename    = corridas_de_captura.9,
  problem text    = {Vamos revisitar o conceito de ``olho vs sem olho''. Preto pode partir para uma corrida de liberdades difícil em A ou simplesmente se conectar em B.},
  answer text one = {\emph{Correto.} Apesar de que Preto não possui olho, há liberdades suficientes para ganhar a corrida.},
  answer text two = {\emph{Incorreto.} Branco vive, salvando um grupo grnade, enquanto Preto não ganha muita coisa localmente, no máximo, sente.},
]

\problemAnswerDiagram[%
  sgf folder      = liberdades/corridas_de_captura,
  sgf filename    = corridas_de_captura.10,
  problem text    = {Há uma pequena mudança agora. Será que é ainda a mesma resposta?},
  answer text one = {\emph{Correto.} Desta vez, Preto precisa se conectar, pois não ganhará a corrida.},
  answer text two = {\emph{Incorreto.} Já que Preto precisa se conectar em 3 antes de reduzir a liberdade de 5, Branco consegue um movimento a mais no exterior, o que muda completamente a corrida. Esta é uma das mágicas do canto, essencial para a leitura de corridas. Preto não conseguirá mais jogar em A pois será auto-atari.},
]

\clearedpage
\clearedpage
