\chapter{Prefácio}

Há 3 pilares para se aprimorar no Go, qual seja o seu nível, desde iniciante até campeão mundial: jogar, revisar e praticar \emph{tsumegos}. Caso o leitor não os conheça, tsumegos são problemas isolados ou localizados no jogo de Go.

Evidentemente, jogando e revisando também é possível praticar tsumegos, mas o volume por unidade de tempo será bem mais baixo, já que, em partidas, há muito mais elementos com que se preocupar. Tsumegos acabam sendo, desta maneira, o análogo à musculação para atletas de esportes físicos: uma maneira eficiente de se treinar algo específico relacionado ao esporte desejado.

Nesse sentido, o propósito principal deste livro, isto é, da coleção \emph{Técnicas de Go}, é servir de referência de sequências e padrões essenciais --- algo como um dicionário --- seja para aqueles que acabaram de aprender as regras, ou até mesmo para jogadores mais experientes e avançados. Começamos basicamente desde as regras de captura, suicídio e ko, e progredimos gradualmente até técnicas que aparecem em partidas mais avançadas, e que podem ser devastadoras para quem nunca as viu, como é o caso de sequências que espremem e destroem a forma adversária. 

Ao longo desta série, os comentários das respostas focam em criar uma ponte entre a visão de alguém que está começando e aquela de um jogador mais avançado. Mesmo assim, progredir linearmente neste livro pode ser bastante desafiador. Se for este o caso, então melhor ainda, pois é sinal de que há muito o que aprender. Persista! Como treinamento, sugiro fortemente refazer o livro múltiplas vezes, provavelmente espaçadas por uma ou mais semanas. Repetição será essencial para tornar as técnicas intuitivas: errar na primeira vez será fácil, errar depois da décima garanto que será muito mais difícil!

A metodologia de ensino deste livro não é, no entanto, necessariamente a melhor possível, pois, afinal, não há uma ideal. Tipicamente, ocidentais são introduzidos ao Go por interesses em estratégia, ou mesmo por curiosidades sobre a cultura e filosofia asiáticas e, sendo assim, é natural um foco maior em conceitos mais abstratos e abertura, ao invés de técnicas e táticas. Nesta série, porém, começamos mais por algo local do que focando em como as pedras interagem no tabuleiro globalmente, pois uma das lições de guerra mais importantes é que, mesmo se você for o melhor estrategista da história, se seus soldados não conseguem executar suas ordens, a guerra estará perdida de qualquer maneira. Claro, isso não quer dizer que táticas sejam mais importantes do que estratégia, ambas são necessárias, e a interação entre as duas é onde tanto guerras quanto Go se tornam complexos e interessantíssimos.

Mas nada impede que o leitor também suplemente seu aprendizado com material complementar, inclusive, é aconselhável. De todo modo, nos próximos volumes, estudaremos os outros tópicos essenciais remanescentes.

Caso o leitor queira um maior volume e diversidade de tsumegos, hoje em dia, não há recurso melhor do que o \shref{https://101weiqi.com}{101weiqi.com}, com mais de 160,000 problemas aprovados por moderadores extremamente fortes. A interface desse site está em chinês (mandarim), mas ela pode ser traduzida para o inglês, em grande parte, com a extensão de navegador \shref{https://chromewebstore.google.com/detail/101weiqilocalizer/emhhlhigmokehndjjmgnailciakdmoba}{101weiqiLocalizer} (o Google Tradutor também ajuda bastante).

E outro recurso importante é o servidor OGS (\shref{https://online-go.com}{online-go.com}), que também possui muitos tsumegos, além de uma excelente página de referências (\shref{https://online-go.com/docs/other-go-resources}{online-go.com/docs/other-go-resources}).

\vspace{-0.05cm}

\begin{center}
  \rule{3.25cm}{0.15mm}
\end{center}

\vspace{0.1cm}

Para aqueles que não me conhecem, sou formado em engenharia elétrica pela Escola Politécnica da USP, onde conheci o Go no final de 2012. Ao redor de quando escrevo este livro, oscilo entre 2 e 3 \emph{dan} --- dan é o equivalente de mestre ou faixa-preta no Go e, na verdade, o termo faixa-preta foi inspirado pelo Go! --- amador, no servidor KGS (\shref{https://www.gokgs.com/}{gokgs.com}), ou algo ao redor de 5 e 6 dan no servidor Fox Weiqi (\shref{https://www.foxwq.com/soft/foreign.html}{foxwq.com/soft/foreign.html}).

Há já pelo menos 5 anos, compartilho conteúdo de Go online especialmente pelo meu canal de YouTube em português: \shref{https://youtube.com/@Fanaro}{youtube.com/@Fanaro}. Mas, também, pelo meu blog: \shref{https://fanaro.io}{fanaro.io}; e meu novo canal em inglês: \shref{https://www.youtube.com/@gowithfanaro}{youtube.com/@gowithfanaro}.

Outro conteúdo útil a mencionar é a tradução do livro \emph{Como Jogar Go --- Uma Introdução Concisa}, de Richard Bozulich e James Davies, que eu fiz em código aberto alguns anos atrás, com \LaTeX. Seu PDF é gratuito, e ela está disponível aqui: \shref{https://github.com/psygo/traducao_como_jogar_go}{github.com/psygo/traducao_como_jogar_go}. É um ótimo material para complementar o aspecto tático deste livro. Este livro e série, inclusive, também foi desenvolvido em código aberto: \shref{https://github.com/psygo/tecnicas_de_go}{github.com/psygo/tecnicas_de_go}.

Como recurso extra, para quem mora em ou perto de São Paulo-SP, existe a Brasil Nihon Kiin, que é a sede da Associação Japonesa de Go (e Shogi) no Brasil. Além de possuírem muitos membros que são jogadores avançados, eles também oferecem aulas, grupos de estudos, torneios, livros, e fazem uma interface com a cultura japonesa no Brasil --- e também possuem presença online.

Por fim, gostaria de agradecer primeiramente à minha família por todo o apoio e incentivo, e, também, especialmente aos meus amigos que ajudaram a revisar e melhorar este livro com sugestões de problemas, diagramas e correções de português. São eles: Eunkyo Do 1p, Simão Gonçalves, Samuel Karasin, Felipe Herman van Riemsdijk --- que é o atual presidente da Brasil Nihon Kiin ---, Renan Cruz, Ronaldo Matayoshi, Helcio Pacheco, Hiroaki Ogawa, Ritsuki Hasegawa, Felipe Pait, Ariel Oliveira, Gilberto Espínola, João Marcus Fernandes e Michael (TakumiGo) Cheung. E, claro, caso o leitor possua sugestões de melhorias, não hesite em me contatar!

\bigskip
\smallskip
\smallskip
\smallskip

\hspace*{\fill} Philippe Fanaro \hspace{0.055cm}

\hspace*{\fill} Novembro de 2024 \hspace{0.05cm}

\clearedpage
\clearedpage