\chapter{Prefácio}

Há 3 pilares para se aprimorar no Go, qual seja o seu nível, desde iniciante até campeão mundial: jogar, revisar e praticar \emph{tsumegos}. Para quem não os conhece, tsumegos são problemas isolados ou localizados, no jogo de Go.

Evidentemente, jogando e revisando também é possível praticar tsumegos, mas o volume por unidade de tempo será bem mais baixo, já que, em partidas, há muito mais com o que se preocupar. Tsumegos são o equivalente de musculação para atletas de esportes físicos: uma maneira eficiente de se treinar algo específico relacionado ao esporte desejado.

O propósito principal deste livro, e da coleção \emph{Tsumegos para Iniciantes}, é servir de referência de técnicas e padrões essenciais, algo como um dicionário, para aqueles que acabaram de aprender as regras, ou já jogaram suas primeiras dezenas, ou mesmo centenas, de partidas. E, assim, criar uma ponte entre a visão de alguém que está começando e a de um jogador mais avançado, além de ter um volume mínimo de exercícios para que se possa ser considerado como uma sessão de treinamento. Inclusive, como treinamento, sugiro fortemente refazer o livro múltiplas vezes, provavelmente espaçadas por uma ou mais semanas. Repetição será essencial para tornar as técnicas intuitivas: errar na primeira vez será fácil, errar depois da décima garanto que será muito mais difícil!

A metodologia de ensino deste livro não é necessariamente a melhor possível, pois não há uma ideal. Tipicamente, ocidentais são introduzidos ao Go por interesses em estratégia, ou mesmo por curiosidades sobre a cultura e filosofia asiáticas. Dessa maneira, é natural um foco maior em conceitos mais abstratos e abertura, ao invés de técnicas e táticas. Nesta série, porém, começamos mais por algo local do que focando em como as pedras interagem no tabuleiro globalmente, pois uma das lições de guerra mais importantes é que, mesmo se você for o melhor estrategista da história, se seus soldados não conseguem executar suas ordens, a guerra estará perdida de qualquer maneira. Claro, isso não quer dizer que táticas são mais importantes do que estratégia, ambas são necessárias, e a interação entre as duas é onde tanto guerras quanto Go se tornam complexos e interessantíssimos.

Mas nada impede que o leitor também suplemente seu aprendizado com material complementar, inclusive, é aconselhável. De todo modo, nos próximos volumes, estudaremos os outros tópicos essenciais remanescentes.

Caso o leitor queira um maior volume e diversidade de tsumegos, hoje em dia, não há recurso melhor do que o \shref{https://101weiqi.com}{101weiqi.com}, com mais de 160,000 problemas aprovados por moderadores. A interface desse site está em chinês, mas ela pode ser traduzida para o inglê em grande parte pela extensão de navegador \shref{https://chromewebstore.google.com/detail/101weiqilocalizer/emhhlhigmokehndjjmgnailciakdmoba}{101weiqiLocalizer} (o Google Tradutor também ajuda bastante).

E outro recurso importante é o servidor OGS (\shref{https://online-go.com}{online-go.com}), que também possui muitos tsumegos, além de uma excelente página de referências.

\bigskip
\bigskip

Para aqueles que não me conhecem, sou formado em engenharia elétria pela Escola Politécnica da USP, onde conheci o Go no final de 2012. Ao redor de quando escrevo este livro, oscilo entre 2 e 3 dan amador --- \emph{dan} é o equivalente de mestre ou faixa-preta no Go, na verdade, o termo faixa-preta foi inspirado pelo Go! --- no servidor KGS, ou algo ao redor de 5 dan no servidor Fox.

Há já pelo menos 5 anos, compartilho conteúdo de Go online especialmente pelo meu canal de YouTube em português: \shref{https://youtube.com/@Fanaro}{youtube.com/@Fanaro}. Mas, também, pelo meu blog: \shref{https://fanaro.io}{fanaro.io}; e meu novo canal em inglês: \shref{https://www.youtube.com/@gowithfanaro}{youtube.com/@gowithfanaro}.

Outro conteúdo útil a mencionar é a tradução do livro \emph{Como Jogar Go --- Uma Introdução Concisa}, de Richard Bozulich e James Davies, que eu fiz em código aberto alguns anos atrás. Ela está disponível aqui, e seu PDF é gratuito: \shref{https://github.com/psygo/traducao\_como\_jogar\_go}{github.com/psygo/traducao_como_jogar_go}. É um ótimo material para complementar o aspecto tático deste livro.

Como recurso extra, para aqueles que moram em ou perto de São Paulo-SP, existe a Brasil Nihon Kiin, que é a sede da associação japonesa de Go (e shogui) no Brasil. Além de possuírem muitos membros que são jogadores avançados, eles também oferecem aulas, grupos de estudos, torneios, livros, e fazem uma interface com a cultura japonesa no Brasil --- eles também possuem presença online.

Por fim, gostaria de agradecer primeiramente à minha família por todo o apoio e incentivo, e, também, aos meus amigos e revisores, que ajudaram a melhorar este livro com sugestões de problemas, diagramas e correções de português. São eles: Samuel Karasin, Felipe Herman van Riemsdijk e Renan Cruz.

\bigskip
\smallskip
\smallskip
\smallskip

\hspace*{\fill} Philippe Fanaro \hspace{0.055cm}

\hspace*{\fill} Outubro de 2024 \hspace{0.05cm}

\blankpage
\blankpage