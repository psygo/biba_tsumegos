\chapter{Vida na Segunda Linha}

\emptypage

\problemAnswerDiagram[%
  sgf folder      = vida_ou_morte/vida_segunda_linha,
  sgf filename    = vida_segunda_linha.1,
  problem text    = {Será que há mais de uma maneira de viver?},
  answer text one = {\emph{Correto.} Preto precisa garantir as 4 intersecções internas.},
  answer text two = {\emph{Incorreto.} Branco utiliza um sacrifício para reduzir o espaço de olho preto para uma forma morta de 3 intersecções. Preto desejava que Branco respondesse em A para depois conectar em 2, mas isso foi um pouco ingênuo infelizmente.},
]

\problemAnswerDiagram[%
  sgf folder               = vida_ou_morte/vida_segunda_linha,
  sgf filename             = vida_segunda_linha.2,
  problem text             = {Branco jogou em A pois imaginou que ainda não estivesse vivo, mas é o suficiente?},
  answer text one          = {\emph{Correto.} Aplicando um sacrifício, conseguimos reduzir a forma de olho a uma forma morta de 3 espaços. Note que 6 pedras na segunda linha não são o suficiente para viver.},
  answer text two          = {\emph{Incorreto.} Se simplesmente bloquearmos, Branco consegue 4 espaços, o que é suficiente para viver.},
  answer diagram clip vert = 9,
]

\problemAnswerDiagram[%
  sgf folder               = vida_ou_morte/vida_segunda_linha,
  sgf filename             = vida_segunda_linha.3,
  problem text             = {Preto está no limiar entre a vida e a morte.},
  answer text one          = {\emph{Correto.} Tanto Preto 1 quanto A funcionariam para criar um espaço de 4 intersecções. Ou seja, 7 pedras na segunda linha vivem, caso Preto jogue primeiro.},
  answer text two          = {\emph{Incorreto.} Jogar em 1 é con-tra-atacado com uma técnica de sacrifício.},
  answer diagram clip vert = 9,
]

\problemAnswerDiagram[%
  sgf folder               = vida_ou_morte/vida_segunda_linha,
  sgf filename             = vida_segunda_linha.4,
  problem text             = {Um problema um pouco diferente. Preto precisa de mais um movimento para viver?},
  answer text one          = {\emph{Correto.} Preto pode fazer tenuki! Não há como reduzir o espaço interno para menos de 4 intersecções, ou seja, 8 pedras na segunda linha já configura um grupo vivo.},
  answer text two          = {\emph{Incorreto.} Preto 1 termina por ser um movimento bastante pequeno de fim de jogo.},
  answer diagram clip vert = 9,
]

\problemAnswerDiagram[%
  sgf folder               = vida_ou_morte/vida_segunda_linha,
  sgf filename             = vida_segunda_linha.5,
  problem text             = {Criar um olho grande ou criar um olho falso?},
  answer text one          = {\emph{Correto.} Em algum momento, Branco terá que conectar em A, o que concluirá a forma morta de 3 pedras internamente.},
  answer text two          = {\emph{Incorreto.} Sacrificar-se em 1 não gera olho falso na verdade. Um sacrifício de 3 pedras na primeira linha gera um olho neste tipo de situação: Preto em 1 após a captura, e Branco ganha um olho ao jogar em A.},
  answer diagram clip vert = 10,
]