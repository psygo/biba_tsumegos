\chapter{Olhos Falsos}

\emptypage

\problemAnswerDiagram[%
  sgf folder      = vida_ou_morte/olhos_falsos,
  sgf filename    = olhos_falsos.5,
  problem text    = {Branco está bem perto de viver.},
  answer text one = {\emph{Correto.} Não há mais como fazer dois olhos.},
  answer text two = {\emph{Incorreto.} Se Branco conseguir jogar em 1, a vida é incondicional.},
]

\problemAnswerDiagram[%
  sgf folder               = vida_ou_morte/olhos_falsos,
  sgf filename             = olhos_falsos.9,
  problem text             = {Branco está bem perto de viver.},
  answer text one          = {\emph{Correto.} Não há mais como fazer dois olhos.},
  answer text two          = {\emph{Incorreto.} Se Branco conseguir jogar em 1, a vida é incondicional.},
  answer diagram clip vert = 10,
]

\problemAnswerDiagram[%
  sgf folder               = vida_ou_morte/olhos_falsos,
  sgf filename             = olhos_falsos.10,
  problem text             = {Branco está bem perto de viver.},
  answer text one          = {\emph{Correto.} Não há mais como fazer dois olhos.},
  answer text two          = {\emph{Incorreto.} Se Branco conseguir jogar em 1, a vida é incondicional.},
  answer diagram clip vert = 9,
]

\problemAnswerDiagram[%
  sgf folder               = vida_ou_morte/olhos_falsos,
  sgf filename             = olhos_falsos.11,
  problem text             = {Branco está bem perto de viver.},
  answer text one          = {\emph{Correto.} Não há mais como fazer dois olhos.},
  answer text two          = {\emph{Incorreto.} Se Branco conseguir jogar em 1, a vida é incondicional.},
  answer diagram clip vert = 9,
]

% \problemAnswerDiagram
%   {vida_ou_morte/olhos_falsos}
%   {olhos_falsos.5}
%   {Branco está bem perto de viver.}
%   {\emph{Correto.} Não há mais como fazer dois olhos.}
%   {\emph{Incorreto.} Se Branco conseguir jogar em 1, a vida é incondicional.}

% \problemAnswerDiagram
%   {vida_ou_morte/olhos_falsos}
%   {olhos_falsos.4}
%   {Preto protegeu parte dos olhos do canto. É o suficiente?}
%   {\emph{Correto.} Preto precisa de mais uma proteção para garantir seus dois olhos.}
%   {\emph{Incorreto.} Branco pode falsificar o segundo olho com um sacrifício.}

% \problemAnswerDiagram
%   {vida_ou_morte/olhos_falsos}
%   {olhos_falsos.1}
%   {Tão perto de viver quanto de morrer.}
%   {\emph{Correto.} Ao conectar, Preto possui espaço interno o suficiente para criar dois olhos. Se Branco jogar em A, Preto joga em B, e vice-versa.}
%   {\emph{Incorreto.} A intersecção de A vira olho falso.}

% \problemAnswerDiagram
%   {vida_ou_morte/olhos_falsos}
%   {olhos_falsos.2}
%   {Um problema bastante traiçoeiro.}
%   {\emph{Correto.} Preto protege seus dois olhos, em A e B.}
%   {\emph{Incorreto.} Ao jogar 2, Branco falsifica o olho de A diretamente, e Preto terá que conectar em B mais tarde, ou seja, também é olho falso!}

% \problemAnswerDiagram
%   {vida_ou_morte/olhos_falsos}
%   {olhos_falsos.7}
%   {Um problema contra-intuitivo, para quem nunca viu algo assim.}
%   {\emph{Correto.} Parece que Branco pode cortar em A, mas ele não possui liberdades suficientes.}
%   {\emph{Incorreto.} Qualquer outro movimento e Branco viverá com 2.}

% \problemAnswerDiagram
%   {vida_ou_morte/olhos_falsos}
%   {olhos_falsos.8}
%   {Branco tem uma jogada preparada.}
%   {\emph{Correto.} Branco não consegue cortar em A, e acabará com somente um olho.}
%   {\emph{Incorreto.} O atari de 1 parece ser simples e eficaz, o que geralmente é boa técnica, mas Branco bloqueia o olho do canto através de uma recaptura.}

% \problemAnswerDiagram
%   {vida_ou_morte/olhos_falsos}
%   {olhos_falsos.6}
%   {Por onde falsificar um olho primeiro?}
%   {\emph{Correto.} Branco não consegue estabelecer um segundo olho no canto.}
%   {\emph{Incorreto.} Ao garantir o ponto vital de 2, que era a resposta para Preto, Branco garante miai de A ou B para seu segundo olho.}