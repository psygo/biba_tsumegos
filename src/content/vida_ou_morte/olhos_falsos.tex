\chapter{Olhos Falsos}

\emptypage

\problemAnswerDiagram[%
  sgf folder      = vida_ou_morte/olhos_falsos,
  sgf filename    = olhos_falsos.5,
  problem text    = {Branco está bem perto de viver.},
  answer text one = {\emph{Correto.} Não há mais como fazer dois olhos.},
  answer text two = {\emph{Variação.} Se Branco conseguir jogar em 1, a vida é incondicional.},
]

\problemAnswerDiagram[%
  sgf folder      = vida_ou_morte/olhos_falsos,
  sgf filename    = olhos_falsos.4,
  problem text    = {Preto protegeu parte dos olhos do canto. É o suficiente?},
  answer text one = {\emph{Correto.} Preto precisa de mais uma proteção para garantir seus dois olhos.},
  answer text two = {\emph{Variação.} Branco pode falsificar o segundo olho com um sacrifício.},
]

\problemAnswerDiagram[%
  sgf folder      = vida_ou_morte/olhos_falsos,
  sgf filename    = olhos_falsos.1,
  problem text    = {Tão perto de viver quanto de morrer.},
  answer text one = {\emph{Correto.} Ao conectar, Preto possui espaço interno o suficiente para criar dois olhos. Se Branco jogar em A, Preto joga em B, e vice-versa.},
  answer text two = {\emph{Incorreto.} A intersecção de A vira olho falso.},
]

\problemAnswerDiagram[%
  sgf folder      = vida_ou_morte/olhos_falsos,
  sgf filename    = olhos_falsos.2,
  problem text    = {Este tipo de situação em que parece haver múltiplas maneiras de viver requer muito cuidado, pois várias podem ser falsos positivos.},
  answer text one = {\emph{Correto.} Preto protege o ponto-chave que mantém os 2 olhos.},
  answer text two = {\emph{Incorreto.} Ao jogar 2, Branco falsifica o olho de A diretamente, e Preto terá que conectar em B mais tarde, ou seja, também é olho falso!},
]

\problemAnswerDiagram[%
  sgf folder      = vida_ou_morte/olhos_falsos,
  sgf filename    = olhos_falsos.7,
  problem text    = {Um problema contra-intuitivo, para quem nunca viu algo assim.},
  answer text one = {\emph{Correto.} Parece que Branco pode cortar em A, mas ele não possui liberdades suficientes.},
  answer text two = {\emph{Incorreto.} Qualquer outro movimento e Branco viverá com 2.},
]

\problemAnswerDiagram[%
  sgf folder      = vida_ou_morte/olhos_falsos,
  sgf filename    = olhos_falsos.8,
  problem text    = {Branco tem uma jogada preparada.},
  answer text one = {\emph{Correto.} Não é possível cortar em A, e o grupo branco acaba com somente um olho.},
  answer text two = {\emph{Incorreto.} O atari de 1 parece ser simples e eficaz, o que geralmente é boa técnica, mas Branco bloqueia o olho do canto através de uma recaptura.},
]

\problemAnswerDiagram[%
  sgf folder               = vida_ou_morte/olhos_falsos,
  sgf filename             = olhos_falsos.6,
  problem text             = {Por onde falsificar um olho primeiro?},
  answer text one          = {\emph{Correto.} Branco não consegue estabelecer um segundo olho no canto.},
  answer text two          = {\emph{Incorreto.} Veja que Preto não pode jogar em A pois seria suicídio, então Branco já possui olho ali, não há como falsificá-lo.},
  answer diagram clip vert = 10,
]

\problemAnswerDiagram[%
  sgf folder      = vida_ou_morte/olhos_falsos,
  sgf filename    = olhos_falsos.3,
  problem text    = {Consegue visualizar a técnica?},
  answer text one = {\emph{Correto.} Preto impossibilita a criação de um segundo olho verdadeiro.},
  answer text two = {\emph{Variação.} Preto precisa ficar atento com o corte em 4 no entanto. Se não houver espaço para fugir, talvez seja melhor deixar Branco viver.},
]

\problemAnswerDiagram[%
  sgf folder      = vida_ou_morte/olhos_falsos,
  sgf filename    = olhos_falsos.9,
  problem text    = {Esta é uma invasão de canto padrão que pode funcionar em muitos casos, mas, nesta situação, há uma força extra com as pedras marcadas. (Branco começa em 1, Preto conecta 6 em 1.)},
  answer text one = {\emph{Correto.} Preto consegue puxar a pedra que falsifica o olho para sua força na segunda linha.},
  answer text two = {\emph{Incorreto.} Branco vive, e o grupo preto nem sequer possui 2 olhos.},
]

\problemAnswerDiagram[%
  sgf folder      = vida_ou_morte/olhos_falsos,
  sgf filename    = olhos_falsos.10,
  problem text    = {Algumas pessoas se referem a formas assim como ``florestas de ponnukis'', que possuem uma grande facilidade para criar olhos. Mas elas não são imunes à morte, como veremos a seguir.},
  answer text one = {\emph{Correto.} Ao jogar em 1, impossibilitamos olhos reais em A e B, ao mesmo tempo.},
  answer text two = {\emph{Incorreto.} Branco aproveita o vacilo para garantir o ponto-chave para si.},
]

\clearedpage
\clearedpage