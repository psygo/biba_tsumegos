\chapter{Vida Mútua}

\emptypage

\problemAnswerDiagram[%
  sgf folder      = vida_ou_morte/vida_mutua,
  sgf filename    = vida_mutua.1,
  problem text    = {Se Preto conseguir resgatar as pedras marcadas, ele fará pontos?},
  answer text one = {\emph{Correto.} Já que nenhum dos lados conseguirá jogar em A ou B, já que é auto-atari, ninguém faz pontos nesta região. Grupos que permanecem no tabuleiro sem pontos, em geral, são chamados de vida mútua, ou, mais comumente, \emph{seki}.},
  answer text two = {\emph{Variação.} Se Branco conseguir conectar aqui, finalmente, ele fará pontos. No caso, serão 8 pontos.},
]

\problemAnswerDiagram[%
  sgf folder      = vida_ou_morte/vida_mutua,
  sgf filename    = vida_mutua.2,
  problem text    = {Qual é a melhor maneira de Preto resolver a situação das pedras prestes a serem capturadas?},
  answer text one = {\emph{Correto.} Ao cortar, Preto estabelece um olho e impossibilita que Branco consiga colocá-lo sob atari com A, pois seria auto-atari para Branco.},
  answer text two = {\emph{Variação.} É sempre possível fazer tenuki. Branco então terá a possibilidade de fazer 11 pontos.},
]

\problemAnswerDiagram[%
  sgf folder               = vida_ou_morte/vida_mutua,
  sgf filename             = vida_mutua.3,
  problem text             = {Aqui há mais valor do que somente um seki.},
  answer text one          = {\emph{Correto.} Se Preto jogar em A, será um seki. Mas por que não fazer pontos quando podemos capturar tudo com 1?},
  answer text two          = {\emph{Variação.} Se Branco jogar em 1, já que Preto não pode mais jogar em A, Branco capturará toda a região! Preto estaria em atari se jogasse em B nessa situação. Branco A também funcionaria, ao invés de 1.},
  answer diagram clip vert = 10,
]

\problemAnswerDiagram[%
  sgf folder      = vida_ou_morte/vida_mutua,
  sgf filename    = vida_mutua.4,
  problem text    = {No canto, sekis podem ser um pouco mais estranhos ou peculiares. Inclusive, há outras maneiras ainda mais raras de se permanecer no tabuleiro, como duplos e triplos kos, e ``vidas eternas''.},
  answer text one = {\emph{Correto.} Branco não pode fazer nada no canto pois seria auto-atari. Mas Preto não conseguirá capturar nada, então ninguém faz pontos.},
  answer text two = {\emph{Incorreto.} Branco joga no ponto-chave preto e estabelece um olho. Quando quiser, ele poderá jogar em A ou B para capturar as pedras pretas.},
]

\problemAnswerDiagram[%
  sgf folder      = vida_ou_morte/vida_mutua,
  sgf filename    = vida_mutua.5,
  problem text    = {Estudar formas mortas é de extrema importância.},
  answer text one = {\emph{Correto.} A não ser em caso específico no canto conhecido como ``bent four in the corner'', ou quatro curvado no canto, a forma de 4 pedras que existiria no caso de Branco jogar A ou B não é forma morta. Mas Preto não pode jogar em A ou B, caso contrário teríamos realmente uma forma morta de 3 pedras.},
  answer text two = {\emph{Incorreto.} Ao jogar em 1, Branco forçará Preto a se conectar em A mais tarde, devido ao atari de B.},
]

\problemAnswerDiagram[%
  sgf folder      = vida_ou_morte/vida_mutua,
  sgf filename    = vida_mutua.6,
  problem text    = {Preto conseguiu a pedra marcada em algum momento, o que possibilitou uma sequência de redução de fim de jogo. Como finalizá-la?},
  answer text one = {\emph{Correto.} Branco termina em seki. Ou seja, seu canto, que teria basicamente 6 pontos, agora foi zerado.},
  answer text two = {\emph{Incorreto.} Se Preto tentar outra coisa, Branco capturará o canto inteiro.},
]

\clearedpage
\clearedpage