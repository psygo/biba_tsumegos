\chapter{Dois Olhos}

\emptypage

\problemAnswerDiagram[%
  sgf folder      = vida_ou_morte/dois_olhos,
  sgf filename    = dois_olhos.1,
  problem text    = {Este exercício é uma das essências do Go. É possível permanecer incondicionalmente no tabuleiro?},
  answer text one = {\emph{Correto.} Sim, com 1, estabelecemos ``dois olhos'', o que impossibilita a captura do grupo. Branco precisaria jogar tanto em A quanto em B para capturar, mas ele só possui um movimento por turno, portanto, cada um deles será considerado suicídio!},
  answer text two = {\emph{Incorreto.} Preto talvez pense que seja possível capturar a pedra 1, mas ele morrerá primeiro. Note que, o melhor movimento do seu adversário (jogar 1 para viver como Preto) é também seu melhor movimento.},
]

\problemAnswerDiagram[%
  sgf folder      = vida_ou_morte/dois_olhos,
  sgf filename    = dois_olhos.2,
  problem text    = {Talvez o grupo estar na lateral, e não no canto, mude o seu status.},
  answer text one = {\emph{Correto.} Não, é exatamente a mesma forma do problema anterior.},
  answer text two = {\emph{Incorreto.} E o mesmo movimento para matar.},
]

\problemAnswerDiagram[%
  sgf folder      = vida_ou_morte/dois_olhos,
  sgf filename    = dois_olhos.3,
  problem text    = {Cercar 3 intersecções é insuficiente para viver, a não ser que seja seu turno. E se estivermos cercando de 4 a 5?},
  answer text one = {\emph{Correto.} Cercar 4 é suficiente, pois é impossível que Branco faça com que todo o espaço vire algo indivisível. Se Branco jogar em 3, Preto responderá em 2.},
  answer text two = {\emph{Incorreto.} Branco em 2 é uma tática ligeiramente mais avançada --- e que veremos mais a fundo em breve --- para criar um olho falso. Preto acaba efetivamente cercando somente 3 intersecções.},
]

\problemAnswerDiagram[%
  sgf folder      = vida_ou_morte/dois_olhos,
  sgf filename    = dois_olhos.6,
  problem text    = {Preto precisa de mais um movimento no canto para garantir sua vida?},
  answer text one = {\emph{Correto.} Sim, é preciso reforçar, para criar dois espaços dissociados.},
  answer text two = {\emph{Incorreto.} Branco pode impossibilitar a separação do espaço interno, efetivamente criando um olho grande mas único, o que é insuficiente para Preto viver.},
]

\problemAnswerDiagram[%
  sgf folder      = vida_ou_morte/dois_olhos,
  sgf filename    = dois_olhos.8,
  problem text    = {Esta é uma outra forma que aparece com frequência em partidas. Como completá-la para que viva?},
  answer text one = {\emph{Correto.} Preto configurará um olho no extremo canto, e outro na diagonal.},
  answer text two = {\emph{Incorreto.} Se Preto jogar algo diferente, ou caso Branco possa jogar, o ponto vital continua o mesmo.},
]

\problemAnswerDiagram[%
  sgf folder      = vida_ou_morte/dois_olhos,
  sgf filename    = dois_olhos.5,
  problem text    = {Pode não parecer a princípio, mas esta é uma variação do problema anterior.},
  answer text one = {\emph{Correto.} É basicamente a mesma forma do problema anterior.},
  answer text two = {\emph{Incorreto.} Ao descer, Branco pega o bom movimento preto para si, e faz com que o espaço interno não possa ser dissociado.},
]

\problemAnswerDiagram[%
  sgf folder      = vida_ou_morte/dois_olhos,
  sgf filename    = dois_olhos.7,
  problem text    = {Se Preto não tomar cuidado, cairá em uma grande armadilha.},
  answer text one = {\emph{Correto.} Preto precisa recuar. Branco não consegue resgatar a pedra A, pois as pedras B estariam sob atari.},
  answer text two = {\emph{Incorreto.} Preto se põe em atari e entra em colapso.},
]

\problemAnswerDiagram[%
  sgf folder      = vida_ou_morte/dois_olhos,
  sgf filename    = dois_olhos.4,
  problem text    = {Uma temática de recuo.},
  answer text one = {\emph{Correto.} Preto garante o olho do canto e, também, impossibilita que Branco resgate a pedra A, pois B seria auto-atari.},
  answer text two = {\emph{Incorreto.} Preto acabará com uma forma morta de 3 pedras.},
]

\problemAnswerDiagram[%
  sgf folder      = vida_ou_morte/dois_olhos,
  sgf filename    = dois_olhos.9,
  problem text    = {Se Preto jogar em A para expandir seu espaço interno, o que acontece? É a resposta?},
  answer text one = {\emph{Correto.} O melhor é proteger o ponto-chave 2-1 para garantir um olho e resiliência. O ponto 2-1 costuma ser muito importante neste tipo de situação.},
  answer text two = {\emph{Incorreto.} Nem sequer será um ko --- Branco 2 em 3, e Preto 3 em Branco 2 ---, se Preto expandir seu espaço interno. O interior cercado é uma forma morta de 5 pedras, o que vermos mais a fundo em um próximo capítulo.},
]

\problemAnswerDiagram[%
  sgf folder      = vida_ou_morte/dois_olhos,
  sgf filename    = dois_olhos.10,
  problem text    = {Mais uma potencial armadilha.},
  answer text one = {\emph{Correto.} Branco pode até reduzir o território, mas não conseguirá eliminar os dois olhos.},
  answer text two = {\emph{Incorreto.} Branco estende o sacrifício, criando um olho único e indissociável.},
]

\clearedpage
\clearedpage