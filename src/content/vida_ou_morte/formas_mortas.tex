\chapter{Formas Mortas}

\emptypage

\problemAnswerDiagram[%
  sgf folder      = vida_ou_morte/formas_mortas,
  sgf filename    = formas_mortas.1,
  problem text    = {Como matar o grupo branco?},
  answer text one = {\emph{Correto.} Ao jogar no eixo de simetria, que é geralmente um bom movimento em problemas simétricos, Preto impossibilita que Branco dissocie seu espaço de olho, além de garantir que o espaço de olho seja reduzido a uma das formas mortas.},
  answer text two = {\emph{Variação.} Se Preto não jogar nada, Branco pode viver.},
]

\problemAnswerDiagram[%
  sgf folder      = vida_ou_morte/formas_mortas,
  sgf filename    = formas_mortas.2,
  problem text    = {Será que conseguimos reduzir este problema a algo mais simples?},
  answer text one = {\emph{Correto.} Ao jogar 1, Preto impossibilita a separação do espaço de olho em 2. (Se Branco jogasse em 1, estaria vivo.)},
  answer text two = {\emph{Continuação.} Preto reduzirá o espaço de olho para uma forma morta de 3 pedras. Em japonês, o termo para forma morta é \emph{nakade}.},
]

\problemAnswerDiagram[%
  sgf folder      = vida_ou_morte/formas_mortas,
  sgf filename    = formas_mortas.17,
  problem text    = {Você consegue visualizar como este problema converge a outros?},
  answer text one = {\emph{Correto.} Preto protege seus dois olhos ao mesmo tempo.},
  answer text two = {\emph{Incorreto.} Expandir o espaço interno nem sempre será a solução.},
]

\problemAnswerDiagram[%
  sgf folder      = vida_ou_morte/formas_mortas,
  sgf filename    = formas_mortas.11,
  problem text    = {E agora? Preto está cercado. Resta somente uma corrida de liberdades para capturar o canto. Será que ele vence?},
  answer text one = {\emph{Correto.} Preto ganhará a corrida por 1 liberdade.},
  answer text two = {\emph{Continuação.} Branco pode capturar as pedras no canto, mas não conseguirá aumentar suas liberdades.},
]

\problemAnswerDiagram[%
  sgf folder      = vida_ou_morte/formas_mortas,
  sgf filename    = formas_mortas.5,
  problem text    = {O ponto-chave do seu oponente é frequentemente o seu ponto-chave, mas para matar.},
  answer text one = {\emph{Correto.} No máximo, Branco conseguirá um olho grande de 2 ou 3 pedras.},
  answer text two = {\emph{Variação.} Agora, há 2 olhos.},
]

\problemAnswerDiagram[%
  sgf folder      = vida_ou_morte/formas_mortas,
  sgf filename    = formas_mortas.12,
  problem text    = {Preto consegue capturar o canto antes de ser capturado? Será que ele consegue até fazer \emph{tenuki} --- ou seja, jogar em outro lugar ignorando a situação local ---? Se sim, quantos?},
  answer text one = {\emph{Correto.} Preto consegue até jogar em outro lugar 1 vez! Branco só consegue 3 liberdades no canto, devido a uma recaptura de forma morta. Esta é uma das magias das formas mortas. É uma leitura avançada, mas é essencial ver como ela acontece, mesmo sem conseguir lê-la.},
  answer text two = {\emph{Continuação.} A corrida termina em sucesso para Preto.},
]

\problemAnswerDiagram[%
  sgf folder      = vida_ou_morte/formas_mortas,
  sgf filename    = formas_mortas.6,
  problem text    = {Talvez não pareça, mas, uma vez que formas mortas se tornarem mais intuitivas, este problema será essencialmente idêntico a vários dos anteriores.},
  answer text one = {\emph{Correto.} Novamente, uma forma morta de 3 pedras.},
  answer text two = {\emph{Incorreto.} Se Branco puder jogar, ele conseguirá dividir seu espaço interno.},
]

\problemAnswerDiagram[%
  sgf folder      = vida_ou_morte/formas_mortas,
  sgf filename    = formas_mortas.3,
  problem text    = {Você consegue acertar o ponto vital do grupo branco?},
  answer text one = {\emph{Correto.} Preto 1 é o ponto-chave para separar o espaço vital em 2.},
  answer text two = {\emph{Incorreto.} Branco ainda consegue dividir seu espaço de olho.},
]

\problemAnswerDiagram[%
  sgf folder      = vida_ou_morte/formas_mortas,
  sgf filename    = formas_mortas.8,
  problem text    = {Preto possui movimentos forçados contra Branco.},
  answer text one = {\emph{Correto.} O atari ajuda a criar uma forma morta.},
  answer text two = {\emph{Variação.} Simplesmente subir também funciona, pois Branco não consegue fazer o auto-atari de A.},
]

\problemAnswerDiagram[%
  sgf folder      = vida_ou_morte/formas_mortas,
  sgf filename    = formas_mortas.4,
  problem text    = {Branco possui um problema de falta de liberdades.},
  answer text one = {\emph{Correto.} Branco não consegue fazer atari em 3 pois seria auto-atari.},
  answer text two = {\emph{Incorreto.} Branco ainda consegue dividir seu espaço de olho.},
]

\problemAnswerDiagram[%
  sgf folder      = vida_ou_morte/formas_mortas,
  sgf filename    = formas_mortas.7,
  problem text    = {Agora, Branco possui um espaço vital de ``quatro quadrado''. É possível viver com este espaço?},
  answer text one = {\emph{Correto.} Não, não é possível viver com o espaço de 4 quadrado, Branco já está morto.},
  answer text two = {\emph{Variação.} Preto impossibilita a criação de 2 olhos.},
]

\problemAnswerDiagram[%
  sgf folder      = vida_ou_morte/formas_mortas,
  sgf filename    = formas_mortas.13,
  problem text    = {Preto pode fazer tenuki? Quantos?},
  answer text one = {\emph{Correto.} Não, não é possível fazer tenuki. Tanto o grupo preto quanto o canto branco possuem 3 liberdades, ou seja, quem jogar primeiro, ganha. Preto 3 em A também funcionaria, assim como Preto 1 em 3, o que é simétrico.},
  answer text two = {\emph{Incorreto.} Se Branco conseguir jogar primeiro, Preto será capturado..},
]

\problemAnswerDiagram[%
  sgf folder      = vida_ou_morte/formas_mortas,
  sgf filename    = formas_mortas.9,
  problem text    = {Branco possui uma outra variação de espaço vital de 4 intersecções agora.},
  answer text one = {\emph{Correto.} Ao acertar no centro deste espaço de 4 intersecções, Preto mata o grupo inteiro.},
  answer text two = {\emph{Variação.} Branco poderia viver, se fosse seu turno.},
]

\problemAnswerDiagram[%
  sgf folder      = vida_ou_morte/formas_mortas,
  sgf filename    = formas_mortas.10,
  problem text    = {Agora, chegamos à praticamente a última forma morta que aparece comumente em partidas e problemas de vida ou morte.},
  answer text one = {\emph{Correto.} Esta forma morta é chamada de ``bulky five'' em inglês, ou ``cinco pesado'', ``cinco corpulento'', em português. Seu ponto fraco é Preto 1.},
  answer text two = {\emph{Incorreto.} Branco faz um olho a partir de 2, e Preto não consegue prosseguir.},
]

\problemAnswerDiagram[%
  sgf folder      = vida_ou_morte/formas_mortas,
  sgf filename    = formas_mortas.14,
  problem text    = {Esta corrida de liberdades não é nada fácil. Ter maestria sobre essas corridas de formas mortas é algo que levará um tempo e repetição. Preto consegue fazer tenuki? Talvez Preto já até esteja morto?},
  answer text one = {\emph{Correto.} Corridas de liberdades com formas mortas são notoriamente recursivas. Dentro de uma forma morta de 5, temos que lidar com uma de 4, e depois uma de 3.},
  answer text two = {\emph{Continuação.} Daqui em diante, a corrida se reduz a uma que já examinamos. A forma morta de 5 toma, então, 8 movimentos para ser capturada (sem contar com o primeiro movimento jogado em 1).},
]

\problemAnswerDiagram[%
  sgf folder      = vida_ou_morte/formas_mortas,
  sgf filename    = formas_mortas.15,
  problem text    = {Uma outra variação de forma morta de 5 intersecções. Como matar?},
  answer text one = {\emph{Correto.} Outro caso de o ponto de simetria, ou do meio, sendo o ponto-chave.},
  answer text two = {\emph{Variação.} O ponto-chave para matar é também o de viver.},
]

\problemAnswerDiagram[%
  sgf folder      = vida_ou_morte/formas_mortas,
  sgf filename    = formas_mortas.16,
  problem text    = {E se Branco tiver uma intersecção a mais?},
  answer text one = {\emph{Correto.} A maioria dos jogadores ainda consideraria Preto 1 como ``jogar no meio ou no centro da forma''.},
  answer text two = {\emph{Variação.} Branco pode utilizar o mesmo raciocínio para viver.},
]

\problemAnswerDiagram[%
  sgf folder      = vida_ou_morte/formas_mortas,
  sgf filename    = formas_mortas.18,
  problem text    = {Preto possui mais de uma maneira de resistir, mas somente uma é a correta.},
  answer text one = {\emph{Correto.} Outro problema simétrico. E, em tais problemas, o eixo de simetria é um bom indício. Com 1, Preto bloqueia os dois lados do canto em boa forma.},
  answer text two = {\emph{Incorreto.} Preto pode fazer um ko, o que é bastante impressionante como resistência, mas nada ideal neste caso.},
]

\problemAnswerDiagram[%
  sgf folder      = vida_ou_morte/formas_mortas,
  sgf filename    = formas_mortas.19,
  problem text    = {Será que Preto consegue matar sem ko?},
  answer text one = {\emph{Correto.} Fazer uma forma morta aqui é o suficiente.},
  answer text two = {\emph{Incorreto.} Parece que Preto possui um ko, mas Branco termina por espremer o grupo internamente, sem gerar uma forma morta.},
]

\problemAnswerDiagram[%
  sgf folder      = vida_ou_morte/formas_mortas,
  sgf filename    = formas_mortas.20,
  problem text    = {Esta forma aparece ocasionalmente em partidas e é um dos tsumegos mais essenciais. Estudaremos este padrão mais a fundo em volumes futuros.},
  answer text one = {\emph{Correto.} Branco parece conseguir um ko, mas Preto possui liberdades externas suficientes para espremer o grupo internamente.},
  answer text two = {\emph{Variação.} Esta forma é um excelente exemplo de como tudo pode mudar dependendo das liberdades exteriores. Agora, Preto não pode mais jogar em A e terá que lutar pelo ko.},
]

\clearedpage
\clearedpage