\usepackage[toc]{glossaries}

\makeglossaries

\newglossaryentry{atari}{
  name        = atari,
  description = {Quando 1 ou mais pedras estão com somente 1 liberdade, isto é, estão prestes a serem capturadas. Também é o termo que deu o nome ao respectivo console de videogames}
}

\newglossaryentry{ko}{
  name        = ko,
  description = {Regra que determina que o tabuleiro não pode ser repetido. Ko vem originalmente do sânscrito, da Índia, através do budismo, e significa ``ciclo infinito''}
}

\newglossaryentry{ponnuki}{
  name        = ponnuki,
  description = {A configuração mínima de pedras necessário para se capturar 1 pedra, sem ser no extremo canto ou lateral}
}

\newglossaryentry{sente}{
  name        = sente,
  description = {A iniciativa. O complemento de \emph{gote}}
}

\newglossaryentry{tenuki}{
  name        = tenuki,
  description = {Ignorar um movimento local e jogar em outro lugar}
}

\newglossaryentry{tsumego}{
  name        = tsumego,
  description = {Problema isolado ou localizado no jogo de Go. O termo também é utilizado em Shogi, como ``tsume shogi''}
}