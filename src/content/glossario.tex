\usepackage[toc]{glossaries}
% From [this answer](https://tex.stackexchange.com/a/730599/64441).
\newglossarystyle{tableStyle}{
  \setglossarystyle{long}
  \renewenvironment{theglossary}
    {\begin{longtable}{p{1.5cm}p{6cm}}}
    {\end{longtable}}
}

\renewcommand{\glsnamefont}[1]{\bfseries#1}
\renewcommand{\glswrglosslocationtextfmt}[1]{\bfseries#1}
\renewcommand{\delimR}{$\boldsymbol{-}$}
\renewcommand{\delimN}{{\bfseries, }}

\makeglossaries

\newglossaryentry{atari}{
  name        = atari,
  description = {Quando 1 ou mais pedras estão com somente 1 liberdade, isto é, estão prestes a serem capturadas. Também é o termo que deu o nome ao respectivo console de videogames}
}

\newglossaryentry{dan}{
  name        = dan,
  description = {Equivalente de mestre ou faixa-preta no Go e muitos outros esportes. E, na verdade, o termo faixa-preta foi inspirado pelo Go. Para amadores, utiliza-se a abreviação ``d'', e.g. ``5d''; e, para profissionais, a abreviação ``p'', e.g. ``7p''}
}

\newglossaryentry{geta}{
  name        = geta,
  description = {Padrão de captura mais comumente conhecido como ``rede''}
}

\newglossaryentry{joseki}{
  name        = joseki,
  description = {Sequência ótima ou parelha para ambos lados}
}

\newglossaryentry{ko}{
  name        = ko,
  description = {Regra que determina que o tabuleiro não pode ser repetido. Ko vem originalmente do sânscrito, da Índia, através do budismo, e significa ``ciclo infinito''}
}

\newglossaryentry{miai}{
  name        = miai,
  description = {Duas opções equivalentes}
}

\newglossaryentry{mini-escada}{
  name        = mini-escada,
  description = {Um tipo de escada que já possui uma pedra pronta para finalizá-la em diagonal, antes de chegar à lateral.}
}

\newglossaryentry{ponnuki}{
  name        = ponnuki,
  description = {A configuração mínima de pedras necessário para se capturar 1 pedra, sem ser no extremo canto ou lateral}
}

\newglossaryentry{quebra-escada}{
  name        = quebra-escada,
  description = {Pedra que quebra o padrão de escada ou shicho}
}

\newglossaryentry{sente}{
  name        = sente,
  description = {A iniciativa. O complemento de \emph{gote}}
}

\newglossaryentry{shicho}{
  name        = shicho,
  description = {Padrão de captura conhecido mais usualmente como ``escada''}
}

\newglossaryentry{tenuki}{
  name        = tenuki,
  description = {Ignorar um movimento local e jogar em outro lugar}
}

\newglossaryentry{tsumego}{
  name        = tsumego,
  description = {Problema isolado ou localizado no jogo de Go. O termo também é utilizado em Shogi, como ``tsume shogi''}
}