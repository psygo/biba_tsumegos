\usepackage[toc]{glossaries}
% From [this answer](https://tex.stackexchange.com/a/730599/64441).
\newglossarystyle{tableStyle}{
  \setglossarystyle{long}
  \renewenvironment{theglossary}
    {\begin{longtable}{p{1.75cm}p{5.25cm}}}
    {\end{longtable}}
}

\renewcommand{\glsnamefont}[1]{\bfseries#1}
\renewcommand{\glswrglosslocationtextfmt}[1]{\bfseries#1}
\renewcommand{\delimR}{$\boldsymbol{-}$}
\renewcommand{\delimN}{{\bfseries, }}

\makeglossaries

\newglossaryentry{atari}{
  name        = atari,
  description = {Quando 1 ou mais pedras estão com somente 1 liberdade, isto é, estão prestes a serem capturadas. Também é o termo que deu o nome ao respectivo console de videogames}
}

\newglossaryentry{boca-de-tigre}{
  name        = boca-de-tigre,
  description = {Forma de três pedras geralmente utilizada para eficiente criar uma conexão}
}

\newglossaryentry{casco de tartaruga}{
  name        = casco de tartaruga,
  description = {O número mínimo de pedras para se capturar um grupo de 2 pedras advesárias. Assim como o ponnuki, é uma forma extremamente forte}
}

\newglossaryentry{crane's nest}{
  name        = crane's nest,
  description = {``Ninho do grou'' é um problema famoso de rede}
}

\newglossaryentry{dan}{
  name        = dan,
  description = {Equivalente de mestre ou faixa-preta no Go e muitos outros esportes. E, na verdade, o termo faixa-preta foi inspirado pelo Go. Para amadores, utiliza-se a abreviação ``d'', e.g. ``5d''; e, para profissionais, a abreviação ``p'', e.g. ``7p''}
}

\newglossaryentry{dumpling}{
  name        = dumpling,
  description = {``Dango'' em japonês, ``bolinho chinês'', ``cacho de uva'' em coreano, é a denominação para um amontoado de pedras, que geralmente possui poucas liberdades e faz pouco território}
}

\newglossaryentry{duplo-ko}{
  name        = duplo-ko,
  description = {tipo de vida ou forma em que se utiliza dois kos para impossibilitar a captura do grupo. kos duplos são conhecidos como ``fábricas de ameaças'', pois o outro lado pode sempre efetuar uma captura interna como ameaça de um outro ko no tabuleiro}
}

\newglossaryentry{escada frouxa}{
  name        = escada frouxa,
  description = {``Loose ladder'' é um tipo de captura em rede que utiliza uma escada aparentemente falha}
}

\newglossaryentry{escorpião}{
  name        = escorpião,
  description = {Nome dado pelos franceses à uma forma muito resiliente que acontece com frequência, especialmente na era pós-IA}
}

\newglossaryentry{geta}{
  name        = geta,
  description = {Padrão de captura mais comumente conhecido como ``rede''. Em inglês, é conhecido como ``net''}
}

\newglossaryentry{joseki}{
  name        = joseki,
  description = {Sequência ótima ou parelha para ambos lados}
}

\newglossaryentry{keima}{
  name        = keima,
  description = {Movimento do cavaleiro, ou o ``L'' do cavalo no xadrez}
}

\newglossaryentry{ko}{
  name        = ko,
  description = {Regra que determina que o tabuleiro não pode ser repetido. Ko vem originalmente do sânscrito, da Índia, através do budismo, e significa ``ciclo infinito''}
}

\newglossaryentry{miai}{
  name        = miai,
  description = {Duas opções equivalentes}
}

\newglossaryentry{mini-escada}{
  name        = mini-escada,
  description = {Um tipo de escada que já possui uma pedra pronta para finalizá-la em diagonal, antes de chegar à lateral}
}

\newglossaryentry{olho vs sem olho}{
  name        = olho vs sem olho,
  description = {Tipo de corrida de liberdade em que, tipicamente, quem tem olho, ganha}
}

\newglossaryentry{ponnuki}{
  name        = ponnuki,
  description = {A configuração mínima de pedras necessário para se capturar 1 pedra, sem ser no extremo canto ou lateral}
}

\newglossaryentry{pulo do elefante}{
  name        = pulo do elefante,
  description = {Pulo de 1 espaço na diagonal. Este termo vem do \emph{xiangqi}, o xadrez chinês, em que a peça elefante se move dessa maneira}
}

\newglossaryentry{quebra-escada}{
  name        = quebra-escada,
  description = {Pedra que quebra o padrão de escada ou shicho}
}

\newglossaryentry{semeai}{
  name        = semeai,
  description = {Corrida de captura em português}
}

\newglossaryentry{sente}{
  name        = sente,
  description = {A iniciativa. O complemento de \emph{gote}}
}

\newglossaryentry{shicho}{
  name        = shicho,
  description = {Padrão de captura conhecido mais usualmente como ``escada''}
}

\newglossaryentry{snapback}{
  name        = snapback,
  description = {Traduzido aqui como ``ricochete'', é uma técnica de captura ou espremida que utiliza o fato de que a captura de um sacrifício reduz as liberdades adversárias}
}

\newglossaryentry{squeeze}{
  name        = squeeze,
  description = {Uma ``espremida'', o que se manifesta geralmente com uma diminuição drástica das liberdades adversárias}
}

\newglossaryentry{tenuki}{
  name        = tenuki,
  description = {Ignorar um movimento local e jogar em outro lugar}
}

\newglossaryentry{tesuji}{
  name        = tesuji,
  description = {Movimento brilhante}
}

\newglossaryentry{tsumego}{
  name        = tsumego,
  description = {Problema isolado ou localizado no jogo de Go. O termo também é utilizado em Shogi, como ``tsume shogi''}
}

\newglossaryentry{warikomi}{
  name        = warikomi,
  description = {Em inglês, ``wedge'' é uma jogada entre as pedras adversárias, para cortá-las}
}