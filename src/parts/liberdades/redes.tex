\chapter{Redes}

\emptypage

\problemAnswerDiagram
  {liberdades/redes}
  {redes.1}
  {É possível capturar a pedra branca incondicionalmente com somente um movimento? Tente começar com os movimentos mais justos possíveis e, depois, extrapole para movimentos mais distantes.}
  {\emph{Correto.} Redes já são um tópico bastante complexo para quem está começando. Com o movimento 1, Preto aprisiona a pedra branca, não há escapatória, e nem como ganhar mais liberdades.}
  {\emph{Incorreto.} Capturar assim pode ser uma opção em alguns casos específicos, mas, em geral, não é ideal pois é uma captura condicional por escada. Jogar em A também configuraria uma escada, mas em outra direção.}

\problemAnswerDiagram
  {liberdades/redes}
  {redes.4}
  {O grupo preto mais abaixo está um pouco pressionado pelo grupo branco do canto, isso muda a rede a ser aplicada?}
  {\emph{Correto.} O grupo preto possui 3 liberdades, o que é o suficiente para se garantir uma rede padrão.}
  {\emph{Incorreto.} No geral, 1 é incorreto, pois acaba em uma escada que vai para o lado. Mas escadas indo para o lado podem ser ocasionalmente melhores do que uma rede.}

\problemAnswerDiagram
  {liberdades/redes}
  {redes.3}
  {A parede branca à esquerda parece dificultar as coisas para o Preto. Ou não?}
  {\emph{Correto.} Não muda em nada em relação à rede do problema 1.}
  {\emph{Incorreto.} Não há escada!}

\problemAnswerDiagram
  {liberdades/redes}
  {redes.7}
  {Algo não parece estar certo na forma branca.}
  {\emph{Correto.} A rede não funciona! Preto resgata sua pedra, retira o segundo olho branco e, assim, mata o grupo todo.}
  {\emph{Incorreto.} Preto presenteia Branco com o necessário segundo olho.}

\problemAnswerDiagram
  {liberdades/redes}
  {redes.8}
  {Preto possui vários cortes e mais de um grupo, uma situação crítica.}
  {\emph{Correto.} Preto captura as pedras de corte, conectando seus grupos e corrigindo as falhas de A, B e C, o que reverte a situação para uma crise branca agora.}
  {\emph{Incorreto.} Se Branco tiver a chance, reforçar a com 1, demole todas as esperanças pretas localmente.}

\problemAnswerDiagram
  {liberdades/redes}
  {redes.9}
  {Branco adicionou uma troca de 1 por 2 em algum momento. Será que uma rede ainda é possível?}
  {\emph{Correto.} Sim. Não é fácil ver. Mas o exterior preto permanece forte o suficiente para capturar as duas pedras brancas.}
  {\emph{Incorreto.} Branco pode escapar. No caso escapar com A é melhor, pois escapar com 1 leva a um ``squeeze'', isto é, Preto sacrifica as pedras para obter uma força excelente no exterior.}
  
\problemAnswerDiagram
  {liberdades/redes}
  {redes.10}
  {Preto possui dois grupos cortados. Há salvação?}
  {\emph{Correto.} Capturar as pedras de corte é praticamente um milagre em uma situação como esta.}
  {\emph{Incorreto.} Branco 1 é um movimento simples e, em geral, é boa técnica atravessar o \emph{keima} --- movimento do cavaleiro, ou o ``L'' do cavalo no xadrez --- de A para B do adversário, mas, desta vez, não funciona.}

\problemAnswerDiagram
  {liberdades/redes}
  {redes.2}
  {Capturar pedras de corte é frequentemente algo de extrema importância no Go.}
  {\emph{Correto.} Novamente, Preto não consegue aumentar suas liberdades ou utilizar fraquezas no exterior preto para poder escapar.}
  {\emph{Incorreto.} Seja esta escada ou a de A, ambas são capturas condicionais, o que é quase sempre subótimo.}

\problemAnswerDiagram
  {liberdades/redes}
  {redes.5}
  {Esta já é uma rede bem mais avançada.}
  {\emph{Correto.} O importante é que Branco não vai conseguir mais de 2 liberdades com a pedra de corte, o que é menos do que as pedras pretas que a cercam.}
  {\emph{Incorreto.} Às vezes, 1 pode sera a rede ótima, mas, neste caso, Branco está muito forte no exterior.}

\problemAnswerDiagram
  {liberdades/redes}
  {redes.6}
  {As pedras pretas cortadas estão com pouquíssimas liberdades. Mesmo assim, ainda é possível fazer algo.}
  {\emph{Correto.} A borda do tabuleiro ajuda muito.}
  {\emph{Incorreto.} Não há escada, e o motivo principal é que, ao fugir, Branco coloca as pedras pretas sob atari.}