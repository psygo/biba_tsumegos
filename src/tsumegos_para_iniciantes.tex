%-----------------------------------------------------------
% Size Configs

\documentclass{book}

\usepackage[paperheight   = 148mm, % ~A6
            paperwidth    = 105mm,
            bindingoffset = 5mm,
            left          = 3.75mm,
            right         = 8mm,
            top           = 15.2mm,
            bottom        = 13.2mm,
            footskip      = 6.35mm]{geometry}

%-----------------------------------------------------------
% Fonts

\renewcommand{\footnotesize}{\normalsize}
\renewcommand{\normalsize}{\fontsize{12pt}{13pt}}

\usepackage[T1]{fontenc}
\usepackage[portuguese]{babel}
\usepackage{microtype}

\usepackage{lmodern}

\usepackage{fontspec}
% From [this comment](https://www.reddit.com/r/identifythisfont/comments/1fiahvg/comment/lngfilr/?utm_source=share&utm_medium=web3x&utm_name=web3xcss&utm_term=1&utm_content=share_button)
\setmainfont{Adobe Garamond Pro}

%-----------------------------------------------------------
% URLs

\usepackage[colorlinks  = true,
            allcolors   = black,
            pdfsubject  = {Go; Baduk; Weiqi},
            pdftitle    = {Tsumegos para Inciantes - Volume 1 - Captura e Vida ou Morte},
            pdfauthor   = {Philippe Fanaro},
            pdfkeywords = {Go; Baduk; Weiqi; Tsumegos},
            pdfproducer = {LaTeX},
            pdfcreator  = {pdflatex; bibtex},
            bookmarksnumbered]{hyperref}
\usepackage{xurl}

%-----------------------------------------------------------
% TOC

\usepackage{tocloft}
\renewcommand\cftchapfont{\mdseries}
\renewcommand\cftchappagefont{\mdseries}
\renewcommand{\cfttoctitlefont}{\bfseries\huge}
\setlength{\cftbeforetoctitleskip}{-25pt}
\setlength{\cftaftertoctitleskip}{15pt}

\usepackage[nottoc,numbib]{tocbibind}
\addto\captionsportuguese{
  \renewcommand{\contentsname}{Índice}
}

%-----------------------------------------------------------
% Chapter and Part Titles

\usepackage{titlesec}
\titleformat{\chapter}[display]
            {\normalfont\huge\bfseries}
            {\thechapter}
            {20pt}
            {\huge}
\titlespacing*{\chapter}
              {0pt}
              {-20pt}
              {20pt}

\renewcommand{\chaptermark}[1]{%
  \markboth{\thechapter.\ #1}{}}
\renewcommand{\partname}{}

% From [this question](https://tex.stackexchange.com/q/726389/64441)
\usepackage{xpatch}
\AtBeginDocument{\xpatchcmd{\NR@part}{\partname\nobreakspace}{}{}{}} % Removing "Part" from part titles
\pagestyle{headings} % Apply the change
\makeatother

%-----------------------------------------------------------
% Page Header

% From [this question](https://tex.stackexchange.com/q/726389/64441)
\usepackage{fancyhdr}
\fancyhead{}
\fancyhead[LE, RO]{\slshape\nouppercase\rightmark}
\fancyhead[LO,RE]{\selectfont\slshape\nouppercase\leftmark}
\fancyhead[LE,RO]{\thepage}
\fancyfoot{}
\setlength{\headheight}{12.80502pt}
\renewcommand\headrulewidth{0pt}
\pagestyle{fancy}
\makeatletter
\renewcommand\chaptermark[1]{\markboth{\if@mainmatter\thechapter. \fi#1}{}}
\makeatother

%-----------------------------------------------------------
% Figures

% Forcing figure floats into being at the top of the page.
% From [this answer](https://tex.stackexchange.com/a/66270/64441)
\makeatletter
\setlength{\@fptop}{0.0cm}
\makeatother

%-----------------------------------------------------------
% Other Stuff

\raggedbottom
\frenchspacing
\usepackage{indentfirst}

%-----------------------------------------------------------
% Custom Packages

\usepackage{./goban_segletes/goban}
\usepackage{./diagrams/diagrams}
\usepackage{./utils}

%-----------------------------------------------------------

\begin{document}
  \newcommand\coverRuler{
  \rule{\textwidth}{1.6pt}\vspace*{-\baselineskip}\vspace*{2pt}
  \rule{\textwidth}{0.4pt}
}

\begin{titlepage}
  \centering
  
  \scshape % Small Caps

  \vspace*{-1.5cm}
  
  %---------------------------------------------------------
  % Título
  
  \coverRuler
  
  \vspace{0.7\baselineskip}
  
  \huge{Tsumegos para Iniciantes}\\
  \vspace*{10pt}
  \large{Volume 1 | 30-20 kyu}\\
  \vspace*{10pt}
  \Large{Liberdades e Vida ou Morte}

  \vspace{0.275\baselineskip}
  
  \coverRuler

  %---------------------------------------------------------

  \vspace*{-0.0675cm}

  %---------------------------------------------------------
  % Imagem
  
  \begin{figure}[h]
    \centering
    \hbox{
      \hspace{0.6cm}
      \begin{goban}[board dimension       = 6,
                    board size            = 9,
                    scale                 = 1,
                    outline line width    = 0.45mm,
                    horizontal clip start = 1,
                    horizontal clip end   = 9,
                    vertical clip start   = 1,
                    vertical clip end     = 9]
        \parseSgfFile{sgf/cover_9x9.sgf}
      \end{goban}
    }
  \end{figure}

  %---------------------------------------------------------
  
  \vfill

  %---------------------------------------------------------
  % Autor

  \Large{Philippe Fanaro}

  %---------------------------------------------------------
\end{titlepage}
  \blankpage

  \frontmatter

  \tableofcontents

  \newcommand\shref[2]{
  \small{\href{#1}{\path{#2}}}
}

\chapter{Prefácio}
  
Há 3 pilares para se aprimorar no Go, qual seja o seu nível, desde iniciante até campeão mundial: jogar, revisar e praticar \emph{tsumegos}. Para quem não os conhece, tsumegos são problemas localizados no jogo de Go.

Evidentemente, jogando e revisando é possível praticar tsumegos, mas o volume por unidade de tempo será bem mais baixo, já que, em partidas, há muito mais com o que se preocupar. Tsumegos são o equivalente de musculação para atletas de esportes físicos: uma maneira eficiente de se treinar em algo específico relacionado ao esporte desejado.

O propósito principal deste livro e da coleção \emph{Tsumegos para Iniciantes} não é, no entanto, ser uma coletânea exaustiva de tsumegos para iniciantes. O intuito aqui é servir de referência de técnicas e padrões essenciais, além de ter um volume mínimo para que se possa ser considerado como uma sessão de treinamento. 

Caso o leitor queira um maior volume e diversidade de tsumegos, hoje em dia, não há recurso melhor do que o \shref{https://101weiqi.com}{101weiqi.com}. Outro recurso importante é o servidor OGS (\shref{https://online-go.com}{online-go.com}), que também possui muitos tsumegos, além de uma excelente página de referências: \shref{https://online-go.com/docs/other-go-resources}{online-go.com/docs/other-go-resources}.

\bigskip
\bigskip

Para quem não me conhece, sou Philippe Fanaro, e jogo Go desde o final de 2012. Ao redor de quando escrevo este livro, oscilo entre 2 e 3 dan no servidor KGS, ou algo entre 5 e 6 dan no servidor Fox.

Venho compartilhando conteúdo de Go online especialmente pelo meu canal de YouTube em português: \shref{https://youtube.com/@Fanaro}{youtube.com/@Fanaro}. Mas, também, pelo meu blog: \shref{https://fanaro.io}{fanaro.io}.

Outro conteúdo útil a mencionar é a tradução do livro \emph{Como Jogar Go --- Uma Introdução Concisa}, de Richard Bozulich e James Davies, que eu fiz em código aberto alguns anos atrás, ela está disponível aqui, cujo PDF é gratuito: \shref{https://github.com/FanaroEngineering/traducao\_como\_jogar\_go}{github.com/FanaroEngineering/traducao_como_jogar_go}.

\bigskip
\bigskip

Philippe Fanaro

12 de Setembro de 2024

\pagebreak

% TODO: add another empty page
% TODO: center vertically

\begin{center}
  \Large \emph{Sempre Preto a jogar.}
\end{center}

  \mainmatter
  \part{Liberdades}
    % \chapter{Captura}

\emptypage

\problemAnswerDiagram
  {liberdades/captura}
  {captura.1}
  {É possível capturar a pedra branca? (Em livros de tsumegos, costuma-se padronizar sempre Preto a jogar.)}
  {\emph{Correto.} Sim, é possível capturar a pedra branca diretamente, já que ela só possui 1 liberdade. (Note que a pedra branca será retirada do tabuleiro.)}
  {\emph{Variação.} Mesmo se Branco tentar resgatar ou contra-capturar a pedra preta, sua pedra do canto ainda possui somente 1 liberdade e pode ser capturada.}
  
\problemAnswerDiagram
  {liberdades/captura}
  {captura.2}
  {E se a pedra branca estiver no lado ao invés do canto, ainda é possível capturá-la?}
  {\emph{Correto.} Branco não deveria jogar 2, pois não há para onde fugir. Suas pedras terão sempre 1 liberdade, e vão colidir com o canto esquerdo no final.}
  {\emph{Variação.} Preto pode sempre fazer \emph{tenuki}, isto é, ignorar movimentos locais e jogar em outro lugar, mas aí é Branco quem poderá capturar a pedra preta.}

\problemAnswerDiagram
  {liberdades/captura}
  {captura.3}
  {Agora a pedra branca está no centro. Ela ainda pode ser capturada?}
  {\emph{Correto.} A captura da pedra branca configura uma forma chamada \emph{ponnuki}, que é o número mínimo de pedras para se capturar uma pedra adversária, quando ela não está nem no canto e nem na borda. (Note que a pedra branca será retirada do tabuleiro.)}
  {\emph{Variação.} Ao escapar com esta pedra, Branco vai de 1 liberdade para 3, o que é bastante eficiente, isto é, 2 liberdades por movimento; em lutas, liberdades são talvez o bem mais crucial. Desta maneira, Branco também expõe os cortes A e B no exterior preto.}

\problemAnswerDiagram
  {liberdades/captura}
  {captura.8}
  {Há uma pedra branca solitária no topo. O que Preto pode fazer com ela?}
  {\emph{Correto.} Preto pode capturá-la e, assim, resolver o problem do corte que existia em A.}
  {\emph{Incorreto.} Preto poderia ter capturado a pedra, o que resolveria o corte de uma maneira mais eficiente, sem jogar em seu próprio território. E esta defesa dá chances para que Branco conecte todas as suas pedras mais tarde.}

\problemAnswerDiagram
  {liberdades/captura}
  {captura.9}
  {Capturas no canto são geralmente de extremo valor pois podem garantir não somente território, mas, também, o espaço vital de grupos, como veremos mais tarde nos exercícios de vida ou morte deste livro.}
  {\emph{Correto.} A captura da pedra do canto limpa o canto para Preto, e ainda deixa a pedra A com somente 1 liberdade.}
  {\emph{Incorreto.} Localmente pelo menos, não jogar aqui como Preto ajuda Branco a se estabilizar localmente, enquanto contra-ataca severamente.}

\problemAnswerDiagram
  {liberdades/captura}
  {captura.7}
  {O grupo branco mais ao topo possui 2 pedras. Isso muda algo em relação à possibilidade de captura?}
  {\emph{Correto.} Apesar de o grupo branco ter mais pedras, ele ainda possui somente 1 liberdade. Esta forma é conhecida como ``casco de tartaruga'', que é o número mínimo de pedras necessário para se capturar 2 pedras. Também era possível capturar com A. (Note que ambas as pedras brancas serão retiradas do tabuleiro.)}
  {\emph{Variação.} Ao fugir, Branco não somente resgata suas pedras como expõe múltiplas fraquezas no exterior preto.}

\problemAnswerDiagram
  {liberdades/captura}
  {captura.10}
  {Ambos os lados estão, localmente, em uma situação crítica.}
  {\emph{Correto.} Preto captura dois pontos, praticamente garante o canto, e a pedra marcada é indiretamente engolida.}
  {\emph{Variação.} Branco reverte para uma captura para si, e o grupo preto agora está flutuando e instável.}

\problemAnswerDiagram
  {liberdades/captura}
  {captura.23}
  {Há mais de 1 grupo preto prestes a ser capturado.}
  {\emph{Correto.} Preto captura as principais pedras de corte e resolve todos os seus problemas.}
  {\emph{Incorreto.} Resgatar pedras diretamente é muitas vezes o melhor momento. Mas por que resgatar se podemos corrigir capturando diretamente?}

\problemAnswerDiagram
  {liberdades/captura}
  {captura.24}
  {Preto já está basicamente seguro, mas uma captura nesta região não somente seriam pontos, mas, também, um possível ataque contra Branco.}
  {\emph{Correto.} Preto captura tudo basicamente.}
  {\emph{Incorreto.} Se Branco tiver a chance, com 1, ele basicamente estará vivo, pois a pedra A é engolida automaticamente. Além disso, o corte em B agora é possível.}

\clearedpage
\clearedpage

% \chapter{Resgate}

% \emptypage

% \chapter{Duplo-Atari}

% \emptypage

% \problemAnswerDiagram
%   {captura}
%   {captura.25}
%   {É difícil imaginar uma situação no começo do jogo em que Preto não responderia neste local.}
%   {\emph{Correto.} Além de Preto limpar os problemas de corte de sua forma, a pedra branca marcada é completamente inutilizada.}
%   {\emph{Incorreto.} Ao conectar suas pedras, Branco expõe múltiplos cortes e disconecta a pedra A.}

% \chapter{Suicídio}

% \emptypage

% \problemAnswerDiagram
%   {captura}
%   {captura.4}
%   {Preto pode conectar suas pedras?}
%   {\emph{Correto.} Conectar as pedras zeraria as liberdades de todas as pedras do grupo, o que as automaticamente capturaria, ou seja, seria suicídio, uma jogada inválida.}
%   {\emph{Variação.} Branco tem sempre a opção de capturar ambas as pedras.}

% \problemAnswerDiagram
%   {captura}
%   {captura.5}
%   {Preto pode capturar as pedras brancas?}
%   {\emph{Correto.} Note que não é suicídio jogar em 1, pois a regra da captura possui precedência. Isto é, primeiro aplicamos a regra da captura, se possível, e, só depois, examinamos se é suicídio. E segue que, se algo for capturado, haverá mais de uma liberdade.}
%   {\emph{Variação.} Mais tarde, se Branco conseguir pedras no exterior, quem pode ser capturado é o Preto! Antes de 1, Preto poderia finalmente capturar o canto, e, dessa maneira, evitar de ser capturado.}

% \chapter{Recaptura}

% \emptypage

% \problemAnswerDiagram
%   {captura}
%   {captura.6}
%   {É possível capturar algo branco? Ou é suicídio?}
%   {\emph{Correto.} Sim, é possível capturar duas pedras brancas. Em seguida, a pedra preta que efetua a captura estará imediatamente em \emph{atari}, isto é, ela possuirá somente 1 liberdade, então Branco terá a opção de uma contra- ou recaptura.}
%   {\emph{Variação.} Caso Preto opte por não capturar, Branco pode salvar as pedras e deixar Preto com fraquezas no exterior. Isso não quer dizer que não capturar é um erro, pois pode haver outros movimentos mais importantes no tabuleiro.}

% \chapter{Ko}

% \emptypage

% \chapter{Captura na Segunda Linha}

% \emptypage

% \problemAnswerDiagram
%   {captura}
%   {captura.11}
%   {Localmente, uma situação assim deveria ser o equivalente a alarmes soando em uma base militar.}
%   {\emph{Correto.} Preto estabiliza seu grupo e fragiliza completamente as pedras brancas.}
%   {\emph{Incorreto.} Branco reverte a situação, e é agora Preto quem está desmoronando.}

% \problemAnswerDiagram
%   {captura}
%   {captura.13}
%   {Um problema similar ao anterior. Você jogaria no mesmo lugar?}
%   {\emph{Correto.} Desta vez, capturamos por fora, pois, caso contrário...}
%   {\emph{Incorreto.} Ao fazer atari por fora, as pedras pretas ficam em atari.}

% \problemAnswerDiagram
%   {captura}
%   {captura.14}
%   {Esta é a situação que gera, com frequência, os cenários dos 2 problemas anteriores. Este é um padrão bastante comum no Go.}
%   {\emph{Correto.} Preto captura a pedra da segunda linha pois ela não tem como estender suas liberdades. Este padrão é um dos principais motivos por invasões na segunda linha raramente funcionarem.}
%   {\emph{Variação.} Branco não tem como salvar suas pedras.}

% \chapter{Pedras de Corte}

% \emptypage

% \problemAnswerDiagram
%   {captura}
%   {captura.12}
%   {Salvar as pedras pretas não somente é uma quantia considerável de pontos, mas uma maneira de contra-atacar.}
%   {\emph{Correto.} Com 2 e 4, Branco consegue espremer Preto --- em inglês, esta técnica é conhecida como ``squeeze'' ---, mas isso ainda não corrige os cortes de A a E.}
%   {\emph{Incorreto.} Preto 2 parece ser uma rede --- veremos esta técnica um pouco mais à frente ---, e uma forma mais bonita e eficiente, mas as pedras pretas não possuem liberdades suficientes para capturar em rede.}

% \problemAnswerDiagram
%   {captura}
%   {captura.15}
%   {Preto precisa proteger dois lados ao mesmo tempo. É mesmo possível?}
%   {\emph{Correto.} As pedras pretas A possuem liberdades o suficiente.}
%   {\emph{Incorreto.} Branco não só garante o canto como o mata o grupo inteiro.}

% \problemAnswerDiagram
%   {captura}
%   {captura.22}
%   {Uma situação bastante confusa, com múltiplos grupos cortados.}
%   {\emph{Correto.} Com esta captura, Preto gera liberdades para o grupo A, que estava em estado crítico, e também basicamente captura as pedras marcadas.}
%   {\emph{Incorreto.} Primeiramente, jogar em 1 é auto-atari --- quando o próprio jogador se põe em atari, ``self-atari'' em inglês --- nas pedras A. Mas Branco pode ir além e capturar as pedras do topo.}

% \problemAnswerDiagram
%   {captura}
%   {captura.19}
%   {Preto está quase conseguindo conectar seus grupos.}
%   {\emph{Correto.} Branco não consegue fugir.}
%   {\emph{Incorreto.} A intenção foi boa, mas a execução falhou.}

% \problemAnswerDiagram
%   {captura}
%   {captura.20}
%   {Preto pode se fortalecer um pouco mais enquanto ataca Branco.}
%   {\emph{Correto.} Preto retira o olho branco, garante mais pontos, e ainda protege um de seus cortes.}
%   {\emph{Incorreto.} Mais tarde, ou mesmo em breve, Branco pode garantir um olho e expor os cortes A e B ao mesmo tempo.}

% \problemAnswerDiagram
%   {captura}
%   {captura.21}
%   {É imprescindível contar as liberdades dos grupos que participam de lutas.}
%   {\emph{Correto.} Preto consegue capturar as pedras de corte.}
%   {\emph{Incorreto.} Em geral, quando uma sequência não funciona, o melhor é simplesmente não jogá-la. Se jogamos 1 e percebemos que não funciona, é melhor não jogar 3 ou 5. No mínimo, poderíamos utilizar 1, 3 e 5 como ameaças de ko no futuro.}

% \chapter{Conectar e Morrer}

% \emptypage

% \problemAnswerDiagram
%   {captura}
%   {captura.17}
%   {Branco parece seguro, mas há um problema gigante em sua forma.}
%   {\emph{Correto.} Se Branco conectar em 2, é capturado. Esta técnica é conhecida como ``connect and die'', ou ``conectar e morrer'', apesar de que muitos diriam que é mais um problema de falta de liberdades.}
%   {\emph{Incorreto.} Em algum momento, Branco pode jogar em 1 para viver.}

% \problemAnswerDiagram
%   {captura}
%   {captura.18}
%   {Preto precisa de um milagre, e rápido.}
%   {\emph{Correto.} Branco não consegue conectar em 3, pois a pedra preta em A garante um auto-atari. Este é um exemplo muito mais emblemático de ``conectar e morrer''.}
%   {\emph{Incorreto.} Capturar em 2 corrige os problemas para Branco.}

% \problemAnswerDiagram
%   {captura}
%   {captura.16}
%   {Mais uma situação extremamente suspeita em termos de liberdades.}
%   {\emph{Correto.} Branco não consegue conectar em 3 diretamente pois seria auto-atari!}
%   {\emph{Incorreto.} Caso Preto não perceba o problema, Branco toma o bom movimento preto para si. Branco A também funcionaria, mas essencialmente tem o mesmo efeito no canto nesta situação, além de retirar uma liberdade das pedras pretas no exterior.}

% \chapter{Corridas de Captura}

% \emptypage

    \chapter{Escadas}

\emptypage

\problemAnswerDiagram[%
  sgf folder      = liberdades/escadas,
  sgf filename    = escadas.1,
  problem text    = {Será que há mais de uma maneira de se capturar a pedra branca? (Em problemas de escada, pressupõe-se que não há outras pedras no resto do tabuleiro.)},
  answer text one = {\emph{Correto.} Este padrão é conhecido como \emph{\gls{shicho}} em japonês, ou ``escada'' em português. Ao chegar na borda do tabuleiro, Branco não conseguirá mais estender suas liberdades. Tanto A quanto B funcionarão para completar a captura quando chegamos à segunda linha.},
  answer text two = {\emph{Incorreto.} Se a configuração do problema estivesse deslocada uma linha para a direita, jogar 1 seria uma opção, mas, nesta situação, jogar 1 é um desastre.},
]

\problemAnswerDiagram[%
  sgf folder      = liberdades/escadas,
  sgf filename    = escadas.2,
  problem text    = {Agora, Branco possui uma pedra no caminho da escada. Isso muda alguma coisa?},
  answer text one = {\emph{Correto.} O melhor que Preto pode fazer em uma situação destas, localmente, é um compromisso ou negociação. Pedras adversárias que impossibilitam a escada são chamadas de ``quebra-escadas''. Para Branco, teria sido melhor ser mais ambicioso, atacando em larga escala com A, ao invés de 1.},
  answer text two = {\emph{Incorreto.} A pedra branca A ajudará a estender as liberdades daquelas em escada, impossibilitando sua conclusão. Caso a escada não funcione, jogar o padrão é um dano indireto pois Preto fica com múltiplos ataris duplos no exterior.},
]

\problemAnswerDiagram[%
  sgf folder      = liberdades/escadas,
  sgf filename    = escadas.3,
  problem text    = {Em relação ao problema anterior, a pedra branca agora está deslocada para o topo. A escada funciona desta vez?},
  answer text one = {\emph{Correto.} Novamente, o melhor que Preto pode fazer é um compromisso. Não há tanto o que reclamar, pois a pedra branca marcada foi bastante danificada, atravessada em relação a 2, um tipo de dano que examinaremos mais à frente. Branco pode decidir jogar de outra maneira, caso a pedra branca marcada seja importante.},
  answer text two = {\emph{Variação.} Para evitar o dano à sua pedra exterior, Branco deveria jogar em 2 a-qui na verdade. Sequências de compromisso dependem muito do tabuleiro global, mas uma possibilidade para Preto é esta.},
]

\problemAnswerDiagram[%
  sgf folder      = liberdades/escadas,
  sgf filename    = escadas.6,
  problem text    = {E agora? Em algum momento, houve a troca de 1 por 2. A escada funciona ou não?},
  answer text one = {\emph{Correto.} Em geral, a pedra Branca que quer quebrar a escada precisa ter 4 liberdades. Afinal, ela vai herdar a falta de liberdades do grupo sob a escada.},
  answer text two = {\emph{Incorreto.} Forçar a escada para a outra direção é muito raramente uma boa ideia pois as pedras à direita ficarão ou muito fragilizadas ou diretamente sob atari.},
]

\problemAnswerDiagram[%
  sgf folder      = liberdades/escadas,
  sgf filename    = escadas.9,
  problem text    = {A troca do problema anterior foi ligeiramente modificada. É escada ou não?},
  answer text one = {\emph{Correto.} A escada não funciona desta vez. Como diria Toshiro Kageyama 7p, não há atalhos, é preciso ler a escada! Se Branco insistir em capturar o topo --- teria sido melhor capturar com 4 diretamente ---, podemos danificar o exterior com uma forma leve e flexível.},
  answer text two = {\emph{Variação.} Globalmente, se a partida ainda estiver no início, seria até mesmo melhor somente danificar a pedra do centro e forçar uma resposta branca com 1 ou A. Visto que o topo já foi desvalorizado, podemos jogar em outro lugar.},
]

\problemAnswerDiagram[%
  sgf folder      = liberdades/escadas,
  sgf filename    = escadas.8,
  problem text    = {Parece haver mais de uma resposta.},
  answer text one = {\emph{Correto.} A presença da pedra A completa a escada. Há alguns que chamam este tipo de forma de mini-escada.},
  answer text two = {\emph{Incorreto.} Branco poderia até capturar diretamente as pedras de corte com A ou B, mas 2 cria mais problemas para Preto. Além disso, se Branco A ou B, Preto pode jogar 2 e C para se conectar por fora.},
]

\problemAnswerDiagram[%
  sgf folder      = liberdades/escadas,
  sgf filename    = escadas.7,
  problem text    = {Este é um \emph{\gls{joseki}} --- uma sequência ótima para os dois lados --- um dos poucos que sobreviveu bem a transição para a era pós-IA. Preto possui basicamente 2 opções a seguir.},
  answer text one = {\emph{Correto.} A regra geral é capturar a pedra de corte, pois gera excelente forma e força. Preto faz o contra-atari do outro lado e captura o canto.},
  answer text two = {\emph{Variação.} Cortar deste lado é outra opção, caso Preto se interesse mais pelo exterior, e tenha a escada externa. Estes duplos cortes a partir da forma marcada configuram um padrão bastante recorrente em partidas reais, desde amadores até profissionais. Se Preto não tiver a escada, Branco pode imediatamente sair em A.},
]

\problemAnswerDiagram[%
  sgf folder               = liberdades/escadas,
  sgf filename             = escadas.4,
  problem text             = {Este problema é muito avançado. Mas é uma bela referência de como escadas aparecem comumente em josekis e lutas no meio de jogo.},
  answer text one          = {\emph{Correto.} Ao jogar em 1 e estender em 3, Preto cria  \emph{\gls{miai}} --- isto é, duas opções equivalentes --- de capturar em escada com A, ou capturar as pedras de corte à direita com B.},
  answer text two          = {\emph{Incorreto.} A pedra branca A oferece o suporte necessário para que Branco 6 cancele a escada no exterior! Em geral, no Go, a ordem dos movimentos é extremamente importante. Fazer a sequência supostamente correta fora de ordem é quase sempre errado!},
  answer diagram clip vert = 10,
]

\problemAnswerDiagram[%
  sgf folder      = liberdades/escadas,
  sgf filename    = escadas.5,
  problem text    = {Mais uma escada bastante avançada. Garanto que há jogadores de nível dan que erram este problema.},
  answer text one = {\emph{Correto.} Os ataris que Preto possui no exterior ajudam-no a armar a escada. É quase sempre melhor omitir movimentos desnecessários no Go, por isso, é melhor jogar em 5 se Branco realmente sair com 4 (o que seria um erro).},
  answer text two = {\emph{Incorreto.} Começar pelo a-tari de 1 parece gerar a mesma sequência, porém de maneira mais simples. Mas Branco terá o tempo de capturar as pedras pretas mais críticas!},
]

\problemAnswerDiagram[%
  sgf folder      = liberdades/escadas,
  sgf filename    = escadas.10,
  problem text    = {Preto acaba de encontrar uma jogada magnífica. Como continuar?},
  answer text one = {\emph{Correto.} Esta é uma escada em duas etapas, com uma mudança de direção, tudo graças aos benefícios da lateral do tabuleiro.},
  answer text two = {\emph{Continuação.} Preto muda a direção da escada para o centro.},
]
    \chapter{Redes}

\emptypage

\problemAnswerDiagram
  {redes}
  {redes.1}
  {É possível capturar a pedra branca incondicionalmente com somente um movimento?}
  {\emph{Correto.} Redes já são um tópico bastante complexo para quem está começando. Com o movimento 1, Preto aprisiona a pedra branca, não há escapatória, e nem como ganhar mais liberdades.}
  {\emph{Incorreto.} Capturar assim pode ser uma opção em alguns casos específicos, mas, em geral, não é ideal pois é uma captura condicional por escada. Jogar em A também configuraria uma escada, mas em outra direção.}

\problemAnswerDiagram
  {redes}
  {redes.2}
  {Capturar pedras de corte é frequentemente algo de extrema importância no Go.}
  {\emph{Correto.} Novamente, Preto não consegue aumentar suas liberdades ou utilizar fraquezas no exterior preto para poder escapar.}
  {\emph{Incorreto.} Seja esta escada ou a de A, ambas são capturas condicionais, o que é quase sempre subótimo.}

\problemAnswerDiagram
  {redes}
  {redes.3}
  {A parede branca à esquerda parece dificultar as coisas para o Preto. Ou não?}
  {\emph{Correto.} Não muda em nada em relação à rede do problema 1.}
  {\emph{Incorreto.} Não há escada!}

\problemAnswerDiagram
  {redes}
  {redes.4}
  {O grupo preto mais abaixo está um pouco pressionado pelo grupo branco do canto, isso muda a rede a ser aplicada?}
  {\emph{Correto.} O grupo preto possui 3 liberdades, o que é o suficiente para se garantir uma rede padrão.}
  {\emph{Incorreto.} No geral, 1 é incorreto, pois acaba em uma escada que vai para o lado. Mas escadas indo para o lado podem ser ocasionalmente melhores do que uma rede.}

\problemAnswerDiagram
  {redes}
  {redes.5}
  {Esta já é uma rede bem mais avançada.}
  {\emph{Correto.} O importante é que Branco não vai conseguir mais de 2 liberdades com a pedra de corte, o que é menos do que as pedras pretas que a cercam.}
  {\emph{Incorreto.} Às vezes, 1 pode sera a rede ótima, mas, neste caso, Branco está muito forte no exterior.}

\problemAnswerDiagram
  {redes}
  {redes.6}
  {As pedras pretas cortadas estão com pouquíssimas liberdades. Mesmo assim, ainda é possível fazer algo.}
  {\emph{Correto.} A borda do tabuleiro ajuda muito.}
  {\emph{Incorreto.} Não há escada, e o motivo principal é que, ao fugir, Branco coloca as pedras pretas sob atari.}

    % \chapter{Sacrifício}

\emptypage

  \part{Vida ou Morte}
    % \chapter{Dois Olhos}

\emptypage

\problemAnswerDiagram
  {dois_olhos}
  {dois_olhos.1}
  {Este exercício é uma das essências do Go. É possível permanecer incondicionalmente no tabuleiro?}
  {\emph{Correto.} Sim, com 1, estabelecemos ``dois olhos'', o que impossibilita a captura do grupo. Branco precisaria jogar tanto em A quanto em B para capturar, mas ele só possui um movimento por turno, portanto, cada um deles será considerado suicídio!}
  {\emph{Incorreto.} Preto talvez pense que seja possível capturar a pedra 1, mas ele morrerá primeiro. Note que, o melhor movimento do seu adversário (jogar 1 para viver como Preto) é também seu melhor movimento.}

\problemAnswerDiagram
  {dois_olhos}
  {dois_olhos.2}
  {Talvez o grupo estar na lateral, e não no canto, mude o seu status.}
  {\emph{Correto.} Não, é exatamente a mesma forma do problema anterior.}
  {\emph{Incorreto.} E o mesmo movimento para matar.}

\problemAnswerDiagram
  {dois_olhos}
  {dois_olhos.3}
  {Cercar 3 intersecções é insuficiente para viver, a não ser que seja seu turno. E se estivermos cercando de 4 a 5?}
  {\emph{Correto.} Cercar 4 é suficiente, pois é impossível que Branco faça com que todo o espaço vire algo indivisível. Se Branco jogar em 3, Preto responderá em 2.}
  {\emph{Incorreto.} Branco em 2 é uma tática ligeiramente mais avançada --- e que veremos em breve --- para criar um olho falso. Preto acaba efetivamente cercando somente 3 intersecções.}

\problemAnswerDiagram
  {dois_olhos}
  {dois_olhos.6}
  {Preto precisa de mais um movimento no canto para garantir sua vida?}
  {\emph{Correto.} Sim, é preciso reforçar, pois é preciso criar dois espaços separados.}
  {\emph{Incorreto.} Branco pode impossibilitar a separação do espaço interno, efetivamente criando um olho grande, o que é insuficiente para Preto viver.}

\problemAnswerDiagram
  {dois_olhos}
  {dois_olhos.5}
  {Esta é uma outra forma que aparece com frequência em partidas. Como completá-la para que viva?}
  {\emph{Correto.} Preto configurará um olho no extremo canto, e outro na diagonal.}
  {\emph{Incorreto.} Ao descer, Branco pega o bom movimento preto para si, e faz com que o espaço interno não possa ser dissociado.}
    % \chapter{Olhos Falsos}

\emptypage

\problemAnswerDiagram
  {vida_ou_morte/olhos_falsos}
  {olhos_falsos.4}
  {Preto protegeu parte dos olhos do canto. É o suficiente?}
  {\emph{Correto.} Preto precisa de mais uma proteção para garantir seus dois olhos.}
  {\emph{Incorreto.} Branco pode falsificar o segundo olho com um sacrifício.}

\problemAnswerDiagram
  {vida_ou_morte/olhos_falsos}
  {olhos_falsos.1}
  {Tão perto de viver quanto de morrer.}
  {\emph{Correto.} Ao conectar, Preto possui espaço interno o suficiente para criar dois olhos. Se Branco jogar em A, Preto joga em B, e vice-versa.}
  {\emph{Incorreto.} A intersecção de A vira olho falso.}

\problemAnswerDiagram
  {vida_ou_morte/olhos_falsos}
  {olhos_falsos.2}
  {Um problema bastante traiçoeiro.}
  {\emph{Correto.} Preto protege seus dois olhos, em A e B.}
  {\emph{Incorreto.} Ao jogar 2, Branco falsifica o olho de A diretamente, e Preto terá que conectar em B mais tarde, ou seja, também é olho falso!}
    % \chapter{Formas Mortas}

\emptypage

\problemAnswerDiagram[%
  sgf folder      = vida_ou_morte/formas_mortas,
  sgf filename    = formas_mortas.1,
  problem text    = {Como matar o grupo branco?},
  answer text one = {\emph{Correto.} Ao jogar no eixo de simetria, que é geralmente um bom movimento em problemas simétricos, Preto impossibilita que Branco dissocie seu espaço de olho, além de garantir que o espaço de olho seja reduzido a uma das formas mortas.},
  answer text two = {\emph{Variação.} Se Preto não jogar nada, Branco pode viver.},
]

\problemAnswerDiagram[%
  sgf folder      = vida_ou_morte/formas_mortas,
  sgf filename    = formas_mortas.2,
  problem text    = {Será que conseguimos reduzir este problema a algo mais simples?},
  answer text one = {\emph{Correto.} Ao jogar 1, Preto impossibilita a separação do espaço de olho em 2. (Se Branco jogasse em 1, estaria vivo.)},
  answer text two = {\emph{Continuação.} Preto reduzirá o espaço de olho para uma forma morta de 3 pedras. Em japonês, o termo para forma morta é \emph{nakade}.},
]

\problemAnswerDiagram[%
  sgf folder      = vida_ou_morte/formas_mortas,
  sgf filename    = formas_mortas.17,
  problem text    = {Você consegue visualizar como este problema converge a outros?},
  answer text one = {\emph{Correto.} Preto protege seus dois olhos ao mesmo tempo.},
  answer text two = {\emph{Incorreto.} Expandir o espaço interno nem sempre será a solução.},
]

\problemAnswerDiagram[%
  sgf folder      = vida_ou_morte/formas_mortas,
  sgf filename    = formas_mortas.11,
  problem text    = {E agora? Preto está cercado. Resta somente uma corrida de liberdades para capturar o canto. Será que ele vence?},
  answer text one = {\emph{Correto.} Preto ganhará a corrida por 1 liberdade.},
  answer text two = {\emph{Continuação.} Branco pode capturar as pedras no canto, mas não conseguirá aumentar suas liberdades.},
]

\problemAnswerDiagram[%
  sgf folder      = vida_ou_morte/formas_mortas,
  sgf filename    = formas_mortas.5,
  problem text    = {O ponto-chave do seu oponente é frequentemente o seu ponto-chave, mas para matar.},
  answer text one = {\emph{Correto.} No máximo, Branco conseguirá um olho grande de 2 ou 3 pedras.},
  answer text two = {\emph{Variação.} Agora, há 2 olhos.},
]

\problemAnswerDiagram[%
  sgf folder      = vida_ou_morte/formas_mortas,
  sgf filename    = formas_mortas.12,
  problem text    = {Preto consegue capturar o canto antes de ser capturado? Será que ele consegue até fazer \emph{tenuki} --- ou seja, jogar em outro lugar ignorando a situação local ---? Se sim, quantos?},
  answer text one = {\emph{Correto.} Preto consegue até jogar em outro lugar 1 vez! Branco só consegue 3 liberdades no canto, devido a uma recaptura de forma morta. Esta é uma das magias das formas mortas. É uma leitura avançada, mas é essencial ver como ela acontece, mesmo sem conseguir lê-la.},
  answer text two = {\emph{Continuação.} A corrida termina em sucesso para Preto.},
]

\problemAnswerDiagram[%
  sgf folder      = vida_ou_morte/formas_mortas,
  sgf filename    = formas_mortas.6,
  problem text    = {Talvez não pareça, mas, uma vez que formas mortas se tornarem mais intuitivas, este problema será essencialmente idêntico a vários dos anteriores.},
  answer text one = {\emph{Correto.} Novamente, uma forma morta de 3 pedras.},
  answer text two = {\emph{Incorreto.} Se Branco puder jogar, ele conseguirá dividir seu espaço interno.},
]

\problemAnswerDiagram[%
  sgf folder      = vida_ou_morte/formas_mortas,
  sgf filename    = formas_mortas.3,
  problem text    = {Você consegue acertar o ponto vital do grupo branco?},
  answer text one = {\emph{Correto.} Preto 1 é o ponto-chave para separar o espaço vital em 2.},
  answer text two = {\emph{Incorreto.} Branco ainda consegue dividir seu espaço de olho.},
]

\problemAnswerDiagram[%
  sgf folder      = vida_ou_morte/formas_mortas,
  sgf filename    = formas_mortas.8,
  problem text    = {Preto possui movimentos forçados contra Branco.},
  answer text one = {\emph{Correto.} O atari ajuda a criar uma forma morta.},
  answer text two = {\emph{Variação.} Simplesmente subir também funciona, pois Branco não consegue fazer o auto-atari de A.},
]

\problemAnswerDiagram[%
  sgf folder      = vida_ou_morte/formas_mortas,
  sgf filename    = formas_mortas.4,
  problem text    = {Branco possui um problema de falta de liberdades.},
  answer text one = {\emph{Correto.} Branco não consegue fazer atari em 3 pois seria auto-atari.},
  answer text two = {\emph{Incorreto.} Branco ainda consegue dividir seu espaço de olho.},
]

\problemAnswerDiagram[%
  sgf folder      = vida_ou_morte/formas_mortas,
  sgf filename    = formas_mortas.7,
  problem text    = {Agora, Branco possui um espaço vital de ``quatro quadrado''. É possível viver com este espaço?},
  answer text one = {\emph{Correto.} Não, não é possível viver com o espaço de 4 quadrado, Branco já está morto.},
  answer text two = {\emph{Variação.} Preto impossibilita a criação de 2 olhos.},
]

\problemAnswerDiagram[%
  sgf folder      = vida_ou_morte/formas_mortas,
  sgf filename    = formas_mortas.13,
  problem text    = {Preto pode fazer tenuki? Quantos?},
  answer text one = {\emph{Correto.} Não, não é possível fazer tenuki. Tanto o grupo preto quanto o canto branco possuem 3 liberdades, ou seja, quem jogar primeiro, ganha. Preto 3 em A também funcionaria, assim como Preto 1 em 3, o que é simétrico.},
  answer text two = {\emph{Incorreto.} Se Branco conseguir jogar primeiro, Preto será capturado..},
]

\problemAnswerDiagram[%
  sgf folder      = vida_ou_morte/formas_mortas,
  sgf filename    = formas_mortas.9,
  problem text    = {Branco possui uma outra variação de espaço vital de 4 intersecções agora.},
  answer text one = {\emph{Correto.} Ao acertar no centro deste espaço de 4 intersecções, Preto mata o grupo inteiro.},
  answer text two = {\emph{Variação.} Branco poderia viver, se fosse seu turno.},
]

\problemAnswerDiagram[%
  sgf folder      = vida_ou_morte/formas_mortas,
  sgf filename    = formas_mortas.10,
  problem text    = {Agora, chegamos à praticamente a última forma morta que aparece comumente em partidas e problemas de vida ou morte.},
  answer text one = {\emph{Correto.} Esta forma morta é chamada de ``bulky five'' em inglês, ou ``cinco pesado'', ``cinco corpulento'', em português. Seu ponto fraco é Preto 1.},
  answer text two = {\emph{Incorreto.} Branco faz um olho a partir de 2, e Preto não consegue prosseguir.},
]

\problemAnswerDiagram[%
  sgf folder      = vida_ou_morte/formas_mortas,
  sgf filename    = formas_mortas.14,
  problem text    = {Esta corrida de liberdades não é nada fácil. Ter maestria sobre estas corridas de formas mortas levará tempo e repetição. Preto consegue fazer tenuki? Talvez Preto já até esteja morto?},
  answer text one = {\emph{Correto.} Corridas de liberdades com formas mortas são notoriamente recursivas. Dentro de uma forma morta de 5, temos que lidar com uma de 4, e depois uma de 3.},
  answer text two = {\emph{Continuação.} Daqui em diante, a corrida se reduz a uma que já examinamos. A forma morta de 5 toma, então, 8 movimentos para ser capturada --- isto é, o movimento 1 e mais 7 outros.},
]

\problemAnswerDiagram[%
  sgf folder      = vida_ou_morte/formas_mortas,
  sgf filename    = formas_mortas.15,
  problem text    = {Uma outra variação de forma morta de 5 intersecções. Como matar?},
  answer text one = {\emph{Correto.} Outro caso de o ponto de simetria, ou do meio, sendo o ponto-chave.},
  answer text two = {\emph{Variação.} O ponto-chave para matar é também o de viver.},
]

\problemAnswerDiagram[%
  sgf folder      = vida_ou_morte/formas_mortas,
  sgf filename    = formas_mortas.16,
  problem text    = {E se Branco tiver uma intersecção a mais?},
  answer text one = {\emph{Correto.} A maioria dos jogadores ainda consideraria Preto 1 como ``jogar no meio ou no centro da forma''.},
  answer text two = {\emph{Variação.} Branco pode utilizar o mesmo raciocínio para viver.},
]

\problemAnswerDiagram[%
  sgf folder      = vida_ou_morte/formas_mortas,
  sgf filename    = formas_mortas.18,
  problem text    = {Preto possui mais de uma maneira de resistir, mas somente uma é a correta.},
  answer text one = {\emph{Correto.} Outro problema simétrico. E, em tais problemas, o eixo de simetria é um bom indício. Com 1, Preto bloqueia os dois lados do canto em boa forma.},
  answer text two = {\emph{Incorreto.} Preto pode fazer um ko, o que é bastante impressionante como resistência, mas nada ideal neste caso.},
]

\problemAnswerDiagram[%
  sgf folder      = vida_ou_morte/formas_mortas,
  sgf filename    = formas_mortas.19,
  problem text    = {Será que Preto consegue matar sem ko?},
  answer text one = {\emph{Correto.} Fazer uma forma morta aqui é o suficiente.},
  answer text two = {\emph{Incorreto.} Parece que Preto possui um ko, mas Branco termina por espremer o grupo internamente, sem gerar uma forma morta.},
]

\problemAnswerDiagram[%
  sgf folder      = vida_ou_morte/formas_mortas,
  sgf filename    = formas_mortas.20,
  problem text    = {Esta forma aparece ocasionalmente em partidas e é um dos tsumegos mais essenciais. Estudaremos este padrão mais a fundo em volumes futuros.},
  answer text one = {\emph{Correto.} Branco parece conseguir um ko, mas Preto possui liberdades externas suficientes para espremer o grupo internamente.},
  answer text two = {\emph{Variação.} Esta forma é um excelente exemplo de como tudo pode mudar dependendo das liberdades exteriores. Agora, Preto não pode mais jogar em A e terá que lutar pelo ko.},
]
    % \chapter{Vida Mútua}

\emptypage

\problemAnswerDiagram[%
  sgf folder      = vida_ou_morte/vida_mutua,
  sgf filename    = vida_mutua.1,
  problem text    = {Se Preto conseguir resgatar as pedras marcadas, ele fará pontos?},
  answer text one = {\emph{Correto.} Já que nenhum dos lados conseguirá jogar em A ou B, ddadoo que é auto-atari, ninguém fará pontos nesta região. Diz-se que grupos que permanecem no tabuleiro sem pontos, em geral, estão em vida mútua, ou, mais comumente, \emph{seki}.},
  answer text two = {\emph{Variação.} Se Branco conseguir conectar aqui, finalmente, ele fará pontos. No caso, serão 8 pontos.},
]

\problemAnswerDiagram[%
  sgf folder      = vida_ou_morte/vida_mutua,
  sgf filename    = vida_mutua.2,
  problem text    = {Qual é a melhor maneira de Preto resolver a situação das pedras prestes a serem capturadas?},
  answer text one = {\emph{Correto.} Ao cortar, Preto estabelece um olho e impossibilita que Branco consiga colocá-lo sob atari com A, pois seria auto-atari para Branco.},
  answer text two = {\emph{Variação.} É sempre possível fazer tenuki. Branco então terá a possibilidade de fazer 11 pontos.},
]

\problemAnswerDiagram[%
  sgf folder               = vida_ou_morte/vida_mutua,
  sgf filename             = vida_mutua.3,
  problem text             = {Aqui há mais valor do que somente um seki.},
  answer text one          = {\emph{Correto.} Se Preto jogar em A, será um seki. Mas por que não fazer pontos quando podemos capturar tudo com 1?},
  answer text two          = {\emph{Variação.} Se Branco jogar em 1, já que Preto não pode mais jogar em A, Branco capturará toda a região! Preto estaria em atari se jogasse em B nessa situação. Branco A também funcionaria, ao invés de 1.},
  answer diagram clip vert = 10,
]

\problemAnswerDiagram[%
  sgf folder      = vida_ou_morte/vida_mutua,
  sgf filename    = vida_mutua.4,
  problem text    = {No canto, sekis podem ser um pouco mais estranhos ou peculiares. Inclusive, há outras maneiras ainda mais raras de se permanecer no tabuleiro, como duplos e triplos kos, e ``vidas eternas''.},
  answer text one = {\emph{Correto.} Branco não pode fazer nada no canto pois seria auto-atari. Mas Preto não conseguirá capturar nada, então ninguém faz pontos.},
  answer text two = {\emph{Variação.} Branco joga no ponto-chave preto e estabelece um olho. Quando quiser, ele poderá jogar em A ou B para capturar as pedras pretas.},
]

\problemAnswerDiagram[%
  sgf folder      = vida_ou_morte/vida_mutua,
  sgf filename    = vida_mutua.5,
  problem text    = {Estudar formas mortas é de extrema importância.},
  answer text one = {\emph{Correto.} Preto não pode jogar em A ou B, caso contrário teríamos realmente uma forma morta de 3 pedras.},
  answer text two = {\emph{Incorreto.} Ao jogar em 1, Branco forçará Preto a se conectar em A mais tarde, devido ao atari de B. Essencialmente, a não ser em um caso específico no canto conhecido como ``bent four in the corner'', ou ``quatro curvado no canto'', a forma de 4 pedras que existiria no caso de Branco jogar A ou C não é forma morta.},
]

\problemAnswerDiagram[%
  sgf folder      = vida_ou_morte/vida_mutua,
  sgf filename    = vida_mutua.6,
  problem text    = {Preto conseguiu a pedra marcada em algum momento, o que possibilitou uma sequência de redução de fim de jogo, e Branco erra ao jogar em 4. Como punir o erro?},
  answer text one = {\emph{Correto.} Branco termina em seki. Ou seja, seu canto, que teria basicamente 6 pontos, agora foi zerado.},
  answer text two = {\emph{Variação.} O melhor teria sido aceitar a redução com 1. Assim, pelo menos, Branco fará 2 pontos.},
]

\problemAnswerDiagram[%
  sgf folder      = vida_ou_morte/vida_mutua,
  sgf filename    = vida_mutua.7,
  problem text    = {Outro seki bastante usual.},
  answer text one = {\emph{Correto.} Em problemas simétricos, frequentemente, o eixo de simetria é provavelmente a resposta, e é o caso aqui. Branco não pode jogar em A ou B pois não seria mais forma morta, então a posição permanecerá assim.},
  answer text two = {\emph{Incorreto.} Branco consegue estabelecer uma forma morta, e Preto está sob atari.},
]

\problemAnswerDiagram[%
  sgf folder      = vida_ou_morte/vida_mutua,
  sgf filename    = vida_mutua.8,
  problem text    = {É possível reduzir este problema a um que já estudamos.},
  answer text one = {\emph{Correto.} É exatamente o problema anterior.},
  answer text two = {\emph{Incorreto.} No final, Branco 4 garante que isso não será seki.},
]

\problemAnswerDiagram[%
  sgf folder      = vida_ou_morte/vida_mutua,
  sgf filename    = vida_mutua.9,
  problem text    = {Entre a vida mútua e a morte.},
  answer text one = {\emph{Correto.} Preto faz com que A ou B sejam miai para uma forma viva.},
  answer text two = {\emph{Variação.} Se Branco puder, jogar 1 garante um olho interno, e Preto é capturado indiretamente.},
]

\problemAnswerDiagram[%
  sgf folder      = vida_ou_morte/vida_mutua,
  sgf filename    = vida_mutua.10,
  problem text    = {Será que é um ko para gerar um seki?},
  answer text one = {\emph{Correto.} Se Branco 2 em 3, Preto joga em 2, e será um seki também, mas Branco terá que fazer um sacrifício em A para manter o seki, ou seja, seria uma perda de 1 ponto.},
  answer text two = {\emph{Incorreto.} Ao assegurar seu olho, Branco nem sequer precisa diretamente lutar pelo ko de imediato, Preto já está capturado.},
]

\clearedpage
\clearedpage
\end{document}