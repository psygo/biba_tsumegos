\documentclass{book}

\usepackage[utf8]{inputenc}
\usepackage[T1]{fontenc}
\usepackage[portuguese]{babel}

\usepackage[paperheight   = 5.2in, % ~A6
            paperwidth    = 4.1in,
            bindingoffset = 0.2in,
            left          = 0.2in,
            right         = 0.3in,
            top           = 0.7in,
            bottom        = 0.6in,
            footskip      = 0.25in]{geometry}

\usepackage[labelformat = simple]{subfig}
% \captionsetup[subfloat]{captionskip=10pt}
\renewcommand{\thesubfigure}{\relax} 

\usepackage{./goban/goban}
\usepackage{./diagrams}

\begin{document}
  \problemDiagrams
  \answerDiagrams%
    {A captura da pedra branca configura uma forma chamada \emph{ponnuki}, que é o número mínimo de pedras para se capturar uma pedra adversária, quando ela não está nem no canto e nem na borda. (Note que a pedra branca será retirada do tabuleiro.)}%
    {Ao escapar com esta pedra, Branco vai de 1 liberdade para 3, o que é bastante eficiente, isto é, 2 liberdades por movimento. Em lutas, liberdades são um bem crucial. Branco também expõe cortes no exterior preto.}%

  \problemDiagrams
  \answerDiagrams%
    {A captura da pedra branca configura uma forma chamada \emph{ponnuki}, que é o número mínimo de pedras para se capturar uma pedra adversária, quando ela não está nem no canto e nem na borda. (Note que a pedra branca será retirada do tabuleiro.)}%
    {Ao escapar com esta pedra, Branco vai de 1 liberdade para 3, o que é bastante eficiente, isto é, 2 liberdades por movimento. Em lutas, liberdades são um bem crucial. Branco também expõe cortes no exterior preto.}%
\end{document}